%%% -*- Mode: LaTeX; -*-

%%%#+BEGIN: bx:dblock:global:this-file :disabledP "false" :mode "auto"
\begin{whenOrg}
*  #This File: /lcnt/lgpc/bystar/permanent/engineering/blee-lcnt_EmacsConf25/bodyPresArtEnFa.tex
\end{whenOrg}
%%%#+END:

%%%#+BEGIN: bx:dblock:lcnt:latex:inputed-title :disabledP "false" :mode "auto"
\begin{whenOrg}
* [[elisp:(show-all)][(>]] [[elisp:(describe-function 'org-dblock-write:bx:dblock:lcnt:latex:inputed-title)][dbf]]
*  [[elisp:(beginning-of-buffer)][Top]] ################  [[elisp:(delete-other-windows)][(1)]]   /*PLPC-180068 -- LCNT Panel -- lcntProc.sh, presProc.sh and Mailings*/
*  [[elisp:(beginning-of-buffer)][Top]] ################  [[elisp:(delete-other-windows)][(1)]]   *Blee-LCNT: An Emacs-Centered Content Production Framework*
*  [[elisp:(beginning-of-buffer)][Top]] ################  [[elisp:(delete-other-windows)][(1)]]   http://www.by-star.net/PLPC/180068
*  [[elisp:(beginning-of-buffer)][Top]] ################  [[elisp:(delete-other-windows)][(1)]]   [[elisp:(find-file "presentationEnFa.ttytex")][Visit ./presentationEnFa.ttytex]] || [[elisp:(find-file "articleEnFa.ttytex")][Visit ./articleEnFa.ttytex]] || [[elisp:(find-file "presArtEnFa.ttytex")][Visit ./presArtEnFa.ttytex]] ||

* [[elisp:(org-shifttab)][<)]] [[elisp:(describe-function 'org-dblock-write:bx:dblock:lcnt:latex:inputed-title)][dbFunc)]]  E|
\end{whenOrg}
%%%#+END:

%%%#+BEGIN: bx:dblock:global:org-controls :disabledP "false" :mode "auto"
\begin{whenOrg}
* [[elisp:(show-all)][(>]] [[elisp:(describe-function 'org-dblock-write:bx:dblock:global:org-controls)][dbf]]
*  /Controls/ ::  [[elisp:(org-cycle)][| ]]  [[elisp:(show-all)][Show-All]]  [[elisp:(org-shifttab)][|O]]  [[elisp:(progn (org-shifttab) (org-content))][|C]] | [[file:Panel.org][Panel]] | [[elisp:(blee:ppmm:org-mode-toggle)][|N]] | [[elisp:(delete-other-windows)][|1]] | [[elisp:(progn (save-buffer) (kill-buffer))][S&Q]]  [[elisp:(save-buffer)][Save]]  [[elisp:(kill-buffer)][Quit]] [[elisp:(org-cycle)][| ]]
*  /Maintain/ ::  [[elisp:(call-interactively (quote cvs-update))][cvs-update]] | [[elisp:(bx:org:agenda:this-file-otherWin)][Agenda-This]] [[elisp:(bx:org:todo:this-file-otherWin)][ToDo-This]] | [[elisp:(bx:org:agenda:these-files-otherWin)][Agenda-These]] [[elisp:(bx:org:todo:these-files-otherWin)][ToDo-These]]

* [[elisp:(org-shifttab)][<)]] [[elisp:(describe-function 'org-dblock-write:bx:dblock:global:org-controls)][dbFunc)]]  E|
\end{whenOrg}
%%%#+END:

%%%#+BEGIN: bx:dblock:lcnt:latex:commands :disabledP "false" :class "art+pres" :langs "en+fa"
\begin{whenOrg}
* [[elisp:(show-all)][(>]] [[elisp:(describe-function 'org-dblock-write:bx:dblock:lcnt:latex:commands)][dbf]]
\begin{whenOrg}
*      IIM Parameters    ::  [[elisp:(bx:iimBash:resultsShow:cmndLineElems)][Show Cmnd Line Elems]] || [[elisp:(setq bx:iimBash:iimParamsArgs "-p extent=build+view")][-p extent=build+view]] || [[elisp:(setq bx:iimBash:iimParamsArgs "-p extent=build")][-p extent=build]] || [[elisp:(setq bx:iimBash:iimParamsArgs "-p extent=view")][-p extent=view]]
*      Related           ::  [[elisp:(find-file "./Panel.org")][Visit ./Panel.org]] | [[elisp:(find-file "./mailing/content.mail")][Visit ./mailing/content.mail]]| [[elisp:(find-file "./mailing/Panel.org")][Visit ./mailing/Panel.org]]
*      Build+Release     ::  [[elisp:(bx:iimBash:cmndLineExec :wrapper "" :name "lcntProc.sh" :iif "buildResultsRelease" :iifArgs "")][lcntProc.sh -i buildResultsRelease]] || [[elisp:(lsip-local-run-command-here "lcntProc.sh -i fullClean")][lcntProc.sh -i fullClean]] || [[elisp:(cvs-update "." t)][Version Control]]
*      Build & Preview   ::  [[elisp:(bx:iimBash:cmndLineExec :wrapper "" :name "lcntProc.sh" :iif "buildPdfPreview" :iifArgs "bodyPresArtEnFa.tex")][lcntProc.sh -i buildPdfPreview bodyPresArtEnFa.tex]] || [[elisp:(bx:iimBash:cmndLineExec :wrapper "" :name "lcntProc.sh" :iif "buildHtmlPreview" :iifArgs "bodyPresArtEnFa.tex")][lcntProc.sh -i buildHtmlPreview bodyPresArtEnFa.tex]]

* [[elisp:(org-shifttab)][<)]] [[elisp:(describe-function 'org-dblock-write:bx:dblock:lcnt:latex:commands)][dbFunc)]]  E|
\end{whenOrg}
%%%#+END:

%%%#+BEGINNOT: bx:dblock:global:org-contents-list :disabledP "false" :mode "auto"
\begin{comment}
*      ################ CONTENTS-LIST  ([[elisp:(reftex-toc)][RefTOC)]]    ###############
*  [[elisp:(org-cycle)][| ]]  *Document Status, TODOs and Notes*          ::  [[elisp:(org-cycle)][| ]]
\end{comment}
%%%#+END:

\begin{comment}
** TODO Prophetic
** TODO Add Chapter and verse.
** TODO Stallman and rest are dead wrong. This has never been about Freedom.
** TODO Mention Debian vs Ubuntu is silly.
\end{comment}

%%%#+BEGIN: bx:dblock:lcnt:latex-section :disabledP "false" :seg-title "Introduction"
%%% Args: :class "book|pres+art" :langs "en+fa" :disabledP "false" :seg-title "str" :short-title "str" :label "auto"
\begin{whenOrg}
*  _[[elisp:(blee:menu-sel:outline:popupMenu)][±]]_ _[[elisp:(blee:menu-sel:navigation:popupMenu)][Ξ]]_ [[elisp:(outline-show-branches+toggle)][|=]] [[elisp:(bx:orgm:indirectBufOther)][|>]] *[[elisp:(blee:ppmm:org-mode-toggle)][|N]]*  Section    [[elisp:(outline-show-subtree+toggle)][||]]   /Introduction/ ::  [[elisp:(org-cycle)][| ]]
\end{whenOrg}

\section{Introduction}
%%%#+END:


%%%#+BEGIN: b:lcnt:pres:frame:begin/blank  :label "thisIsMohsen" :comment "body=video"
\begin{whenOrg}
*****  _[[elisp:(blee:menu-sel:outline:popupMenu)][±]]_ _[[elisp:(blee:menu-sel:navigation:popupMenu)][Ξ]]_ [[elisp:(outline-show-branches+toggle)][|=]] [[elisp:(bx:orgm:indirectBufOther)][|>]] *[[elisp:(blee:ppmm:org-mode-toggle)][|N]]*  /Frame:begin-blank/ [[elisp:(outline-show-subtree+toggle)][||]] *Label=thisIsMohsen* UnSpecified -- body=video
\end{whenOrg}

\begin{frame}[fragile,plain,label=thisIsMohsen]
    \frametitle{}
%%BxPy: impressiveFrameParSet('thisIsMohsen', 'always', 'True')
%%BxPy: impressiveFrameParSet('thisIsMohsen', 'transition', 'UnSpecified')
%%%#+END:

%%%#+BEGIN: b:lcnt:pres:frame:body:mm/video :videoPath "./video/mbIntro.mp4" :comment "thisIsMohsen"
\begin{whenOrg}
******  _[[elisp:(blee:menu-sel:outline:popupMenu)][±]]_ _[[elisp:(blee:menu-sel:navigation:popupMenu)][Ξ]]_ [[elisp:(outline-show-branches+toggle)][|=]] [[elisp:(bx:orgm:indirectBufOther)][|>]] *[[elisp:(blee:ppmm:org-mode-toggle)][|N]]*  FrmCntnt-Video [[elisp:(outline-show-subtree+toggle)][||]] Label=UnSpecified UnSpecified
\end{whenOrg}
\begin{presentationMode}
\begin{htmlonly}
  \begin{rawhtml}
<video preload="auto" data-audio-controls src="./video/mbIntro.mp4"></video>
  \end{rawhtml}
\end{htmlonly}
\end{presentationMode}

\begin{articleMode}
\begin{htmlonly}
  \begin{rawhtml}
      <!-- data-autoplay  controls -->
    <p>
    <video  controls   preload="auto"  src="./video/mbIntro.mp4"  height="50%%" width="50%%">
    </video>
    </p>
  \end{rawhtml}
\end{htmlonly}
\end{articleMode}

\begin{presentationMode}
\begin{latexonly}
    \begin{center}
      Video File: ./video/mbIntro.mp4
    \end{center}
\end{latexonly}
\end{presentationMode}

\begin{articleMode}
\begin{latexonly}
    \begin{center}
      Video File: ./video/mbIntro.mp4
    \end{center}
\end{latexonly}
\end{articleMode}
%%%#+END:

\pnote{
  
Greetings. Salaam.

This is Mohsen Banan.

I am a software and internet engineer.

The title of this presentation is
``Blee-LCNT: An Emacs Centered Content Production
and Self-Publication Framework''.

Blee stands for ByStar Libre-Halaal Emacs Environment.
In last year's EmacsConf, I introduced Blee,
BISOS and ByStar as concepts and as foundations.

This year I want to focus on one concrete capability.
Content Production and Self-Publication is
a foundational Blee and BISOS Capability Bundle.

Both this presentation and the Nature of Polyexistentials
book were developed with Blee-LCNT.

In this presentation I want to look at Emacs as a
central ingredient for a usage
environment that we can use to orchestrate production
of quite fancy multi-media presentations.

}

\end{frame}



%%%#+BEGIN: bx:dblock:lcnt:latex-section :mode "auto" :seg-title "Scope: A Complete Multi-Media Content Authorship, Generation, Publication and Distribution Framework" :label ""
%%% Args: :class "book|pres+art" :langs "en+fa" :disabledP "false" :seg-title "str" :short-title "str" :label "auto"
\begin{whenOrg}
*  _[[elisp:(blee:menu-sel:outline:popupMenu)][±]]_ _[[elisp:(blee:menu-sel:navigation:popupMenu)][Ξ]]_ [[elisp:(outline-show-branches+toggle)][|=]] [[elisp:(bx:orgm:indirectBufOther)][|>]] *[[elisp:(blee:ppmm:org-mode-toggle)][|N]]*  Section    [[elisp:(outline-show-subtree+toggle)][||]]   /Scope: A Complete Multi-Media Content Authorship, Generation, Publication and Distribution Framework/ ::  [[elisp:(org-cycle)][| ]]
\end{whenOrg}

\section{Scope: A Complete Multi-Media Content Authorship, Generation, Publication and Distribution Framework}
%%%#+END:

%%%#+BEGIN: b:lcnt:pres:frame/derivedImage :title "Scope:\\\\ A Complete Multi-Media Content Processing Framework" :subtitle "Content Authorship, Generation, Publication and Distribution" :reveal "plain" :beamer "plain" :label "ScopeOfBleeLcnt"
\begin{whenOrg}
*****  _[[elisp:(blee:menu-sel:outline:popupMenu)][±]]_ _[[elisp:(blee:menu-sel:navigation:popupMenu)][Ξ]]_ [[elisp:(outline-show-branches+toggle)][|=]] [[elisp:(bx:orgm:indirectBufOther)][|>]] *[[elisp:(blee:ppmm:org-mode-toggle)][|N]]*  derivedImage [[elisp:(outline-show-subtree+toggle)][||]] Label=ScopeOfBleeLcnt Scope:\\ A Complete Multi-Media Content Processing Framework
\end{whenOrg}

\begin{htmlonly}

\begin{frame}[fragile,plain,label=ScopeOfBleeLcnt]
    \frameaudio{"audio/ScopeOfBleeLcnt.mp3"}
    \frametitle{}
    \framesubtitle{}
    \begin{rawhtml}
<div class="center">
<img src="./disposition.gened/ScopeOfBleeLcnt/slide-1.png" height="500">
</div>
    \end{rawhtml}
\end{frame}
\end{htmlonly}

\begin{verblatex}

\begin{frame}[fragile,plain,label=ScopeOfBleeLcnt]
    \frameaudio{"audio/ScopeOfBleeLcnt.mp3"}
    \frametitle{Scope:\\ A Complete Multi-Media Content Processing Framework}
    \framesubtitle{Content Authorship, Generation, Publication and Distribution}
%%BxPy: impressiveFrameParSet('ScopeOfBleeLcnt', 'always', 'True')
%%BxPy: impressiveFrameParSet('ScopeOfBleeLcnt', 'transition', 'UnSpecified')
%%%#+END:

    \begin{multicols}{3}

    Content Types

      \begin{itemize}
        \item \color{violet} \textbf{Videos}
        \item \color{violet} \textbf{Presentations}
        \item \color{black} Articles/Papers
        \item Books, eBooks
        \item Web Pages
        \item Bus Cards
        \item Name Tags
        \item Posters, Flyers
        \item etc.
      \end{itemize}

      \columnbreak

      Content Production

      \begin{itemize}
      \item \color{violet} \textbf{Authorship}
      \item \color{violet} \textbf{Generation}
      \item \color{violet} \textbf{Disposition}
      \item \color{black} Categorization
      \item Metadata
      \end{itemize}

      \columnbreak

      Content Exposition

      \begin{itemize}
      \item Autonomous Publication
      \item \color{violet} \textbf{Federated Re-Publication}
      \item \color{black} Distribution
      \item Mailings
      \item Engagement
      \item Archival
      \end{itemize}


    \end{multicols}


\pnote{

  Let's consider two different scopes.

  First, the scope of Blee-LCNT Capabilities Bundle, which is that of a
complete multi-media content authorship, generation, publication and distribution framework.
That complete scope is presented in this slide and it spans both black ink and violet ink.
Second, the scope of this presentation, which is more limited.
In this presentation I confine myself to the bullets is violet ink.
Here, I focus on presentation and video as content types and their
authorship and generation and their federated re-publication.

}

\end{frame}
\end{verblatex}



%%%#+BEGIN: bx:dblock:lcnt:latex-section :mode "auto" :seg-title "Prior Art, Similar Art and Some Examples"
%%% Args: :class "book|pres+art" :langs "en+fa" :disabledP "false" :seg-title "str" :short-title "str" :label "auto"
\begin{whenOrg}
*  _[[elisp:(blee:menu-sel:outline:popupMenu)][±]]_ _[[elisp:(blee:menu-sel:navigation:popupMenu)][Ξ]]_ [[elisp:(outline-show-branches+toggle)][|=]] [[elisp:(bx:orgm:indirectBufOther)][|>]] *[[elisp:(blee:ppmm:org-mode-toggle)][|N]]*  Section    [[elisp:(outline-show-subtree+toggle)][||]]   /Prior Art, Similar Art and Some Examples/ ::  [[elisp:(org-cycle)][| ]]
\end{whenOrg}

\section{Prior Art, Similar Art and Some Examples}
%%%#+END:

%%% Not derivedVideo

%%%#+BEGIN: b:lcnt:pres:frame/derivedImage :title "Prior Art and Similar Art" :subtitle "" :label "priorArt" :beamer "plain" :comment "body=odg-image" :transition "default"
\begin{whenOrg}
*****  _[[elisp:(blee:menu-sel:outline:popupMenu)][±]]_ _[[elisp:(blee:menu-sel:navigation:popupMenu)][Ξ]]_ [[elisp:(outline-show-branches+toggle)][|=]] [[elisp:(bx:orgm:indirectBufOther)][|>]] *[[elisp:(blee:ppmm:org-mode-toggle)][|N]]*  derivedImage [[elisp:(outline-show-subtree+toggle)][||]] Label=priorArt Prior Art and Similar Art body=odg-image
\end{whenOrg}

\begin{htmlonly}

\begin{frame}[fragile,plain,label=priorArt]
    \transition{default}
    \frameaudio{"audio/priorArt.mp3"}
    \frametitle{}
    \framesubtitle{}
    \begin{rawhtml}
<div class="center">
<img src="./disposition.gened/priorArt/slide-1.png" height="500">
</div>
    \end{rawhtml}
\end{frame}
\end{htmlonly}

\begin{verblatex}

\begin{frame}[fragile,plain,label=priorArt]
    \transition{default}
    \frameaudio{"audio/priorArt.mp3"}
    \frametitle{Prior Art and Similar Art}
    \framesubtitle{}
%%BxPy: impressiveFrameParSet('priorArt', 'always', 'True')
%%BxPy: impressiveFrameParSet('priorArt', 'transition', 'default')
%%%#+END:



\begin{tabular}{|p{2.5in}|p{0.75in}|p{0.3in}|p{1.5in}|}
  \hline
  \textbf{Title} & \textbf{Author} & \textbf{Year} & \textbf{URL} \\
  \hline
  \small LaTeX export in org-mode: the overhaul  & \small P. Gutiérrez & \small 2025 & \tiny\url{emacsconf.org/2025/talks/latex} \\
  \hline
  \small Emacs Writing Studio & \small Peter Prevos & \small 2024 & \tiny\url{emacsconf.org/2024/talks/writing} \\
  \hline
  \small Authoring \& presenting university courses & \small James & \small 2023 & \tiny\url{emacsconf.org/2023/talks/uni} \\
  \small with Emacs and a full libre software stack & \small Howell &  &  \\  
  \hline
  \small Teaching computer and data science with literate programming tools & \small Marcus Birkenkrahe & \small 2023 & \tiny\url{emacsconf.org/2023/talks/teaching} \\
  \hline
  \small {\color{violet} How we use Org Mode and TRAMP to}  & \small Sacha & \small 2023 & \tiny\url{emacsconf.org/2023/talks/emacsconf} \\
  \small {\color{violet} organize and run a multi-track conference} & \small Chua &  & \tiny {\color{violet} \textbf{Reveal.js Example}}\\
  \hline
  \small {\color{teal} emacs-gc-stats: Does garbage collection}  & \small Ihor  & \small 2023 & \tiny\url{emacsconf.org/2023/talks/gc} \\
  \small {\color{teal} actually slow down Emacs?} & \small Radchenko &  & \tiny {\color{teal} \textbf{Beamer Warsaw LaTeX Example}} \\
  \hline
  \small Moving from Jekyll to OrgMode, & \small Adolfo  & \small 2020 & \tiny\url{emacsconf.org/2020/talks/15} \\
  \small an experience report & \small Villafiorita &  & \\
  \hline
\end{tabular}

\pnote{

  This is a common topic. It makes good sense for us to start with a review of prior art and similar art.
  I went through the past EmacsConf talks and found a good number of them that also deal
  with the topic of content generation.
  A few of these are included in black ink in this slide.
  Many of these have chosen the Babel, in other words Org-Mode+LaTeX as primary input.
  I prefer the inverse of that.

  I also looked for past talks which have used Reveal.js and LaTeX-Beamer.
  For example, Sacha's use of Reveal.js is shown in violet inK.
  And Ihor's use of Beamer is in teal ink.

  This presentation is about a combination of Reveal.js and LaTeX-Beamer.
}

\end{frame}
\end{verblatex}


%%%#+BEGIN: bx:dblock:lcnt:latex-section :mode "auto" :seg-title "LaTeX-Beamer + Reveal.js With Blee and BISOS"
%%% Args: :class "book|pres+art" :langs "en+fa" :disabledP "false" :seg-title "str" :short-title "str" :label "auto"
\begin{whenOrg}
*  _[[elisp:(blee:menu-sel:outline:popupMenu)][±]]_ _[[elisp:(blee:menu-sel:navigation:popupMenu)][Ξ]]_ [[elisp:(outline-show-branches+toggle)][|=]] [[elisp:(bx:orgm:indirectBufOther)][|>]] *[[elisp:(blee:ppmm:org-mode-toggle)][|N]]*  Section    [[elisp:(outline-show-subtree+toggle)][||]]   /LaTeX-Beamer + Reveal.js With Blee and BISOS/ ::  [[elisp:(org-cycle)][| ]]
\end{whenOrg}

\section{LaTeX-Beamer + Reveal.js With Blee and BISOS}
%%%#+END:

%%% Not derivedVideo

%%%#+BEGIN: b:lcnt:pres:frame/derivedImage :title "LaTeX-Beamer + Reveal.js With Blee and BISOS" :subtitle "" :label "beamerPlusReveal" :beamer "plain" :comment "body=odg-image" :transition "default"
\begin{whenOrg}
*****  _[[elisp:(blee:menu-sel:outline:popupMenu)][±]]_ _[[elisp:(blee:menu-sel:navigation:popupMenu)][Ξ]]_ [[elisp:(outline-show-branches+toggle)][|=]] [[elisp:(bx:orgm:indirectBufOther)][|>]] *[[elisp:(blee:ppmm:org-mode-toggle)][|N]]*  derivedImage [[elisp:(outline-show-subtree+toggle)][||]] Label=beamerPlusReveal LaTeX-Beamer + Reveal.js With Blee and BISOS body=odg-image
\end{whenOrg}

\begin{htmlonly}

\begin{frame}[fragile,plain,label=beamerPlusReveal]
    \transition{default}
    \frameaudio{"audio/beamerPlusReveal.mp3"}
    \frametitle{}
    \framesubtitle{}
    \begin{rawhtml}
<div class="center">
<img src="./disposition.gened/beamerPlusReveal/slide-1.png" height="500">
</div>
    \end{rawhtml}
\end{frame}
\end{htmlonly}

\begin{verblatex}

\begin{frame}[fragile,plain,label=beamerPlusReveal]
    \transition{default}
    \frameaudio{"audio/beamerPlusReveal.mp3"}
    \frametitle{LaTeX-Beamer + Reveal.js With Blee and BISOS}
    \framesubtitle{}
%%BxPy: impressiveFrameParSet('beamerPlusReveal', 'always', 'True')
%%BxPy: impressiveFrameParSet('beamerPlusReveal', 'transition', 'default')
%%%#+END:


    \begin{block}{About LaTeX Beamer:}
      Beamer is a document class for the LaTeX typesetting system that is used to create presentations.
      It is in very common use within academic communities.
    \end{block}

    \begin{block}{About Reveal.js:}
          reveal.js is an HTML presentation framework that allows you to create interactive and visually appealing slide decks.
    \end{block}

    \begin{exampleblock}{Blee-LCNT Presentations as LaTeX-Beamer + Reveal.js:}
          A single LaTeX+org-mode source is used to generate LaTeX images and HTML which are injected into Reveal.js and
          then augmented by voice-overs and videos.
    \end{exampleblock}

    \begin{center}
      \colorbox{blue!30}{\color{black}
       \textbf{All of this comes integrated as a BISOS-Capability-Bundle (BCB)}}
      \colorbox{blue!30}{\color{black}
       \textbf{Ready for use (when you buy into ByStar/BISOS/Blee)}}
    \end{center}

\pnote{

  For those who may not be familiar with Beamer and Reveal, here is a quick intro.

  Among academics, LaTeX-Beamer is the go to tool for producing presentations.

  Reveal.js is recognized as the best of breed for dispensing HTML slide decks.

  For many, Reveal and Beamer live in different universes.

  Beamer is pdf oriented and  Reveal is html oriented.

  Combining two powerful tools makes for an even more powerful tool.

  This Blee-LCNT Presentations combines the best of LaTeX-Beamer with Reveal.js.

}

\end{frame}
\end{verblatex}


%%%#+BEGIN: bx:dblock:lcnt:latex-section :mode "auto" :seg-title "Blee-LCNT Novel Concepts"
%%% Args: :class "book|pres+art" :langs "en+fa" :disabledP "false" :seg-title "str" :short-title "str" :label "auto"
\begin{whenOrg}
*  _[[elisp:(blee:menu-sel:outline:popupMenu)][±]]_ _[[elisp:(blee:menu-sel:navigation:popupMenu)][Ξ]]_ [[elisp:(outline-show-branches+toggle)][|=]] [[elisp:(bx:orgm:indirectBufOther)][|>]] *[[elisp:(blee:ppmm:org-mode-toggle)][|N]]*  Section    [[elisp:(outline-show-subtree+toggle)][||]]   /Blee-LCNT Novel Concepts/ ::  [[elisp:(org-cycle)][| ]]
\end{whenOrg}

\section{Blee-LCNT Novel Concepts}
%%%#+END:

%%% Not derivedVideo

%%%#+BEGIN: b:lcnt:pres:frame/derivedImage :title "Blee-LCNT Novel Concepts" :subtitle "" :label "novelConcepts" :beamer "plain" :comment "body=odg-image" :transition "default"
\begin{whenOrg}
*****  _[[elisp:(blee:menu-sel:outline:popupMenu)][±]]_ _[[elisp:(blee:menu-sel:navigation:popupMenu)][Ξ]]_ [[elisp:(outline-show-branches+toggle)][|=]] [[elisp:(bx:orgm:indirectBufOther)][|>]] *[[elisp:(blee:ppmm:org-mode-toggle)][|N]]*  derivedImage [[elisp:(outline-show-subtree+toggle)][||]] Label=novelConcepts Blee-LCNT Novel Concepts body=odg-image
\end{whenOrg}

\begin{htmlonly}

\begin{frame}[fragile,plain,label=novelConcepts]
    \transition{default}
    \frameaudio{"audio/novelConcepts.mp3"}
    \frametitle{}
    \framesubtitle{}
    \begin{rawhtml}
<div class="center">
<img src="./disposition.gened/novelConcepts/slide-1.png" height="500">
</div>
    \end{rawhtml}
\end{frame}
\end{htmlonly}

\begin{verblatex}

\begin{frame}[fragile,plain,label=novelConcepts]
    \transition{default}
    \frameaudio{"audio/novelConcepts.mp3"}
    \frametitle{Blee-LCNT Novel Concepts}
    \framesubtitle{}
%%BxPy: impressiveFrameParSet('novelConcepts', 'always', 'True')
%%BxPy: impressiveFrameParSet('novelConcepts', 'transition', 'default')
%%%#+END:

   \begin{multicols}{2}

     \color{magenta} \textbf{Common Prior Art Key Concepts}

      \begin{itemize}
        \item \color{black} \textbf{Linux and Emacs}
        \item \color{violet} \textbf{Org-Mode + LaTeX}
        \item \color{violet} \textbf{Org-Babel \& Literate}
        \item \color{violet}  \textbf{Reveal.js By Itself}
        \item \color{violet}  \textbf{LaTeX Beamer By Itself}
        \item \color{violet}  \textbf{DIY with more Packages}
        \item \color{violet}  \textbf{All-In-One Result Videos}
        % \item \color{violet}  \textbf{Video As Primary Output}

      \end{itemize}

      \columnbreak

      \color{blue} \textbf{Key Novel Concepts}

      \begin{itemize}
        \item \color{black} \textbf{Linux and Emacs}
        \item \color{teal} \textbf{LaTeX + Org-Mode}
        \item \color{teal} \textbf{COMEEGA \& Surrounded}
        \item \color{teal}  \textbf{LaTeX Beamer + Reveal.js}
        \item \color{teal}  \textbf{A BISOS Capability Bundle}
        \item \color{teal}  \textbf{Modular Results}
        % \item \color{teal}  \textbf{Series of Output}
      \end{itemize}

    \end{multicols}

    \small LaTeX Beamer pdf result is disected into frame images and inserted in Reveal.js.

    \small LaTeX Beamer frames can also be translated into html with HeVeA and inserted in Reveal.js. Voice-overs for Beamer frames are correlated to frame names and applied to image or html frames in Reveal.js.


\pnote{

  Beamer primarily functions as producer and Reveal functions as dispenser and multi-media enhancer.

  Here is how the combination works.

  LaTeX Beamer pdf result is disected into named frame images which can then be inserted in Reveal.js.
  
  LaTeX Beamer frames can also be translated into html with HeVeA which can also be
  inserted in Reveal.js.

  Voice-overs for Beamer frames can be correlated to frame names
  and applied to image or html frames. Screen captures and image narrations as videos can
  be directly dispensed through Reveal.

  There are various additional novel concepts with regard to the way that we have
  integrated all of this together. Instead of Org-Mode+LaTeX, we do LaTeX+Org-Mode.
  Instead of Babel, we do COMEEGA, instead of the Literate model we introduce the
  Surrounded model.

  You shall see various examples of these shortly.

}

\end{frame}
\end{verblatex}



%%%#+BEGIN: bx:dblock:lcnt:latex-section :mode "auto" :seg-title "Part of a Bigger Picture -- Part of a Series" :label ""
%%% Args: :class "book|pres+art" :langs "en+fa" :disabledP "false" :seg-title "str" :short-title "str" :label "auto"
\begin{whenOrg}
*  _[[elisp:(blee:menu-sel:outline:popupMenu)][±]]_ _[[elisp:(blee:menu-sel:navigation:popupMenu)][Ξ]]_ [[elisp:(outline-show-branches+toggle)][|=]] [[elisp:(bx:orgm:indirectBufOther)][|>]] *[[elisp:(blee:ppmm:org-mode-toggle)][|N]]*  Section    [[elisp:(outline-show-subtree+toggle)][||]]   /Part of a Bigger Picture -- Part of a Series/ ::  [[elisp:(org-cycle)][| ]]
\end{whenOrg}

\section{Part of a Bigger Picture -- Part of a Series}
%%%#+END:

%%%#+BEGIN: b:lcnt:pres:frame/derivedImage :title "Part of a Bigger Picture -- Part of a Series" :subtitle "ByStar, BISOS, Blee" :reveal "plain" :beamer "plain" :label "PartOfByStar"
\begin{whenOrg}
*****  _[[elisp:(blee:menu-sel:outline:popupMenu)][±]]_ _[[elisp:(blee:menu-sel:navigation:popupMenu)][Ξ]]_ [[elisp:(outline-show-branches+toggle)][|=]] [[elisp:(bx:orgm:indirectBufOther)][|>]] *[[elisp:(blee:ppmm:org-mode-toggle)][|N]]*  derivedImage [[elisp:(outline-show-subtree+toggle)][||]] Label=PartOfByStar Part of a Bigger Picture -- Part of a Series
\end{whenOrg}

\begin{htmlonly}

\begin{frame}[fragile,plain,label=PartOfByStar]
    \frameaudio{"audio/PartOfByStar.mp3"}
    \frametitle{}
    \framesubtitle{}
    \begin{rawhtml}
<div class="center">
<img src="./disposition.gened/PartOfByStar/slide-1.png" height="500">
</div>
    \end{rawhtml}
\end{frame}
\end{htmlonly}

\begin{verblatex}

\begin{frame}[fragile,plain,label=PartOfByStar]
    \frameaudio{"audio/PartOfByStar.mp3"}
    \frametitle{Part of a Bigger Picture -- Part of a Series}
    \framesubtitle{ByStar, BISOS, Blee}
%%BxPy: impressiveFrameParSet('PartOfByStar', 'always', 'True')
%%BxPy: impressiveFrameParSet('PartOfByStar', 'transition', 'UnSpecified')
%%%#+END:

\begin{itemize}
    \item EmacsConf 2024 -- About Blee: enveloping our own autonomy directed digital ecosystem with Emacs -- \url{https://emacsconf.org/2024/talks/blee}
    \begin{itemize}
      \item introduced Blee, the ByStar Libre-Halaal Emacs Environment
      \item introduced BISOS, the ByStar Internet Services Operating System
      \item introduced ByStar, the Libre-Halaal ByStar Digital Ecosystem
    \end{itemize}
    \item EmacsConf 2022 -- Revisiting the anatomy of Emacs mail user agents -- \url{https://emacsconf.org/2022/talks/mail}
    \begin{itemize}
      \item introduced a context for the ByStar Inter-Personal Communications Capability
    \end{itemize}
    \item EmacsConf 2021 -- Perso-Arabic Input Methods And Making More Emacs Apps BIDI Aware -- \url{https://emacsconf.org/2021/talks/perso-arabic}
    \begin{itemize}
      \item introduced a needed component for multi-lingual content authorship and generation
    \end{itemize}
\end{itemize}

\end{frame}
\end{verblatex}

\pnote{

  All of this is part of a bigger picture. A much bigger picture.
  My talks at EmacsConf 2021, 2022 and 2024 are related. This 2025 talk builds on those.

  Last year's talk ``About Blee: enveloping our own autonomy directed digital ecosystem
  with Emacs'' in particular, lays the foundations for this talk. If you have not seen
  that, it would make good send to review it.

  In my previous talks I have been criticized of having a ``prophetic'' style.
  The scope of ByStar is lofty and immense. In many ways it is unbelievable.
  And EmacsConf talks are meant to be short. So, as a result, sometimes I end up being cryptic.
}




%%%#+BEGIN: bx:dblock:lcnt:latex-part :disabledP "false" :seg-title "The ``Nature of Polyexistentials'' Book"
%%% Args: :toc "NU" :tocDepth 3 :part "NU" :label "auto|spec" :partpage t
\begin{whenOrg}
*      ================
*  [[elisp:(blee:ppmm:org-mode-toggle)][|n]] [[elisp:(blee:menu-sel:outline:popupMenu)][+-]] [[elisp:(blee:menu-sel:navigation:popupMenu)][==]]  *Part*   _The ``Nature of Polyexistentials'' Book_ ::  [[elisp:(org-cycle)][| ]]
\end{whenOrg}

\newpage
\part{The ``Nature of Polyexistentials'' Book}
%%%#+END:

%%%#+BEGIN: b:lcnt:pres:frame/derivedImage :title "Introducing The Nature of Polyexistentials Book" :subtitle "" :label "part_polyexistentialsBook" :comment "body=TOC" :reveal "plain" :beamer ""
\begin{whenOrg}
*****  _[[elisp:(blee:menu-sel:outline:popupMenu)][±]]_ _[[elisp:(blee:menu-sel:navigation:popupMenu)][Ξ]]_ [[elisp:(outline-show-branches+toggle)][|=]] [[elisp:(bx:orgm:indirectBufOther)][|>]] *[[elisp:(blee:ppmm:org-mode-toggle)][|N]]*  derivedImage [[elisp:(outline-show-subtree+toggle)][||]] Label=part_polyexistentialsBook Introducing The Nature of Polyexistentials Book body=text
\end{whenOrg}

\begin{htmlonly}

\begin{frame}[fragile,plain,label=part_polyexistentialsBook]
    \frameaudio{"audio/part_polyexistentialsBook.mp3"}
    \frametitle{}
    \framesubtitle{}
    \begin{rawhtml}
<div class="center">
<img src="./disposition.gened/part_polyexistentialsBook/slide-1.png" height="500">
</div>
    \end{rawhtml}
\end{frame}
\end{htmlonly}

\begin{verblatex}

\begin{frame}[fragile,label=part_polyexistentialsBook]
    \frameaudio{"audio/part_polyexistentialsBook.mp3"}
    \frametitle{Introducing The Nature of Polyexistentials Book}
    \framesubtitle{}
%%BxPy: impressiveFrameParSet('part_polyexistentialsBook', 'always', 'True')
%%BxPy: impressiveFrameParSet('part_polyexistentialsBook', 'transition', 'UnSpecified')
%%%#+END:

    \tableofcontents

\pnote{

  Having accepted the ``prophetic'' criticism as legitimate, I now need to put a book on the table.
  With that book in place, moving forward, when needing to be cryptic, I shall cite Chapter and Verse.

}
\end{frame}
\end{verblatex}



%%%#+BEGIN: bx:dblock:lcnt:latex-section :mode "auto" :seg-title "The ``Nature of Polyexistentials'' Book and The Full ByStar Story"
%%% Args: :class "book|pres+art" :langs "en+fa" :disabledP "false" :seg-title "str" :short-title "str" :label "auto"
\begin{whenOrg}
*  _[[elisp:(blee:menu-sel:outline:popupMenu)][±]]_ _[[elisp:(blee:menu-sel:navigation:popupMenu)][Ξ]]_ [[elisp:(outline-show-branches+toggle)][|=]] [[elisp:(bx:orgm:indirectBufOther)][|>]] *[[elisp:(blee:ppmm:org-mode-toggle)][|N]]*  Section    [[elisp:(outline-show-subtree+toggle)][||]]   /The ``Nature of Polyexistentials'' Book and The Full ByStar Story/ ::  [[elisp:(org-cycle)][| ]]
\end{whenOrg}

\section{The ``Nature of Polyexistentials'' Book and The Full ByStar Story}
%%%#+END:

%%%  :title "``Nature of Polyexistentials'' Book and The Full ByStar Story" :subtitle "" :label "fullBystarStory" :comment "body=text" :reveal "plain" :beamer "plain"


%%%#+BEGIN: b:lcnt:pres:frame/derivedImage :title "" :subtitle "" :label "fullBystarStory" :comment "body=text" :reveal "plain" :beamer "plain"
\begin{whenOrg}
*****  _[[elisp:(blee:menu-sel:outline:popupMenu)][±]]_ _[[elisp:(blee:menu-sel:navigation:popupMenu)][Ξ]]_ [[elisp:(outline-show-branches+toggle)][|=]] [[elisp:(bx:orgm:indirectBufOther)][|>]] *[[elisp:(blee:ppmm:org-mode-toggle)][|N]]*  derivedImage [[elisp:(outline-show-subtree+toggle)][||]] Label=fullBystarStory  body=text
\end{whenOrg}

\begin{htmlonly}

\begin{frame}[fragile,plain,label=fullBystarStory]
    \frameaudio{"audio/fullBystarStory.mp3"}
    \frametitle{}
    \framesubtitle{}
    \begin{rawhtml}
<div class="center">
<img src="./disposition.gened/fullBystarStory/slide-1.png" height="500">
</div>
    \end{rawhtml}
\end{frame}
\end{htmlonly}

\begin{verblatex}

\begin{frame}[fragile,plain,label=fullBystarStory]
    \frameaudio{"audio/fullBystarStory.mp3"}
    \frametitle{}
    \framesubtitle{}
%%BxPy: impressiveFrameParSet('fullBystarStory', 'always', 'True')
%%BxPy: impressiveFrameParSet('fullBystarStory', 'transition', 'UnSpecified')
%%%#+END:

\begin{center}
  {\huge \textbf{Nature of Polyexistentials}}
\end{center}


\begin{center}
  {\Large \colorbox{blue!50}{\color{white} Basis for Abolishment of}}
\end{center}

\begin{center}
  {\Large \colorbox{blue!50}{\color{white} the Western Intellectual Property Rights Regime}}
\end{center}


\begin{center}
  {\Large And Introduction of}
\end{center}

\begin{center}
  {\Large {\color{blue} \textbf{the Libre-Halaal ByStar Digital Ecosystem}}}
\end{center}


\begin{center}
 \href{https://github.com/bxplpc/120033}{https://github.com/bxplpc/120033}
\end{center}

\begin{center}
\vspace{0.2in}
\qrcode[height=0.5in]{https://github.com/bxplpc/120033}
\end{center}


\pnote{

  I am delighted to announce the availability of my recent book, ``Nature of
  Polyexistentials''.

  The full title of my book is:
  Nature Of Polyexistentials ---
  Basis For Abolishment Of The Western Intellectual Property Rights Regime ---
  And Introduction Of The Libre-Halaal ByStar Digital Ecosystem.

  Knowledge, know-how, uses of know-how, ideas, formulas, software and information are
  inherently non-scarce. They are \textbf{polyexistentials}. Unlike monoexistentials which
  exist in singular, polyexistentials naturally exist in multiples.

  What is abundant in nature is being made artificially scarce through man-made
  ownership rules called copyright and patents.

  These mistaken ownership rules, the so called Western IPR regime, has immense ramifications on
  the shape the direction of the American Digital Ecosystem.
  It would be an understatement to say that the American Digital Ecosystem has put humanity in danger.
}
\end{frame}
\end{verblatex}

%%%#+BEGIN: bx:dblock:lcnt:latex-section :mode "auto" :seg-title "A Layered Overview of The Polyexistentials Book"
%%% Args: :class "book|pres+art" :langs "en+fa" :disabledP "false" :seg-title "str" :short-title "str" :label "auto"
\begin{whenOrg}
*  _[[elisp:(blee:menu-sel:outline:popupMenu)][±]]_ _[[elisp:(blee:menu-sel:navigation:popupMenu)][Ξ]]_ [[elisp:(outline-show-branches+toggle)][|=]] [[elisp:(bx:orgm:indirectBufOther)][|>]] *[[elisp:(blee:ppmm:org-mode-toggle)][|N]]*  Section    [[elisp:(outline-show-subtree+toggle)][||]]   /A Layered Overview of The Polyexistentials Book/ ::  [[elisp:(org-cycle)][| ]]
\end{whenOrg}

\section{A Layered Overview of The Polyexistentials Book}
%%%#+END:

%%%#+BEGIN: b:lcnt:pres:frame/derivedImage :title "" :subtitle "" :reveal "plain" :beamer "blank" :label "polyBookLayers" :comment "body=derivedImage"
\begin{whenOrg}
*****  _[[elisp:(blee:menu-sel:outline:popupMenu)][±]]_ _[[elisp:(blee:menu-sel:navigation:popupMenu)][Ξ]]_ [[elisp:(outline-show-branches+toggle)][|=]] [[elisp:(bx:orgm:indirectBufOther)][|>]] *[[elisp:(blee:ppmm:org-mode-toggle)][|N]]*  derivedImage [[elisp:(outline-show-subtree+toggle)][||]] Label=polyBookLayers  body=derivedInsert
\end{whenOrg}

\begin{htmlonly}

\begin{frame}[fragile,plain,label=polyBookLayers]
    \frameaudio{"audio/polyBookLayers.mp3"}
    \frametitle{}
    \framesubtitle{}
    \begin{rawhtml}
<div class="center">
<img src="./disposition.gened/polyBookLayers/slide-1.png" height="500">
</div>
    \end{rawhtml}
\end{frame}
\end{htmlonly}

\begin{verblatex}

\begin{frame}[fragile,plain,label=polyBookLayers]
    \frameaudio{"audio/polyBookLayers.mp3"}
    \frametitle{}
    \framesubtitle{}
%%BxPy: impressiveFrameParSet('polyBookLayers', 'always', 'True')
%%BxPy: impressiveFrameParSet('polyBookLayers', 'transition', 'UnSpecified')
%%%#+END:


%%%#+BEGIN: bx:dblock:lcnt:body:odg-artpres :beamerSize "max" :fig-file "/de/lcnt/lgpc/bystar/permanent/common/figures/polyBookLayers-9x16.odg"
\begin{comment}
******  [[elisp:(org-cycle)][| ]]  [[elisp:(blee:ppmm:org-mode-toggle)][Nat]] [[elisp:(beginning-of-buffer)][Top]] [[elisp:(delete-other-windows)][(1)]] || /Figure/ =ODG-ArtPres=  *polyBookLayers-9x16* -- Nature of Polyexistentials --- Book Outline ::  [[elisp:(org-cycle)][| ]]
\end{comment}

\begin{presentationMode}

\begin{latexonly}
  \begin{figure}
    \begin{center}
       \includegraphics[width=145mm,keepaspectratio]{/de/lcnt/lgpc/bystar/permanent/common/figures/polyBookLayers-9x16}
    \end{center}
  \end{figure}
\end{latexonly}

\begin{htmlonly}
  \begin{rawhtml}
<div class="center">
<img src="/de/lcnt/lgpc/bystar/permanent/common/figures/polyBookLayers-9x16.png" height="450">
</div>
  \end{rawhtml}
\end{htmlonly}

\end{presentationMode}


\begin{articleMode}

\begin{latexonly}
  \begin{figure}[H]
    \begin{center}
      \includegraphics[width=\textwidth]{/de/lcnt/lgpc/bystar/permanent/common/figures/polyBookLayers-9x16}
      \caption{Nature of Polyexistentials --- Book Outline}
      \label{fig:polyBookLayers-9x16}
    \end{center}
  \end{figure}
\end{latexonly}

\begin{htmlonly}
  %BEGIN IMAGE
  \begin{center}
      \includegraphics[width=\textwidth]{/de/lcnt/lgpc/bystar/permanent/common/figures/polyBookLayers-9x16}
  \end{center}
  %END IMAGE
  %HEVEA\imageflush

  \begin{figure}
      \caption{Nature of Polyexistentials --- Book Outline}
      \label{fig:polyBookLayers-9x16}
  \end{figure}
\end{htmlonly}

\end{articleMode}

%%%#+END:


\pnote{

Two parts of the book, in particular are of immediate relevance.

Part III, the ethics layer, focuses on contours of cures.

Having dismissed the Western intellectual property rights (IPR) regime as an erroneous
governance model for polyexistentials, I propose the Libre-Halaal model of governance of
polyexistentials towards facilitating conviviality of tools.

Part IV, the engineering layer, introduces the Libre-Halaal ByStar Digital Ecosystem.
as an ethical alternative to the prevailing proprietary American digital
ecosystem.

The book also provides additional details about the content generation and publication
facilities that I am presenting here. And the book itself, as content, was generated
and published using the facilities that I am presenting here.

}
\end{frame}
\end{verblatex}


%%%#+BEGIN: bx:dblock:lcnt:latex-section :mode "auto" :seg-title "Nature of Polyexistentials Book -- Vol I and II Key Concepts"
%%% Args: :class "book|pres+art" :langs "en+fa" :disabledP "false" :seg-title "str" :short-title "str" :label "auto"
\begin{whenOrg}
*  _[[elisp:(blee:menu-sel:outline:popupMenu)][±]]_ _[[elisp:(blee:menu-sel:navigation:popupMenu)][Ξ]]_ [[elisp:(outline-show-branches+toggle)][|=]] [[elisp:(bx:orgm:indirectBufOther)][|>]] *[[elisp:(blee:ppmm:org-mode-toggle)][|N]]*  Section    [[elisp:(outline-show-subtree+toggle)][||]]   /Nature of Polyexistentials Book -- Vol I and II Key Concepts/ ::  [[elisp:(org-cycle)][| ]]
\end{whenOrg}

\section{Nature of Polyexistentials Book -- Vol I and II Key Concepts}
%%%#+END:


%%%#+BEGIN: b:lcnt:pres:frame/derivedImage :title "Nature of Polyexistentials Book -- Key Concepts" :subtitle "Volumes I and II" :label "twoVolumesConcepts" :comment "body=enumerate" :reveal "plain" :beamer "plain"
\begin{whenOrg}
*****  _[[elisp:(blee:menu-sel:outline:popupMenu)][±]]_ _[[elisp:(blee:menu-sel:navigation:popupMenu)][Ξ]]_ [[elisp:(outline-show-branches+toggle)][|=]] [[elisp:(bx:orgm:indirectBufOther)][|>]] *[[elisp:(blee:ppmm:org-mode-toggle)][|N]]*  derivedImage [[elisp:(outline-show-subtree+toggle)][||]] Label=twoVolumesConcepts Nature of Polyexistentials Book -- Key Concepts body=enumerate
\end{whenOrg}

\begin{htmlonly}

\begin{frame}[fragile,plain,label=twoVolumesConcepts]
    \frameaudio{"audio/twoVolumesConcepts.mp3"}
    \frametitle{}
    \framesubtitle{}
    \begin{rawhtml}
<div class="center">
<img src="./disposition.gened/twoVolumesConcepts/slide-1.png" height="500">
</div>
    \end{rawhtml}
\end{frame}
\end{htmlonly}

\begin{verblatex}

\begin{frame}[fragile,plain,label=twoVolumesConcepts]
    \frameaudio{"audio/twoVolumesConcepts.mp3"}
    \frametitle{Nature of Polyexistentials Book -- Key Concepts}
    \framesubtitle{Volumes I and II}
%%BxPy: impressiveFrameParSet('twoVolumesConcepts', 'always', 'True')
%%BxPy: impressiveFrameParSet('twoVolumesConcepts', 'transition', 'UnSpecified')
%%%#+END:


    \begin{columns}

      \begin{column}{0.45\textwidth}

        \begin{center}
          \colorbox{blue!30}{\color{black} \textbf{Volume I:}}
          \colorbox{blue!30}{\color{black} \textbf{Rejection of the IPR Regime}}
        \end{center}
      \end{column}

      \begin{column}{0.55\textwidth}

        \begin{center}
          \colorbox{blue!60}{\color{white} \textbf{Volume II:}}
          \colorbox{blue!60}{\color{white} \textbf{Libre-Halaal ByStar Digital Ecosystem}}
        \end{center}
      \end{column}

    \end{columns}

    \bigskip

    \begin{columns}

      \begin{column}{0.5\textwidth}

        \begin{itemize}
          \item Polyexistentials
          \item Western IPR and Americanism
          \item Abolishment of IPR Regime
          \item Libre-Halaal Governance of Polyexistentials
          % \item Polyexistential Capitalism
          \item Inside of IPR: Unique Software Public License: Affero GPL
          \item Outside of IPR: Eastern Strategies
        \end{itemize}
      \end{column}

      \begin{column}{0.5\textwidth}

        \begin{itemize}
          \item Libre-Halaal ByStar DE
                \begin{itemize}
                  \item Service Portability, Service Possession Assertion, Autonomy, Privacy
                \end{itemize}
          \item BISOS - By* Internet Services OS
                \begin{itemize}
                  \item PyCS, By* Portable Objects (BPOs), PALS
                \end{itemize}
          \item Blee - By* Libre Emacs Env
                \begin{itemize}
                  \item COMEEGA, Blee Panels, PyCS Player
                \end{itemize}
        \end{itemize}
      \end{column}

    \end{columns}



\pnote{

  You can think of this book as being in two volumes.

  Our focus are Blee and BISOS in Volume II.

  Volume I deals with the general concept of polyexistence and invalidity of IPR
  and our terminoloy of Libre-Halaal --- instead of the common but ill directed
  vocabulary of Free Software and Open-Source and FOSS.

  In Chapter 11, I introduce the very sensitive and potent vocabulary of Halaal and
  Libre-Halaal.

}

\end{frame}
\end{verblatex}



%%%#+BEGIN: bx:dblock:lcnt:latex-section :mode "auto" :seg-title "Obtaining the Nature of Polyexistentials Book"
%%% Args: :class "book|pres+art" :langs "en+fa" :disabledP "false" :seg-title "str" :short-title "str" :label "auto"
\begin{whenOrg}
*  _[[elisp:(blee:menu-sel:outline:popupMenu)][±]]_ _[[elisp:(blee:menu-sel:navigation:popupMenu)][Ξ]]_ [[elisp:(outline-show-branches+toggle)][|=]] [[elisp:(bx:orgm:indirectBufOther)][|>]] *[[elisp:(blee:ppmm:org-mode-toggle)][|N]]*  Section    [[elisp:(outline-show-subtree+toggle)][||]]   /Obtaining the Nature of Polyexistentials Book/ ::  [[elisp:(org-cycle)][| ]]
\end{whenOrg}

\section{Obtaining the Nature of Polyexistentials Book}
%%%#+END:

%%%#+BEGIN: b:lcnt:pres:frame/insertDerivedVideo :title "" :subtitle "" :label "obtainingPolyBookInsert" :comment "body=derivedInsert"  :derivedLabel "obtainingPolyBook"
\begin{whenOrg}
*****  _[[elisp:(blee:menu-sel:outline:popupMenu)][±]]_ _[[elisp:(blee:menu-sel:navigation:popupMenu)][Ξ]]_ [[elisp:(outline-show-branches+toggle)][|=]] [[elisp:(bx:orgm:indirectBufOther)][|>]] *[[elisp:(blee:ppmm:org-mode-toggle)][|N]]*  insertDerivedVideo [[elisp:(outline-show-subtree+toggle)][||]] Label=obtainingPolyBookInsert  body=derivedInsert
\end{whenOrg}

\begin{frame}[fragile,plain,label=obtainingPolyBookInsert]
    \frametitle{}
    \framesubtitle{}


\begin{htmlonly} %% Must be at begining of line
   \begin{presentationMode}
    \begin{rawhtml}
<video preload="auto" data-audio-controls src="./video/derived-obtainingPolyBook.mp4"></video>
    \end{rawhtml}
    \end{presentationMode}

    \begin{articleMode}
    \begin{rawhtml}
    <!-- data-autoplay  controls -->
    <p>
    <video controls  preload="auto" src="./video/derived-obtainingPolyBook.mp4" height="50%%" width="50%%">
    </video>
    </p>
     \end{rawhtml}
    \end{articleMode}
\end{htmlonly}

\begin{latexonly}

Reveal Video Insertion Of src="./video/derived-obtainingPolyBook.mp4"

\end{latexonly}

%%%#+END:

\pnote{

  The contents of this book belong to all of humanity and verbatim copying of it is
  unrestricted. If you want to read it, this book is yours.

  The ``Nature of Polyexistentials'' book is available both online and in print.

  This book is available as two editions. The US Edition and the International edition. The
  US Edition is written with a slightly milder Western unfriendly tone, while the International Edition
  includes additional original content in Farsi. I consider the International Edition to
  be the authoritative version. However, many readers in the US and Western countries may
  prefer the US Edition.

  I maintain separate Git repositories for each edition on GitHub:
  US Edition is at bxplpc/120033 and International Edition: bxplpc/120074

  Cloning these repositories will give you access to the book in PDF format (suitable for
  both A4 and US Letter printing) and in EPUB format. Alternatively, the content can be
  downloaded directly from your browser without needing to clone the repositories.

  To ensure broader online availability and stability, I have also published the book on
  Zenodo, complete with a DOI (Digital Object Identifier). You can download both the A4 and
  8.5 x 11 PDFs from there as well.

  The book is also available in print on Amazon and at most major bookstores in the US and
  Western regions. The ISBNs for both editions are included in this slide.
  Additionally, I have published this book in Iran through Jangal Publishers.

  I did not write this book for profit. My aim is to share my thoughts and encourage
  readers to engage with my views and ideas. Your feedback is welcome, and I am genuinely
  interested in hearing your perspectives.

  In Western markets, I have priced the print edition somewhat above production costs. If
  you find value in the book and the ByStar project, purchasing a copy will help support
  my work. Thanks in advance for your support.

}
\end{frame}



%%%#+BEGIN: b:lcnt:pres:frame:begin/plain :title "Obtaining the Nature of Polyexistentials Book" :subtitle "" :label "obtainingPolyBook" :comment "body=itemize"
\begin{whenOrg}
*****  _[[elisp:(blee:menu-sel:outline:popupMenu)][±]]_ _[[elisp:(blee:menu-sel:navigation:popupMenu)][Ξ]]_ [[elisp:(outline-show-branches+toggle)][|=]] [[elisp:(bx:orgm:indirectBufOther)][|>]] *[[elisp:(blee:ppmm:org-mode-toggle)][|N]]*  /Frame:begin-plain/ [[elisp:(outline-show-subtree+toggle)][||]] *Label=obtainingPolyBook* Obtaining the Nature of Polyexistentials Book -- body=itemize
\end{whenOrg}

\begin{frame}[fragile,plain,label=obtainingPolyBook]
    \frameaudio{"audio/obtainingPolyBook.mp3"}
    \frametitle{Obtaining the Nature of Polyexistentials Book}
    \framesubtitle{}
%%BxPy: impressiveFrameParSet('obtainingPolyBook', 'always', 'True')
%%BxPy: impressiveFrameParSet('obtainingPolyBook', 'transition', 'UnSpecified')
%%%#+END:

\begin{htmlonly} %% Must be at begining of line
    \begin{center}
          \textbf{Online:}
    \end{center}
\end{htmlonly}

\begin{latexonly}
    \begin{center}
          \colorbox{blue!30}{\color{black} \textbf{Online:}}
    \end{center}
\end{latexonly}

    \settowidth{\leftmargini}{\usebeamertemplate{itemize item}}
    \addtolength{\leftmargini}{\labelsep}

    \begin{itemize}
      \item At Github: (Pdf, EPUB, Html) \\
      US Edition:   \url{https://github.com/bxplpc/120033} - \href{https://github.com/bxplpc/120033/blob/main/pdf/c-120033-1_05-book-a4-col-emb-pub.pdf}{pdf-A4} -
      \href{https://github.com/bxplpc/120033/blob/main/pdf/c-120033-1_05-book-8.5x11-col-emb-pub.pdf}{pdf-US}\\
      International: \url{https://github.com/bxplpc/120074} - \href{https://github.com/bxplpc/120074/blob/main/pdf/c-120074-1_05-book-a4-col-emb-pub.pdf}{pdf-A4} -   \href{https://github.com/bxplpc/120074/blob/main/pdf/c-120074-1_05-book-8.5x11-col-emb-pub.pdf}{pdf-US}\\
      \item DOI at Zenodo: (Pdf)\\
      US Edition:   \href{https://zenodo.org/records/11100505}{zenodo.org/records/11100505}  -   \href{https://zenodo.org/records/11100527}{pdf-A4}\\
      International: \href{https://zenodo.org/records/11100543}{zenodo.org/records/11100543}  -   \href{https://zenodo.org/records/11100551}{pdf-8.5x11}\\
    \end{itemize}

    \begin{center}
      \colorbox{blue!60}{\color{white} \textbf{In Print:}}
    \end{center}

    \begin{itemize}
      \item At Amazon, Wallmart, etc: (ISBNs 978-1-960957-01-6  978-1-960957-11-5)\\
      US Edition:   \url{https://www.amazon.com/dp/1960957015}\\
      International Edition: \url{https://www.amazon.com/dp/1960957112}\\
      \item In Iran at Jangal Publishers (International) -- ISBN 978-622-238-588-0:\\
      \href{https://jangal.com/fa/product/252689/nature-of-polyexistentials}{jangal.com/fa/product/252689/nature-of-polyexistentials}
    \end{itemize}

\pnote{
  And here are the same links as a native Reveal slide.

  If instead of a video,  you are viewing this presentation as a Reveal web page,
  you can just click on the pointers and URLs.
}
\end{frame}


%%%#+BEGIN: bx:dblock:lcnt:latex-part :disabledP "false" :seg-title "Content Processing -- A ByStar/BISOS/Blee Capability Bundle"
%%% Args: :toc "NU" :tocDepth 3 :part "NU" :label "auto|spec" :partpage t
\begin{whenOrg}
*      ================
*  [[elisp:(blee:ppmm:org-mode-toggle)][|n]] [[elisp:(blee:menu-sel:outline:popupMenu)][+-]] [[elisp:(blee:menu-sel:navigation:popupMenu)][==]]  *Part*   _Content Processing -- A ByStar/BISOS/Blee Capability Bundle_ ::  [[elisp:(org-cycle)][| ]]
\end{whenOrg}

\newpage
\part{Content Processing -- A ByStar/BISOS/Blee Capability Bundle}
%%%#+END:


%%%#+BEGIN: b:lcnt:pres:frame/derivedImage :title "Content Processing --\\\\ A ByStar/BISOS/Blee Capability Bundle (BCB)" :subtitle "" :label "part_bystarCapability" :comment "body=text -- Part TOC" :reveal "plain" :beamer ""
\begin{whenOrg}
*****  _[[elisp:(blee:menu-sel:outline:popupMenu)][±]]_ _[[elisp:(blee:menu-sel:navigation:popupMenu)][Ξ]]_ [[elisp:(outline-show-branches+toggle)][|=]] [[elisp:(bx:orgm:indirectBufOther)][|>]] *[[elisp:(blee:ppmm:org-mode-toggle)][|N]]*  derivedImage [[elisp:(outline-show-subtree+toggle)][||]] Label=part_bystarCapability Content Processing --\\ A ByStar/BISOS/Blee Capability Bundle (BCB) body=text -- Part TOC
\end{whenOrg}

\begin{htmlonly}

\begin{frame}[fragile,plain,label=part_bystarCapability]
    \frameaudio{"audio/part_bystarCapability.mp3"}
    \frametitle{}
    \framesubtitle{}
    \begin{rawhtml}
<div class="center">
<img src="./disposition.gened/part_bystarCapability/slide-1.png" height="500">
</div>
    \end{rawhtml}
\end{frame}
\end{htmlonly}

\begin{verblatex}

\begin{frame}[fragile,label=part_bystarCapability]
    \frameaudio{"audio/part_bystarCapability.mp3"}
    \frametitle{Content Processing --\\ A ByStar/BISOS/Blee Capability Bundle (BCB)}
    \framesubtitle{}
%%BxPy: impressiveFrameParSet('part_bystarCapability', 'always', 'True')
%%BxPy: impressiveFrameParSet('part_bystarCapability', 'transition', 'UnSpecified')
%%%#+END:

    \tableofcontents

\pnote{

  Instead of the traditional model of giving you recipes in a DIY context towards the goal of creating
  content processing capabilities on top of what you may already have, I am doing the opposite.

  I am saying: take this whole BISOS and Blee thing, and in there you will also have the content processing capabilities that I am speaking of here.

}

\end{frame}
\end{verblatex}




%%%#+BEGIN: bx:dblock:lcnt:latex-section :mode "auto" :seg-title "ByStar/BISOS/Blee Capability: Content Production and Publication"
%%% Args: :class "book|pres+art" :langs "en+fa" :disabledP "false" :seg-title "str" :short-title "str" :label "auto"
\begin{whenOrg}
*  _[[elisp:(blee:menu-sel:outline:popupMenu)][±]]_ _[[elisp:(blee:menu-sel:navigation:popupMenu)][Ξ]]_ [[elisp:(outline-show-branches+toggle)][|=]] [[elisp:(bx:orgm:indirectBufOther)][|>]] *[[elisp:(blee:ppmm:org-mode-toggle)][|N]]*  Section    [[elisp:(outline-show-subtree+toggle)][||]]   /ByStar/BISOS/Blee Capability: Content Production and Publication/ ::  [[elisp:(org-cycle)][| ]]
\end{whenOrg}

\section{ByStar/BISOS/Blee Capability: Content Production and Publication}
%%%#+END:

%%%#+BEGIN: b:lcnt:pres:frame/derivedImage :title "Content Processing as a ByStar/BISOS/Blee Capability:" :subtitle "" :label "lcntCapability" :beamer "plain" :comment "body=itemize"
\begin{whenOrg}
*****  _[[elisp:(blee:menu-sel:outline:popupMenu)][±]]_ _[[elisp:(blee:menu-sel:navigation:popupMenu)][Ξ]]_ [[elisp:(outline-show-branches+toggle)][|=]] [[elisp:(bx:orgm:indirectBufOther)][|>]] *[[elisp:(blee:ppmm:org-mode-toggle)][|N]]*  derivedImage [[elisp:(outline-show-subtree+toggle)][||]] Label=lcntCapability Content Processing as a ByStar/BISOS/Blee Capability: body=itemize
\end{whenOrg}

\begin{htmlonly}

\begin{frame}[fragile,plain,label=lcntCapability]
    \frameaudio{"audio/lcntCapability.mp3"}
    \frametitle{}
    \framesubtitle{}
    \begin{rawhtml}
<div class="center">
<img src="./disposition.gened/lcntCapability/slide-1.png" height="500">
</div>
    \end{rawhtml}
\end{frame}
\end{htmlonly}

\begin{verblatex}

\begin{frame}[fragile,plain,label=lcntCapability]
    \frameaudio{"audio/lcntCapability.mp3"}
    \frametitle{Content Processing as a ByStar/BISOS/Blee Capability:}
    \framesubtitle{}
%%BxPy: impressiveFrameParSet('lcntCapability', 'always', 'True')
%%BxPy: impressiveFrameParSet('lcntCapability', 'transition', 'UnSpecified')
%%%#+END:


    \begin{block}{The Libre-Halaal ByStar Digital Ecosystem:}
           ByStar is about redecentralization of internet application services and autonomous \& private usage of these services in a virtual-edge oriented model.
    \end{block}

    \begin{block}{BISOS (ByStar Internet Services OS):}
           A layer on top of Debian which provides for creation of self-hosted edge environments and software-service continumms.
    \end{block}

    \begin{block}{Blee (ByStar Libre-Halaal Emacs Enviornment):}
           A layer on top of Emacs which creates a BISOS integrated usage environment.
    \end{block}

    \begin{exampleblock}{ByStar/BISOS/Blee Capability Bundle (BCB):}
          Content Generation and Publication Framework
    \end{exampleblock}


\pnote{

  So, at the top level we have our own autonomy and privacy directed digital ecosystem,
  which in contrast to the center oriented American digital ecosystem, is edge oriented.
  We call it: ``The Libre-Halaal ByStar Digital Ecosystem''.

  All the systems in ByStar, run BISOS (By* Internet Services OS), which is a layer on top of Debian.

  The usage environment of ByStar and BISOS is Blee which is a layer on top of Emacs.

  With those in place, we then create a capability bundle called Blee-LCNT.

  So, when you buy into Blee and BISOS, you will naturally also get these content processing
  capabilities --- without a need for any recipies or DIY effort.

}

\end{frame}
\end{verblatex}



%%%#+BEGIN: bx:dblock:lcnt:latex-section :mode "auto" :seg-title "ByStar Containment Hierarchy and ByStar Capability Bundles"
%%% Args: :class "book|pres+art" :langs "en+fa" :disabledP "false" :seg-title "str" :short-title "str" :label "auto"
\begin{whenOrg}
*  _[[elisp:(blee:menu-sel:outline:popupMenu)][±]]_ _[[elisp:(blee:menu-sel:navigation:popupMenu)][Ξ]]_ [[elisp:(outline-show-branches+toggle)][|=]] [[elisp:(bx:orgm:indirectBufOther)][|>]] *[[elisp:(blee:ppmm:org-mode-toggle)][|N]]*  Section    [[elisp:(outline-show-subtree+toggle)][||]]   /ByStar Containment Hierarchy and ByStar Capability Bundles/ ::  [[elisp:(org-cycle)][| ]]
\end{whenOrg}

\section{ByStar Containment Hierarchy and ByStar Capability Bundles}
%%%#+END:

%%%#+BEGIN: b:lcnt:pres:frame/derivedImage :title "ByStar Containment Hierarchy and ByStar Capability Bundles" :subtitle "" :label "bxContainmentHierarchy" :comment "body=itemize" :reveal "plain" :beamer "plain"
\begin{whenOrg}
*****  _[[elisp:(blee:menu-sel:outline:popupMenu)][±]]_ _[[elisp:(blee:menu-sel:navigation:popupMenu)][Ξ]]_ [[elisp:(outline-show-branches+toggle)][|=]] [[elisp:(bx:orgm:indirectBufOther)][|>]] *[[elisp:(blee:ppmm:org-mode-toggle)][|N]]*  derivedImage [[elisp:(outline-show-subtree+toggle)][||]] Label=bxContainmentHierarchy ByStar Containment Hierarchy and ByStar Capability Bundles body=itemize
\end{whenOrg}

\begin{htmlonly}

\begin{frame}[fragile,plain,label=bxContainmentHierarchy]
    \frameaudio{"audio/bxContainmentHierarchy.mp3"}
    \frametitle{}
    \framesubtitle{}
    \begin{rawhtml}
<div class="center">
<img src="./disposition.gened/bxContainmentHierarchy/slide-1.png" height="500">
</div>
    \end{rawhtml}
\end{frame}
\end{htmlonly}

\begin{verblatex}

\begin{frame}[fragile,plain,label=bxContainmentHierarchy]
    \frameaudio{"audio/bxContainmentHierarchy.mp3"}
    \frametitle{ByStar Containment Hierarchy and ByStar Capability Bundles}
    \framesubtitle{}
%%BxPy: impressiveFrameParSet('bxContainmentHierarchy', 'always', 'True')
%%BxPy: impressiveFrameParSet('bxContainmentHierarchy', 'transition', 'UnSpecified')
%%%#+END:

%%%#+BEGIN: bx:dblock:lcnt:body:odg-artpres :beamerSize "max" :fig-file "/de/lcnt/lgpc/bystar/permanent/common/figures/bx-containment.odg"
\begin{comment}
******  [[elisp:(org-cycle)][| ]]  [[elisp:(blee:ppmm:org-mode-toggle)][Nat]] [[elisp:(beginning-of-buffer)][Top]] [[elisp:(delete-other-windows)][(1)]] || /Figure/ =ODG-ArtPres=  *bx-containment* -- ByStar Containment Hierarchy ::  [[elisp:(org-cycle)][| ]]
\end{comment}

\begin{presentationMode}

\begin{latexonly}
  \begin{figure}
    \begin{center}
       \includegraphics[width=145mm,keepaspectratio]{/de/lcnt/lgpc/bystar/permanent/common/figures/bx-containment}
    \end{center}
  \end{figure}
\end{latexonly}

\begin{htmlonly}
  \begin{rawhtml}
<div class="center">
<img src="/de/lcnt/lgpc/bystar/permanent/common/figures/bx-containment.png" height="450">
</div>
  \end{rawhtml}
\end{htmlonly}

\end{presentationMode}


\begin{articleMode}

\begin{latexonly}
  \begin{figure}[H]
    \begin{center}
      \includegraphics[width=\textwidth]{/de/lcnt/lgpc/bystar/permanent/common/figures/bx-containment}
      \caption{ByStar Containment Hierarchy}
      \label{fig:bx-containment}
    \end{center}
  \end{figure}
\end{latexonly}

\begin{htmlonly}
  %BEGIN IMAGE
  \begin{center}
      \includegraphics[width=\textwidth]{/de/lcnt/lgpc/bystar/permanent/common/figures/bx-containment}
  \end{center}
  %END IMAGE
  %HEVEA\imageflush

  \begin{figure}
      \caption{ByStar Containment Hierarchy}
      \label{fig:bx-containment}
  \end{figure}
\end{htmlonly}

\end{articleMode}

%%%#+END:

\pnote{

  If you were to look at the model that I introduced as containment hierarchies,
  it would look like this.

}

\end{frame}
\end{verblatex}



%%%#+BEGIN: bx:dblock:lcnt:latex-section :mode "auto" :seg-title "Aggregate Conviviality of ByStar Capabilities"
%%% Args: :class "book|pres+art" :langs "en+fa" :disabledP "false" :seg-title "str" :short-title "str" :label "auto"
\begin{whenOrg}
*  _[[elisp:(blee:menu-sel:outline:popupMenu)][±]]_ _[[elisp:(blee:menu-sel:navigation:popupMenu)][Ξ]]_ [[elisp:(outline-show-branches+toggle)][|=]] [[elisp:(bx:orgm:indirectBufOther)][|>]] *[[elisp:(blee:ppmm:org-mode-toggle)][|N]]*  Section    [[elisp:(outline-show-subtree+toggle)][||]]   /Aggregate Conviviality of ByStar Capabilities/ ::  [[elisp:(org-cycle)][| ]]
\end{whenOrg}

\section{Aggregate Conviviality of ByStar Capabilities}
%%%#+END:

%%%#+BEGIN: b:lcnt:pres:frame/derivedImage :title "" :subtitle "" :label "aggregateConviviality" :comment "body=itemize" :reveal "plain" :beamer "plain"
\begin{whenOrg}
*****  _[[elisp:(blee:menu-sel:outline:popupMenu)][±]]_ _[[elisp:(blee:menu-sel:navigation:popupMenu)][Ξ]]_ [[elisp:(outline-show-branches+toggle)][|=]] [[elisp:(bx:orgm:indirectBufOther)][|>]] *[[elisp:(blee:ppmm:org-mode-toggle)][|N]]*  derivedImage [[elisp:(outline-show-subtree+toggle)][||]] Label=aggregateConviviality  body=itemize
\end{whenOrg}

\begin{htmlonly}

\begin{frame}[fragile,plain,label=aggregateConviviality]
    \frameaudio{"audio/aggregateConviviality.mp3"}
    \frametitle{}
    \framesubtitle{}
    \begin{rawhtml}
<div class="center">
<img src="./disposition.gened/aggregateConviviality/slide-1.png" height="500">
</div>
    \end{rawhtml}
\end{frame}
\end{htmlonly}

\begin{verblatex}

\begin{frame}[fragile,plain,label=aggregateConviviality]
    \frameaudio{"audio/aggregateConviviality.mp3"}
    \frametitle{}
    \framesubtitle{}
%%BxPy: impressiveFrameParSet('aggregateConviviality', 'always', 'True')
%%BxPy: impressiveFrameParSet('aggregateConviviality', 'transition', 'UnSpecified')
%%%#+END:

%%%#+BEGIN: bx:dblock:lcnt:body:odg-artpres :beamerSize "max" :fig-file "/de/lcnt/lgpc/bystar/permanent/common/figures/bisosAndBlee.odg"
\begin{comment}
******  [[elisp:(org-cycle)][| ]]  [[elisp:(blee:ppmm:org-mode-toggle)][Nat]] [[elisp:(beginning-of-buffer)][Top]] [[elisp:(delete-other-windows)][(1)]] || /Figure/ =ODG-ArtPres=  *bisosAndBlee* -- BISOS and Blee are Intertwined ::  [[elisp:(org-cycle)][| ]]
\end{comment}

\begin{presentationMode}

\begin{latexonly}
  \begin{figure}
    \begin{center}
       \includegraphics[width=145mm,keepaspectratio]{/de/lcnt/lgpc/bystar/permanent/common/figures/bisosAndBlee}
    \end{center}
  \end{figure}
\end{latexonly}

\begin{htmlonly}
  \begin{rawhtml}
<div class="center">
<img src="/de/lcnt/lgpc/bystar/permanent/common/figures/bisosAndBlee.png" height="450">
</div>
  \end{rawhtml}
\end{htmlonly}

\end{presentationMode}


\begin{articleMode}

\begin{latexonly}
  \begin{figure}[H]
    \begin{center}
      \includegraphics[width=\textwidth]{/de/lcnt/lgpc/bystar/permanent/common/figures/bisosAndBlee}
      \caption{BISOS and Blee are Intertwined}
      \label{fig:bisosAndBlee}
    \end{center}
  \end{figure}
\end{latexonly}

\begin{htmlonly}
  %BEGIN IMAGE
  \begin{center}
      \includegraphics[width=\textwidth]{/de/lcnt/lgpc/bystar/permanent/common/figures/bisosAndBlee}
  \end{center}
  %END IMAGE
  %HEVEA\imageflush

  \begin{figure}
      \caption{BISOS and Blee are Intertwined}
      \label{fig:bisosAndBlee}
  \end{figure}
\end{htmlonly}

\end{articleMode}

%%%#+END:

\pnote{

  We love Emacs and we love Unix because their design is convivial.
  By convivial, I am referring to Ivan Illich's  concept and terminology of ``Tools for Conviviality''.
  It was first published in 1973. It's a must read.

  A goal of the design of the ByStar Digital Ecosystem is to enlarge the aggregated conviviality of its
  capabilities.

  What distinguishes Blee-Lcnt from other content processing tools and frameworks, is our emphasis on
  enhancing the aggregated conviviality. These tools let you express yourself. They let you be in charge.

}

\end{frame}
\end{verblatex}


%%%#+BEGIN: bx:dblock:lcnt:latex-section :mode "" :seg-title "Parts List: Integrated Components" :label ""
%%% Args: :class "book|pres+art" :langs "en+fa" :disabledP "false" :seg-title "str" :short-title "str" :label "auto"
\begin{whenOrg}
*  _[[elisp:(blee:menu-sel:outline:popupMenu)][±]]_ _[[elisp:(blee:menu-sel:navigation:popupMenu)][Ξ]]_ [[elisp:(outline-show-branches+toggle)][|=]] [[elisp:(bx:orgm:indirectBufOther)][|>]] *[[elisp:(blee:ppmm:org-mode-toggle)][|N]]*  Section    [[elisp:(outline-show-subtree+toggle)][||]]   /Parts List: Integrated Components/ ::  [[elisp:(org-cycle)][| ]]
\end{whenOrg}

\section{Parts List: Integrated Components}
%%%#+END:


%%%#+BEGIN: b:lcnt:pres:frame/derivedImage :mode "auto" :title "Parts List: Integrated Components" :subtitle "" :reveal "plain" :beamer "plain" :label "partsList" :transition "default"
\begin{whenOrg}
*****  _[[elisp:(blee:menu-sel:outline:popupMenu)][±]]_ _[[elisp:(blee:menu-sel:navigation:popupMenu)][Ξ]]_ [[elisp:(outline-show-branches+toggle)][|=]] [[elisp:(bx:orgm:indirectBufOther)][|>]] *[[elisp:(blee:ppmm:org-mode-toggle)][|N]]*  derivedImage [[elisp:(outline-show-subtree+toggle)][||]] Label=partsList Parts List: Integrated Components
\end{whenOrg}

\begin{htmlonly}

\begin{frame}[fragile,plain,label=partsList]
    \transition{default}
    \frameaudio{"audio/partsList.mp3"}
    \frametitle{}
    \framesubtitle{}
    \begin{rawhtml}
<div class="center">
<img src="./disposition.gened/partsList/slide-1.png" height="500">
</div>
    \end{rawhtml}
\end{frame}
\end{htmlonly}

\begin{verblatex}

\begin{frame}[fragile,plain,label=partsList]
    \transition{default}
    \frameaudio{"audio/partsList.mp3"}
    \frametitle{Parts List: Integrated Components}
    \framesubtitle{}
%%BxPy: impressiveFrameParSet('partsList', 'always', 'True')
%%BxPy: impressiveFrameParSet('partsList', 'transition', 'default')
%%%#+END:

\begin{description}
\item[TeX:] XeLaTeX, HaVeA, Beamer, bidi, {\color{violet}bxlcnt/whenenv}, {\color{violet}bxlcnt/bxtex}
\item[elisp:] AucTeX, RefTeX, yasnippet, {\color{violet}blee-dblock}, {\color{violet}blee.COMEEGA}
\item[Java-Script:] node-modules, reveal.js, audio-reveal, video.js
\item[Python-Integration:] {\color{violet}bisos.lcnt},
\item[Bash-Inetgration:] {\color{violet}bisos/lcnt/bin}, {\color{violet}seedLcntProc.sh}, {\color{violet}lcntProc.sh}
\item[Images:] LibreOffice Draw, Gimp
\item[Audio:] Audacity, {\color{violet}lcaAudioManage.sh}
\item[Video:] obs, ffmpeg, pdfpc (annotations)
\item [Internationalization:]  {\color{violet}perso-arabic input methods}, {\color{violet}HaVeA-Bidi}
\item[Teleprompter:] QPrompt, pdfpc (notes)
\item[Captions:] subed, OpenAI Whisper, aeneas
\item[Documentation:] {\color{violet}Blee Panels}
\end{description}

\pnote{

  Here is our parts list. These are the components that we have chosen to bring together
  towards our goal of creating convivial tools. In this slide we are using black ink to
  denote exisiting tools and we use violet ink to denote pieces that we have developed
  towards cohesive integration.

}

\end{frame}
\end{verblatex}


% %%%#+BEGIN: bx:dblock:lcnt:latex-section :mode "auto" :seg-title "Models Tussle: Org as Primary vs LaTeX as Primary"
% %%% Args: :class "book|pres+art" :langs "en+fa" :disabledP "false" :seg-title "str" :short-title "str" :label "auto"
% \begin{whenOrg}
% *  _[[elisp:(blee:menu-sel:outline:popupMenu)][±]]_ _[[elisp:(blee:menu-sel:navigation:popupMenu)][Ξ]]_ [[elisp:(outline-show-branches+toggle)][|=]] [[elisp:(bx:orgm:indirectBufOther)][|>]] *[[elisp:(blee:ppmm:org-mode-toggle)][|N]]*  Section    [[elisp:(outline-show-subtree+toggle)][||]]   /Models Tussle: Org as Primary vs LaTeX as Primary/ ::  [[elisp:(org-cycle)][| ]]
% \end{whenOrg}

% \section{Models Tussle: Org as Primary vs LaTeX as Primary}
% %%%#+END:



% %%%#+BEGIN: b:lcnt:pres:frame/derivedImage :title "Models Tussle: Org as Primary vs LaTeX as Primary" :subtitle "A Bigger and Different Vision for Emacs" :label "modelsTussle" :reveal "plain" :beamer "plain" :comment "body=table"
% \begin{whenOrg}
% *****  _[[elisp:(blee:menu-sel:outline:popupMenu)][±]]_ _[[elisp:(blee:menu-sel:navigation:popupMenu)][Ξ]]_ [[elisp:(outline-show-branches+toggle)][|=]] [[elisp:(bx:orgm:indirectBufOther)][|>]] *[[elisp:(blee:ppmm:org-mode-toggle)][|N]]*  derivedImage [[elisp:(outline-show-subtree+toggle)][||]] Label=modelsTussle Models Tussle: Org as Primary vs LaTeX as Primary body=table
% \end{whenOrg}

% \begin{htmlonly}

% \begin{frame}[fragile,plain,label=modelsTussle]
%     \frameaudio{"audio/modelsTussle.mp3"}
%     \frametitle{}
%     \framesubtitle{}
%     \begin{rawhtml}
% <div class="center">
% <img src="./disposition.gened/modelsTussle/slide-1.png" height="500">
% </div>
%     \end{rawhtml}
% \end{frame}
% \end{htmlonly}

% \begin{verblatex}

% \begin{frame}[fragile,plain,label=modelsTussle]
%     \frameaudio{"audio/modelsTussle.mp3"}
%     \frametitle{Models Tussle: Org as Primary vs LaTeX as Primary}
%     \framesubtitle{A Bigger and Different Vision for Emacs}
% %%BxPy: impressiveFrameParSet('modelsTussle', 'always', 'True')
% %%BxPy: impressiveFrameParSet('modelsTussle', 'transition', 'UnSpecified')
% %%%#+END:


% \begin{small}
% %\begin{table}[h]
% \begin{tabular}{|p{4in}@{}|@{}p{4in}|}
%  %  BEGIN RECEIVE ORGTBL newTable

% \toprule

%   \multicolumn{1}{|c|}{\color{black} \textbf{Common  Emacs/FOSS Culture}} &
%   \multicolumn{1}{c|}{\color{blue} \textbf{Libre-Halaal ByStar Digital Ecosystem}} \\

% \midrule[.15em]

%   \multicolumn{1}{|l|}{\color{violet} \textbf{Org-Mode + LaTeX}} &
%   \multicolumn{1}{l|}{\color{teal} \textbf{LaTeX + Org-Mode}} \\

% % \midrule

%   \multicolumn{1}{|l|}{\color{violet} \textbf{Org-Babel}} &
%   \multicolumn{1}{l|}{\color{teal} \textbf{LaTeX-COMEEGA}} \\

% % \midrule

%   \multicolumn{1}{|l|} {\color{violet} \textbf{Literate Programming}} &
%   \multicolumn{1}{l|}{\color{teal} \textbf{Surrounded Programming}} \\


% \midrule[.15em]

%   \multicolumn{1}{|l|} {\color{violet} \textbf{DIY Recipes for Content Generation}} &
%   \multicolumn{1}{l|}{\color{teal} \textbf{Content Generation as a BISOS Capability}} \\

% \midrule[.15em]

%   \multicolumn{1}{|l|} {\color{violet} \textbf{Multi-Platform:}} &
%   \multicolumn{1}{l|}{\color{teal} \textbf{Universal Debian Only -}} \\

%   \multicolumn{1}{|l|} {\color{violet} \textbf{Linux, Windows, Mac, etc.}} &
%   \multicolumn{1}{l|}{\color{teal} \textbf{Cohesive Software-Service Continuums}} \\


% \midrule[.15em]

%   \multicolumn{1}{|l|} {\color{violet} \textbf{Inside of Emacs Integration}} &
%   \multicolumn{1}{l|}{\color{teal} \textbf{Around Emacs Integration}} \\

% \midrule[.15em]

%   \multicolumn{1}{|l|} {\color{violet} \textbf{Niche Solutions}} &
%   \multicolumn{1}{l|}{\color{teal} \textbf{Comprehensive and Complete Solutions}} \\



% \bottomrule

%    %    END RECEIVE ORGTBL newTable
% \end{tabular}
% %\caption{Step 2 -- Activities}
% \label{tab:stepTwo}
% %\end{table}
% \end{small}


% \pnote{

%   The choices that we have made and the model that we follow is different from the common Emacs and FOSS culture.

%   For content development many use Org-Mode+LaTeX which leads to the models of  Babel and Literate programming.
%   With Blee-Lcnt, we have done the opposite, LaTeX+Org-Mode leading to the models of COMEEGA and Surrounded programming.
%   I'll expand on all of this shortly.

% }

%   % Org-Mode has done an excellent job of integrating with various other modes, including with LaTeX.


% \end{frame}
% \end{verblatex}



%%%#+BEGIN: bx:dblock:lcnt:latex-part :disabledP "false" :seg-title "Resulting Contents -- Output Forms and Formats"
%%% Args: :toc "NU" :tocDepth 3 :part "NU" :label "auto|spec" :partpage t
\begin{whenOrg}
*      ================
*  [[elisp:(blee:ppmm:org-mode-toggle)][|n]] [[elisp:(blee:menu-sel:outline:popupMenu)][+-]] [[elisp:(blee:menu-sel:navigation:popupMenu)][==]]  *Part*   _Resulting Contents -- Output Forms and Formats_ ::  [[elisp:(org-cycle)][| ]]
\end{whenOrg}

\newpage
\part{Resulting Contents -- Output Forms and Formats}
%%%#+END:


%%%#+BEGIN: b:lcnt:pres:frame/derivedImage :title "Resulting Contents -- Output Forms and Formats" :subtitle "" :label "part_outputFormats" :comment "body=text -- Part TOC" :reveal "plain" :beamer ""
\begin{whenOrg}
*****  _[[elisp:(blee:menu-sel:outline:popupMenu)][±]]_ _[[elisp:(blee:menu-sel:navigation:popupMenu)][Ξ]]_ [[elisp:(outline-show-branches+toggle)][|=]] [[elisp:(bx:orgm:indirectBufOther)][|>]] *[[elisp:(blee:ppmm:org-mode-toggle)][|N]]*  derivedImage [[elisp:(outline-show-subtree+toggle)][||]] Label=part_outputFormats Resulting Contents -- Output Forms and Formats body=text -- Part TOC
\end{whenOrg}

\begin{htmlonly}

\begin{frame}[fragile,plain,label=part_outputFormats]
    \frameaudio{"audio/part_outputFormats.mp3"}
    \frametitle{}
    \framesubtitle{}
    \begin{rawhtml}
<div class="center">
<img src="./disposition.gened/part_outputFormats/slide-1.png" height="500">
</div>
    \end{rawhtml}
\end{frame}
\end{htmlonly}

\begin{verblatex}

\begin{frame}[fragile,label=part_outputFormats]
    \frameaudio{"audio/part_outputFormats.mp3"}
    \frametitle{Resulting Contents -- Output Forms and Formats}
    \framesubtitle{}
%%BxPy: impressiveFrameParSet('part_outputFormats', 'always', 'True')
%%BxPy: impressiveFrameParSet('part_outputFormats', 'transition', 'UnSpecified')
%%%#+END:

    \tableofcontents

\pnote{

  To see how all of this comes together let's first consider what gets produced.

  For example, for this video, the video is just one of the outputs. There are other outputs as well.

}

\end{frame}
\end{verblatex}



%%%#+BEGIN: bx:dblock:lcnt:latex-section :mode "auto" :seg-title "Content Processing and Output Forms and Formats"
%%% Args: :class "book|pres+art" :langs "en+fa" :disabledP "false" :seg-title "str" :short-title "str" :label "auto"
\begin{whenOrg}
*  _[[elisp:(blee:menu-sel:outline:popupMenu)][±]]_ _[[elisp:(blee:menu-sel:navigation:popupMenu)][Ξ]]_ [[elisp:(outline-show-branches+toggle)][|=]] [[elisp:(bx:orgm:indirectBufOther)][|>]] *[[elisp:(blee:ppmm:org-mode-toggle)][|N]]*  Section    [[elisp:(outline-show-subtree+toggle)][||]]   /Content Processing and Output Forms and Formats/ ::  [[elisp:(org-cycle)][| ]]
\end{whenOrg}

\section{Content Processing and Output Forms and Formats}
%%%#+END:


%%%#+BEGIN: b:lcnt:pres:frame/derivedImage :reveal "plain" :beamer "plain" :mode "auto" :title "Content Processing and Output Forms and Formats" :subtitle "" :label "figmmDocOutputForms"
\begin{whenOrg}
*****  _[[elisp:(blee:menu-sel:outline:popupMenu)][±]]_ _[[elisp:(blee:menu-sel:navigation:popupMenu)][Ξ]]_ [[elisp:(outline-show-branches+toggle)][|=]] [[elisp:(bx:orgm:indirectBufOther)][|>]] *[[elisp:(blee:ppmm:org-mode-toggle)][|N]]*  derivedImage [[elisp:(outline-show-subtree+toggle)][||]] Label=figmmDocOutputForms Content Processing and Output Forms and Formats
\end{whenOrg}

\begin{htmlonly}

\begin{frame}[fragile,plain,label=figmmDocOutputForms]
    \frameaudio{"audio/figmmDocOutputForms.mp3"}
    \frametitle{}
    \framesubtitle{}
    \begin{rawhtml}
<div class="center">
<img src="./disposition.gened/figmmDocOutputForms/slide-1.png" height="500">
</div>
    \end{rawhtml}
\end{frame}
\end{htmlonly}

\begin{verblatex}

\begin{frame}[fragile,plain,label=figmmDocOutputForms]
    \frameaudio{"audio/figmmDocOutputForms.mp3"}
    \frametitle{Content Processing and Output Forms and Formats}
    \framesubtitle{}
%%BxPy: impressiveFrameParSet('figmmDocOutputForms', 'always', 'True')
%%BxPy: impressiveFrameParSet('figmmDocOutputForms', 'transition', 'UnSpecified')
%%%#+END:

%%%#+BEGIN: bx:dblock:lcnt:body:odg-artpres  :fig-file "./figures/bxMmDocPublish.odg"
\begin{comment}
******  [[elisp:(org-cycle)][| ]]  [[elisp:(blee:ppmm:org-mode-toggle)][Nat]] [[elisp:(beginning-of-buffer)][Top]] [[elisp:(delete-other-windows)][(1)]] || /Figure/ =ODG-ArtPres=  *bxMmDocPublish* -- ByStar Multimedia Document Publication And Distribution ::  [[elisp:(org-cycle)][| ]]
\end{comment}

\begin{presentationMode}

\begin{latexonly}
  \begin{figure}
    \begin{center}
       \includegraphics[width=108mm,keepaspectratio]{./figures/bxMmDocPublish}
    \end{center}
  \end{figure}
\end{latexonly}

\begin{htmlonly}
  \begin{rawhtml}
<div class="center">
<img src="./figures/bxMmDocPublish.png" height="450">
</div>
  \end{rawhtml}
\end{htmlonly}

\end{presentationMode}


\begin{articleMode}

\begin{latexonly}
  \begin{figure}[H]
    \begin{center}
      \includegraphics[width=\textwidth]{./figures/bxMmDocPublish}
      \caption{ByStar Multimedia Document Publication And Distribution}
      \label{fig:bxMmDocPublish}
    \end{center}
  \end{figure}
\end{latexonly}

\begin{htmlonly}
  %BEGIN IMAGE
  \begin{center}
      \includegraphics[width=\textwidth]{./figures/bxMmDocPublish}
  \end{center}
  %END IMAGE
  %HEVEA\imageflush

  \begin{figure}
      \caption{ByStar Multimedia Document Publication And Distribution}
      \label{fig:bxMmDocPublish}
  \end{figure}
\end{htmlonly}

\end{articleMode}

%%%#+END:


\pnote{

  In this figure, the outputs are shown in the top layer.

  Using this video as an example, this presentation's output also include
  the ``Presentation Form'' and the ``Article-Presentation Form''.

  Let's look at these more closely.

}


\end{frame}
\end{verblatex}

%%%#+BEGIN: bx:dblock:lcnt:latex-section :mode "" :seg-title "Resulting Output Formats" :label ""
%%% Args: :class "book|pres+art" :langs "en+fa" :disabledP "false" :seg-title "str" :short-title "str" :label "auto"
\begin{whenOrg}
*  _[[elisp:(blee:menu-sel:outline:popupMenu)][±]]_ _[[elisp:(blee:menu-sel:navigation:popupMenu)][Ξ]]_ [[elisp:(outline-show-branches+toggle)][|=]] [[elisp:(bx:orgm:indirectBufOther)][|>]] *[[elisp:(blee:ppmm:org-mode-toggle)][|N]]*  Section    [[elisp:(outline-show-subtree+toggle)][||]]   /Resulting Output Formats/ ::  [[elisp:(org-cycle)][| ]]
\end{whenOrg}

\section{Resulting Output Formats}
%%%#+END:

%%%#+BEGIN: b:lcnt:pres:frame/derivedImage :mode "auto" :title "Resulting Output Formats" :subtitle "" :label "resultingOutputs" :transition "default" :reveal "plain" :beamer "plain"
\begin{whenOrg}
*****  _[[elisp:(blee:menu-sel:outline:popupMenu)][±]]_ _[[elisp:(blee:menu-sel:navigation:popupMenu)][Ξ]]_ [[elisp:(outline-show-branches+toggle)][|=]] [[elisp:(bx:orgm:indirectBufOther)][|>]] *[[elisp:(blee:ppmm:org-mode-toggle)][|N]]*  derivedImage [[elisp:(outline-show-subtree+toggle)][||]] Label=resultingOutputs Resulting Output Formats
\end{whenOrg}

\begin{htmlonly}

\begin{frame}[fragile,plain,label=resultingOutputs]
    \transition{default}
    \frameaudio{"audio/resultingOutputs.mp3"}
    \frametitle{}
    \framesubtitle{}
    \begin{rawhtml}
<div class="center">
<img src="./disposition.gened/resultingOutputs/slide-1.png" height="500">
</div>
    \end{rawhtml}
\end{frame}
\end{htmlonly}

\begin{verblatex}

\begin{frame}[fragile,plain,label=resultingOutputs]
    \transition{default}
    \frameaudio{"audio/resultingOutputs.mp3"}
    \frametitle{Resulting Output Formats}
    \framesubtitle{}
%%BxPy: impressiveFrameParSet('resultingOutputs', 'always', 'True')
%%BxPy: impressiveFrameParSet('resultingOutputs', 'transition', 'default')
%%%#+END:

    Primary Output Forms and Formats:

    \bigskip

    \begin{itemize}
    \item  Video Form
        \begin{itemize}
          \item Screen Capture of Reveal.js
        \end{itemize}
    \item  Presentation Form
        \begin{itemize}
          \item Reveal.js  -- HeVeA html output -- Interactive viewing
          \item Pdf  -- Beamer-LaTeX output -- pdf viewing and printing
        \end{itemize}
    \item  Article-Presentation Form
        \begin{itemize}
          \item Html  -- HeVeA html article output -- Includes: images, notes, audio and video
          \item Pdf  -- Article-LaTeX output -- Include images and notes
        \end{itemize}
    \end{itemize}


\pnote{

  For Presentations, there are 3 different forms. The Video Form, the Presentation From and the Article-Presentation Form.
  The Presentation Form produces both a pdf output and Reveal output.

  Next we will walkthrough some of the benefits that availability of these forms and formats provide.

}


\end{frame}
\end{verblatex}



%%%#+BEGIN: bx:dblock:lcnt:latex-section :mode "" :seg-title "Output Forms Walkthrough" :label ""
%%% Args: :class "book|pres+art" :langs "en+fa" :disabledP "false" :seg-title "str" :short-title "str" :label "auto"
\begin{whenOrg}
*  _[[elisp:(blee:menu-sel:outline:popupMenu)][±]]_ _[[elisp:(blee:menu-sel:navigation:popupMenu)][Ξ]]_ [[elisp:(outline-show-branches+toggle)][|=]] [[elisp:(bx:orgm:indirectBufOther)][|>]] *[[elisp:(blee:ppmm:org-mode-toggle)][|N]]*  Section    [[elisp:(outline-show-subtree+toggle)][||]]   /Output Forms Walkthrough/ ::  [[elisp:(org-cycle)][| ]]
\end{whenOrg}

\section{Output Forms Walkthrough}
%%%#+END:

%%%#+BEGIN: b:lcnt:pres:frame:begin/blank  :label "outputFormsWalkthrough" :comment "body=video"
\begin{whenOrg}
*****  _[[elisp:(blee:menu-sel:outline:popupMenu)][±]]_ _[[elisp:(blee:menu-sel:navigation:popupMenu)][Ξ]]_ [[elisp:(outline-show-branches+toggle)][|=]] [[elisp:(bx:orgm:indirectBufOther)][|>]] *[[elisp:(blee:ppmm:org-mode-toggle)][|N]]*  /Frame:begin-blank/ [[elisp:(outline-show-subtree+toggle)][||]] *Label=outputFormsWalkthrough* UnSpecified -- body=video
\end{whenOrg}

\begin{frame}[fragile,plain,label=outputFormsWalkthrough]
    \frametitle{}
%%BxPy: impressiveFrameParSet('outputFormsWalkthrough', 'always', 'True')
%%BxPy: impressiveFrameParSet('outputFormsWalkthrough', 'transition', 'UnSpecified')
%%%#+END:

%%%#+BEGIN: b:lcnt:pres:frame:body:mm/video :videoPath "./video/outputWalkthrough.mp4" :comment ""
\begin{whenOrg}
******  _[[elisp:(blee:menu-sel:outline:popupMenu)][±]]_ _[[elisp:(blee:menu-sel:navigation:popupMenu)][Ξ]]_ [[elisp:(outline-show-branches+toggle)][|=]] [[elisp:(bx:orgm:indirectBufOther)][|>]] *[[elisp:(blee:ppmm:org-mode-toggle)][|N]]*  FrmCntnt-Video [[elisp:(outline-show-subtree+toggle)][||]] Label=UnSpecified UnSpecified
\end{whenOrg}
\begin{presentationMode}
\begin{htmlonly}
  \begin{rawhtml}
<video preload="auto" data-audio-controls src="./video/outputWalkthrough.mp4"></video>
  \end{rawhtml}
\end{htmlonly}
\end{presentationMode}

\begin{articleMode}
\begin{htmlonly}
  \begin{rawhtml}
      <!-- data-autoplay  controls -->
    <p>
    <video  controls   preload="auto"  src="./video/outputWalkthrough.mp4"  height="50%%" width="50%%">
    </video>
    </p>
  \end{rawhtml}
\end{htmlonly}
\end{articleMode}

\begin{presentationMode}
\begin{latexonly}
    \begin{center}
      Video File: ./video/outputWalkthrough.mp4
    \end{center}
\end{latexonly}
\end{presentationMode}

\begin{articleMode}
\begin{latexonly}
    \begin{center}
      Video File: ./video/outputWalkthrough.mp4
    \end{center}
\end{latexonly}
\end{articleMode}
%%%#+END:

\pnote{

The video presentation that you are watching is just one of the outputs of the Blee-LCNT
machinery. There are two PDF format outputs and two HTML outputs that are also quite useful.

The primary output of Beamer is a set of slides that people use to give their talks with.
Typically that's done live. In my case I dissect the images of each frame and do a voiceover on it
and then dispense it through reveal. In a second you will see that as well.

This PDF output is very useful. You get the table of contents of course and in addition to that
Beamer generates navigations for you where on any part you get a small table of content
as well.

This is heavily used amongst academics and it's a good output on its own and I'm augmenting
it in a variety of ways. In addition to the presentation PDF format there is also an
article-presentation PDF format which gives you the same content but it gives it to you in
a textual form with the table of content and the rest. This is a good form to use when you are giving
for example class lectures and the students often prefer this format.
Now for the HTML format outputs the most relevant of course is the reveal itself. If you have not used reveal
before. In my view it's a HTML slide dispenser. I don't look at it as a presentation framework.
I use, as you are seeing, we use Beamer to feed into it and we use it to dispense the information.
It has all the typical navigation capabilities that you would expect and most of
what I have as slides are images but occasionally particularly when there is a need to provide
pointers, HTML pointers. I then also include a textual output. This is also produced from the Beamer
latex source but it's HTML through textual HTML through HeVeA not the image.
You can you get a table of contents. You can navigate and there are a whole lot of other features
that reveal also provides. So to generate the video, what I do is I come to the very beginning of
the presentation. I turn on the screen capture recorder and then I start playing the voiceover for each
slide and at the very end you get a video but what you just did is you dispensed every frame
one at a time through reveal. In addition to this HTML form you also get a article presentation form of it
with a full table of content and the videos are there and the notes are there and this is also quite useful.


}

\end{frame}


%%%#+BEGIN: bx:dblock:lcnt:latex-part :disabledP "false" :seg-title "A Unified Single Input -- A Sequence of Frames"
%%% Args: :toc "NU" :tocDepth 3 :part "NU" :label "auto|spec" :partpage t
\begin{whenOrg}
*      ================
*  [[elisp:(blee:ppmm:org-mode-toggle)][|n]] [[elisp:(blee:menu-sel:outline:popupMenu)][+-]] [[elisp:(blee:menu-sel:navigation:popupMenu)][==]]  *Part*   _A Unified Single Input -- A Sequence of Frames_ ::  [[elisp:(org-cycle)][| ]]
\end{whenOrg}

\newpage
\part{A Unified Single Input -- A Sequence of Frames}
%%%#+END:


%%%#+BEGIN: b:lcnt:pres:frame/derivedImage :title "A Unified Single Input -- A Sequence of Frames" :subtitle "" :label "part_inputFormats" :comment "body=text -- Part TOC" :reveal "plain" :beamer ""
\begin{whenOrg}
*****  _[[elisp:(blee:menu-sel:outline:popupMenu)][±]]_ _[[elisp:(blee:menu-sel:navigation:popupMenu)][Ξ]]_ [[elisp:(outline-show-branches+toggle)][|=]] [[elisp:(bx:orgm:indirectBufOther)][|>]] *[[elisp:(blee:ppmm:org-mode-toggle)][|N]]*  derivedImage [[elisp:(outline-show-subtree+toggle)][||]] Label=part_inputFormats A Unified Single Input -- A Sequence of Frames body=text -- Part TOC
\end{whenOrg}

\begin{htmlonly}

\begin{frame}[fragile,plain,label=part_inputFormats]
    \frameaudio{"audio/part_inputFormats.mp3"}
    \frametitle{}
    \framesubtitle{}
    \begin{rawhtml}
<div class="center">
<img src="./disposition.gened/part_inputFormats/slide-1.png" height="500">
</div>
    \end{rawhtml}
\end{frame}
\end{htmlonly}

\begin{verblatex}

\begin{frame}[fragile,label=part_inputFormats]
    \frameaudio{"audio/part_inputFormats.mp3"}
    \frametitle{A Unified Single Input -- A Sequence of Frames}
    \framesubtitle{}
%%BxPy: impressiveFrameParSet('part_inputFormats', 'always', 'True')
%%BxPy: impressiveFrameParSet('part_inputFormats', 'transition', 'UnSpecified')
%%%#+END:

    \tableofcontents

\pnote{

  Now, let's look at the one single input file that produced all of the outputs that we just saw.

}

\end{frame}
\end{verblatex}



%%%#+BEGIN: bx:dblock:lcnt:latex-section :mode "auto" :seg-title "Obtaining Sources Of This Document -- Pointers To Git Repos"
%%% Args: :class "book|pres+art" :langs "en+fa" :disabledP "false" :seg-title "str" :short-title "str" :label "auto"
\begin{whenOrg}
*  _[[elisp:(blee:menu-sel:outline:popupMenu)][±]]_ _[[elisp:(blee:menu-sel:navigation:popupMenu)][Ξ]]_ [[elisp:(outline-show-branches+toggle)][|=]] [[elisp:(bx:orgm:indirectBufOther)][|>]] *[[elisp:(blee:ppmm:org-mode-toggle)][|N]]*  Section    [[elisp:(outline-show-subtree+toggle)][||]]   /Obtaining Sources Of This Document -- Pointers To Git Repos/ ::  [[elisp:(org-cycle)][| ]]
\end{whenOrg}

\section{Obtaining Sources Of This Document -- Pointers To Git Repos}
%%%#+END:


%%%#+BEGIN: b:lcnt:pres:frame/insertDerivedImage :reveal "plain" :beamer "plain" :mode "auto" :title "Obtaining Sources Of This Document -- Pointers To Git Repos" :subtitle "" :label "obtainingSourcesInsert" :derivedLabel "obtainingSources"
\begin{whenOrg}
*****  _[[elisp:(blee:menu-sel:outline:popupMenu)][±]]_ _[[elisp:(blee:menu-sel:navigation:popupMenu)][Ξ]]_ [[elisp:(outline-show-branches+toggle)][|=]] [[elisp:(bx:orgm:indirectBufOther)][|>]] *[[elisp:(blee:ppmm:org-mode-toggle)][|N]]*  insertDerivedImage [[elisp:(outline-show-subtree+toggle)][||]] Label=obtainingSourcesInsert Obtaining Sources Of This Document -- Pointers To Git Repos
\end{whenOrg}

\begin{frame}[fragile,plain,label=obtainingSourcesInsert]
    \frameaudio{"audio/obtainingSourcesInsert.mp3"}
    \frametitle{}
    \framesubtitle{}


\begin{htmlonly} %% Must be at begining of line
    \begin{rawhtml}
<div class="center">
<img src="./disposition.gened/obtainingSources/slide-1.png" height="500">
</div>
    \end{rawhtml}
\end{htmlonly}

\begin{latexonly}

Reveal Image Insertion Of src="./disposition.gened/obtainingSources/slide-1.png"

\end{latexonly}

%%%#+END:

\pnote{

  I have put both the input file and some of the output files for this presentation on github.

  Here are some links to these repos and files.

}

\end{frame}


%%%#+BEGIN: b:lcnt:pres:frame:begin/plain  :reveal "plain" :beamer "plain" :mode "auto" :title "Obtaining Sources Of This Document -- Pointers To Git Repos" :subtitle "" :label "obtainingSources"
\begin{whenOrg}
*****  _[[elisp:(blee:menu-sel:outline:popupMenu)][±]]_ _[[elisp:(blee:menu-sel:navigation:popupMenu)][Ξ]]_ [[elisp:(outline-show-branches+toggle)][|=]] [[elisp:(bx:orgm:indirectBufOther)][|>]] *[[elisp:(blee:ppmm:org-mode-toggle)][|N]]*  /Frame:begin-plain/ [[elisp:(outline-show-subtree+toggle)][||]] *Label=obtainingSources* Obtaining Sources Of This Document -- Pointers To Git Repos --
\end{whenOrg}

\begin{frame}[fragile,plain,label=obtainingSources]
    \frameaudio{"audio/obtainingSources.mp3"}
    \frametitle{Obtaining Sources Of This Document -- Pointers To Git Repos}
    \framesubtitle{}
%%BxPy: impressiveFrameParSet('obtainingSources', 'always', 'True')
%%BxPy: impressiveFrameParSet('obtainingSources', 'transition', 'UnSpecified')
%%%#+END:

    \url{https://github.com/bxplpc/180068}

    \bigskip

    \url{https://github.com/bxGenesis/start}

    \url{https://github.com/bxlcnt}

    \url{https://github.com/bx-blee}

    \url{https://github.com/bisos-pip}

    \bigskip

    \url{https://github.com/mohsenBanan}

    \url{http://www.by-star.net}

\pnote{

    And here are the same links as a native Reveal slide.

}


\end{frame}


%%%#+BEGIN: bx:dblock:lcnt:latex-section :mode "auto" :seg-title "Overview Of ByStar Multi-Media Document Authorship and Generation"
%%% Args: :class "book|pres+art" :langs "en+fa" :disabledP "false" :seg-title "str" :short-title "str" :label "auto"
\begin{whenOrg}
*  _[[elisp:(blee:menu-sel:outline:popupMenu)][±]]_ _[[elisp:(blee:menu-sel:navigation:popupMenu)][Ξ]]_ [[elisp:(outline-show-branches+toggle)][|=]] [[elisp:(bx:orgm:indirectBufOther)][|>]] *[[elisp:(blee:ppmm:org-mode-toggle)][|N]]*  Section    [[elisp:(outline-show-subtree+toggle)][||]]   /Overview Of ByStar Multi-Media Document Authorship and Generation/ ::  [[elisp:(org-cycle)][| ]]
\end{whenOrg}

\section{Overview Of ByStar Multi-Media Document Authorship and Generation}
%%%#+END:


%%%#+BEGIN: b:lcnt:pres:frame/derivedImage :reveal "plain" :beamer "plain" :mode "auto" :title "By* Multi-Media Document Authorship and Generation" :subtitle "" :label "figmmDocProduction"
\begin{whenOrg}
*****  _[[elisp:(blee:menu-sel:outline:popupMenu)][±]]_ _[[elisp:(blee:menu-sel:navigation:popupMenu)][Ξ]]_ [[elisp:(outline-show-branches+toggle)][|=]] [[elisp:(bx:orgm:indirectBufOther)][|>]] *[[elisp:(blee:ppmm:org-mode-toggle)][|N]]*  derivedImage [[elisp:(outline-show-subtree+toggle)][||]] Label=figmmDocProduction By* Multi-Media Document Authorship and Generation
\end{whenOrg}

\begin{htmlonly}

\begin{frame}[fragile,plain,label=figmmDocProduction]
    \frameaudio{"audio/figmmDocProduction.mp3"}
    \frametitle{}
    \framesubtitle{}
    \begin{rawhtml}
<div class="center">
<img src="./disposition.gened/figmmDocProduction/slide-1.png" height="500">
</div>
    \end{rawhtml}
\end{frame}
\end{htmlonly}

\begin{verblatex}

\begin{frame}[fragile,plain,label=figmmDocProduction]
    \frameaudio{"audio/figmmDocProduction.mp3"}
    \frametitle{By* Multi-Media Document Authorship and Generation}
    \framesubtitle{}
%%BxPy: impressiveFrameParSet('figmmDocProduction', 'always', 'True')
%%BxPy: impressiveFrameParSet('figmmDocProduction', 'transition', 'UnSpecified')
%%%#+END:

%%%#+BEGIN: bx:dblock:lcnt:body:odg-artpres  :fig-file "./figures/bxMmDocProc.odg"
\begin{comment}
******  [[elisp:(org-cycle)][| ]]  [[elisp:(blee:ppmm:org-mode-toggle)][Nat]] [[elisp:(beginning-of-buffer)][Top]] [[elisp:(delete-other-windows)][(1)]] || /Figure/ =ODG-ArtPres=  *bxMmDocProc* -- ByStar Multimedia Document Authorship And Generation ::  [[elisp:(org-cycle)][| ]]
\end{comment}

\begin{presentationMode}

\begin{latexonly}
  \begin{figure}
    \begin{center}
       \includegraphics[width=108mm,keepaspectratio]{./figures/bxMmDocProc}
    \end{center}
  \end{figure}
\end{latexonly}

\begin{htmlonly}
  \begin{rawhtml}
<div class="center">
<img src="./figures/bxMmDocProc.png" height="450">
</div>
  \end{rawhtml}
\end{htmlonly}

\end{presentationMode}


\begin{articleMode}

\begin{latexonly}
  \begin{figure}[H]
    \begin{center}
      \includegraphics[width=\textwidth]{./figures/bxMmDocProc}
      \caption{ByStar Multimedia Document Authorship And Generation}
      \label{fig:bxMmDocProc}
    \end{center}
  \end{figure}
\end{latexonly}

\begin{htmlonly}
  %BEGIN IMAGE
  \begin{center}
      \includegraphics[width=\textwidth]{./figures/bxMmDocProc}
  \end{center}
  %END IMAGE
  %HEVEA\imageflush

  \begin{figure}
      \caption{ByStar Multimedia Document Authorship And Generation}
      \label{fig:bxMmDocProc}
  \end{figure}
\end{htmlonly}

\end{articleMode}

%%%#+END:

\pnote{

    This figure gives us an overview of how one set of inputs encapsulted in a single file
    can produce all of the outputs that we saw. The main TeX file shown at the bottom is
    processed by both XeLaTeX and by HeVeA. That main TeX file, in addition to LaTeX
    syntax also include org-mode constructs that facilitate addition of audio and video
    files. Later, I'll walkthrough the bodyPresArtEnFa.tex file that
    generated this very presentation with you.

}


\end{frame}
\end{verblatex}


%%%#+BEGIN: bx:dblock:lcnt:latex-section :mode "" :seg-title "Unified Source -- Combinations Of Abstract Forms And Concrete Outputs" :label ""
%%% Args: :class "book|pres+art" :langs "en+fa" :disabledP "false" :seg-title "str" :short-title "str" :label "auto"
\begin{whenOrg}
*  _[[elisp:(blee:menu-sel:outline:popupMenu)][±]]_ _[[elisp:(blee:menu-sel:navigation:popupMenu)][Ξ]]_ [[elisp:(outline-show-branches+toggle)][|=]] [[elisp:(bx:orgm:indirectBufOther)][|>]] *[[elisp:(blee:ppmm:org-mode-toggle)][|N]]*  Section    [[elisp:(outline-show-subtree+toggle)][||]]   /Unified Source -- Combinations Of Abstract Forms And Concrete Outputs/ ::  [[elisp:(org-cycle)][| ]]
\end{whenOrg}

\section{Unified Source -- Combinations Of Abstract Forms And Concrete Outputs}
 %%%#+END:

%%%#+BEGIN: b:lcnt:pres:frame/derivedImage :reveal "plain" :beamer "plain" :mode "auto" :title "Unified Source Controlling All Results" :subtitle "Combinations Of Abstract Formats And Concrete OutputsPaginated: Text+Images" :label "auto" :transition "default"
\begin{whenOrg}
*****  _[[elisp:(blee:menu-sel:outline:popupMenu)][±]]_ _[[elisp:(blee:menu-sel:navigation:popupMenu)][Ξ]]_ [[elisp:(outline-show-branches+toggle)][|=]] [[elisp:(bx:orgm:indirectBufOther)][|>]] *[[elisp:(blee:ppmm:org-mode-toggle)][|N]]*  derivedImage [[elisp:(outline-show-subtree+toggle)][||]] Label=auto Unified Source Controlling All Results
\end{whenOrg}

\begin{htmlonly}

\begin{frame}[fragile,plain,label=UnifiedSourceControllingAllResults]
    \transition{default}
    \frameaudio{"audio/UnifiedSourceControllingAllResults.mp3"}
    \frametitle{}
    \framesubtitle{}
    \begin{rawhtml}
<div class="center">
<img src="./disposition.gened/UnifiedSourceControllingAllResults/slide-1.png" height="500">
</div>
    \end{rawhtml}
\end{frame}
\end{htmlonly}

\begin{verblatex}

\begin{frame}[fragile,plain,label=UnifiedSourceControllingAllResults]
    \transition{default}
    \frameaudio{"audio/UnifiedSourceControllingAllResults.mp3"}
    \frametitle{Unified Source Controlling All Results}
    \framesubtitle{Combinations Of Abstract Formats And Concrete OutputsPaginated: Text+Images}
%%BxPy: impressiveFrameParSet('UnifiedSourceControllingAllResults', 'always', 'True')
%%BxPy: impressiveFrameParSet('UnifiedSourceControllingAllResults', 'transition', 'default')
%%%#+END:

    % Combinations, Selections And Control Of Language, Abstract Form And Concrete Output Formats

    \begin{itemize}
    \item Abstract Multilingualization Selections
      \begin{description}
      \item[Lr+Bidi:] Left-To-Right -- Globish/English+bidi -- article\&,presentation
      \item[Rl+Bidi:] Right-To-Left -- Farsi+bidi  -- article\&presentation
      \end{description}
    \end{itemize}

    \begin{itemize}
    \item Abstract Form Selections (Left-To-Right (EnFa))
      \begin{description}
      \item[presentation:] presentationEnfa.ttytex + bodyPresentationEnFa.tex
      \item[presentation + article:]  bodyPresArtEnFa.ttytex + bodyArticleEnFa.ttytex
      \end{description}
    \end{itemize}

    \begin{itemize}
    \item Single Source Abstract Form And Concrete Output Formats Control
      \begin{description}
      \item[Abstract Form Controls:] article-mode presentation-mode
      \item[Concrete Formats Only:] latex-only html-only raw-html
      \end{description}
    \end{itemize}


\pnote{

  When you construct that primary TeX file, there are several abstractions that you need
  to keep in mind. Is my presentation going to go from Left-To-Right or from Right-To-Left?
  Perso-Arabic presentations go from Right-To-Left. Another consideration is the types of
  forms of results that you want. Just the presentation or Article-Presentation as well?
  With those choices in place you can produce condition based text for each of your desired outputs.

}


\end{frame}
\end{verblatex}


%%%#+BEGIN: bx:dblock:lcnt:latex-section :mode "" :seg-title "Frame Control Types" :label ""
%%% Args: :class "book|pres+art" :langs "en+fa" :disabledP "false" :seg-title "str" :short-title "str" :label "auto"
\begin{whenOrg}
*  _[[elisp:(blee:menu-sel:outline:popupMenu)][±]]_ _[[elisp:(blee:menu-sel:navigation:popupMenu)][Ξ]]_ [[elisp:(outline-show-branches+toggle)][|=]] [[elisp:(bx:orgm:indirectBufOther)][|>]] *[[elisp:(blee:ppmm:org-mode-toggle)][|N]]*  Section    [[elisp:(outline-show-subtree+toggle)][||]]   /Frame Control Types/ ::  [[elisp:(org-cycle)][| ]]
\end{whenOrg}

\section{Frame Control Types}
%%%#+END:


%%%#+BEGIN: b:lcnt:pres:frame/derivedImage :mode "auto" :title "Frame Control Types" :subtitle "" :label "frameControlTypes"
\begin{whenOrg}
*****  _[[elisp:(blee:menu-sel:outline:popupMenu)][±]]_ _[[elisp:(blee:menu-sel:navigation:popupMenu)][Ξ]]_ [[elisp:(outline-show-branches+toggle)][|=]] [[elisp:(bx:orgm:indirectBufOther)][|>]] *[[elisp:(blee:ppmm:org-mode-toggle)][|N]]*  derivedImage [[elisp:(outline-show-subtree+toggle)][||]] Label=frameControlTypes Frame Control Types
\end{whenOrg}

\begin{htmlonly}

\begin{frame}[fragile,plain,label=frameControlTypes]
    \frameaudio{"audio/frameControlTypes.mp3"}
    \frametitle{}
    \framesubtitle{}
    \begin{rawhtml}
<div class="center">
<img src="./disposition.gened/frameControlTypes/slide-1.png" height="500">
</div>
    \end{rawhtml}
\end{frame}
\end{htmlonly}

\begin{verblatex}

\begin{frame}[fragile,label=frameControlTypes]
    \frameaudio{"audio/frameControlTypes.mp3"}
    \frametitle{Frame Control Types}
    \framesubtitle{}
%%BxPy: impressiveFrameParSet('frameControlTypes', 'always', 'True')
%%BxPy: impressiveFrameParSet('frameControlTypes', 'transition', 'UnSpecified')
%%%#+END:

\small
\begin{tabular}{|p{1.5in}|p{2.08in}|p{1.75in}|}
  \hline
  \textbf{Dblock} & \textbf{LaTeX} & \textbf{Reveal.js} \\
  \hline
  TitlePage & In presentationEnFa.ttytex & \\
  \hline
  frame:begin/blank & no title (text) & no title  HaVeA -> html \\
  \hline
  frame:begin/plain & title+subTitle & HaVeA -> html \\
  \hline
  frame:begin/regular & title+subTitle+Guide & HaVeA -> html \\
  \hline
  frame/derivedImage & frame head & brings beamer image \\
  \hline
  frame/derivedVideo & frame head & brings video of beamer image \\
  \hline
  frame/insertDerivedImage & text saying: image derivedLabel & rawhtml: image derivedLabel \\
                  & label: here, derivedLabel: inserted & derivedLabel HaVeA -> html \\
  \hline
  frame/insertDerivedVideo & text saying: video derivedLabel & rawhtml: video derivedLabel \\
                  & label: here, derivedLabel: inserted & derivedLabel HaVeA -> html \\
  \hline
\end{tabular}
\normalsize


\pnote{

  Think of this video presentation as a sequence of frames.
  Each frame is controlled by an org-mode dynamic block.
  This table lists available dblocks from which you can choose.
  For example, this particular frame that we are watching is controlled by b:lcnt:pres:frame/derivedImage.
  Beamer creates a pdf file that includes the image of this slide.
  That image is then injected into Reveal.
  And in the end a video of that image is produced with the narrations that I am uttering right now.
  All of this has similarly been applied to each and every frame that you have been watching.

}


\end{frame}
\end{verblatex}


%%%#+BEGIN: bx:dblock:lcnt:latex-section :mode "" :seg-title "Frame Body Types" :label ""
%%% Args: :class "book|pres+art" :langs "en+fa" :disabledP "false" :seg-title "str" :short-title "str" :label "auto"
\begin{whenOrg}
*  _[[elisp:(blee:menu-sel:outline:popupMenu)][±]]_ _[[elisp:(blee:menu-sel:navigation:popupMenu)][Ξ]]_ [[elisp:(outline-show-branches+toggle)][|=]] [[elisp:(bx:orgm:indirectBufOther)][|>]] *[[elisp:(blee:ppmm:org-mode-toggle)][|N]]*  Section    [[elisp:(outline-show-subtree+toggle)][||]]   /Frame Body Types/ ::  [[elisp:(org-cycle)][| ]]
\end{whenOrg}

\section{Frame Body Types}
%%%#+END:

%%%#+BEGIN: b:lcnt:pres:frame/derivedImage :mode "auto" :title "Frame Body Types" :subtitle "" :label "frameBodyTypes"
\begin{whenOrg}
*****  _[[elisp:(blee:menu-sel:outline:popupMenu)][±]]_ _[[elisp:(blee:menu-sel:navigation:popupMenu)][Ξ]]_ [[elisp:(outline-show-branches+toggle)][|=]] [[elisp:(bx:orgm:indirectBufOther)][|>]] *[[elisp:(blee:ppmm:org-mode-toggle)][|N]]*  derivedImage [[elisp:(outline-show-subtree+toggle)][||]] Label=frameBodyTypes Frame Body Types
\end{whenOrg}

\begin{htmlonly}

\begin{frame}[fragile,plain,label=frameBodyTypes]
    \frameaudio{"audio/frameBodyTypes.mp3"}
    \frametitle{}
    \framesubtitle{}
    \begin{rawhtml}
<div class="center">
<img src="./disposition.gened/frameBodyTypes/slide-1.png" height="500">
</div>
    \end{rawhtml}
\end{frame}
\end{htmlonly}

\begin{verblatex}

\begin{frame}[fragile,label=frameBodyTypes]
    \frameaudio{"audio/frameBodyTypes.mp3"}
    \frametitle{Frame Body Types}
    \framesubtitle{}
%%BxPy: impressiveFrameParSet('frameBodyTypes', 'always', 'True')
%%BxPy: impressiveFrameParSet('frameBodyTypes', 'transition', 'UnSpecified')
%%%#+END:

\small
\begin{tabular}{|p{1.2in}|p{1.75in}|p{1.75in}|}
  \hline
  \textbf{Dblock} & \textbf{LaTeX} & \textbf{HTML} \\
  \hline
  body:odg-artpres & BeamerSize - ArtSize & RevealSize - ArtSize \\
  \hline
  body:mm/video & Textual Video File Path & Sized for Reveal or Article \\
  \hline
  None & Condition Based LaTeX & Condition Based Html \\
  \hline
\end{tabular}
\normalsize


\pnote{

  Similar to Frame Controls, there are org-mode dynamic blocks for ``Frame Body Types''.
  You can easily insert an image which is typically created by OpenOffice Draw into a frame.
  Same with say a screen capture video.

}


\end{frame}
\end{verblatex}


%%%#+BEGIN: bx:dblock:lcnt:latex-part :disabledP "false" :seg-title "Generating and Processing the Content"
%%% Args: :toc "NU" :tocDepth 3 :part "NU" :label "auto|spec" :partpage t
\begin{whenOrg}
*      ================
*  [[elisp:(blee:ppmm:org-mode-toggle)][|n]] [[elisp:(blee:menu-sel:outline:popupMenu)][+-]] [[elisp:(blee:menu-sel:navigation:popupMenu)][==]]  *Part*   _Generating and Processing the Content_ ::  [[elisp:(org-cycle)][| ]]
\end{whenOrg}

\newpage
\part{Generating and Processing the Content}
%%%#+END:

%%%#+BEGIN: b:lcnt:pres:frame/derivedImage :title "Generating and Processing the Content" :subtitle "" :label "part_processingContent" :comment "body=text" :reveal "plain" :beamer ""
\begin{whenOrg}
*****  _[[elisp:(blee:menu-sel:outline:popupMenu)][±]]_ _[[elisp:(blee:menu-sel:navigation:popupMenu)][Ξ]]_ [[elisp:(outline-show-branches+toggle)][|=]] [[elisp:(bx:orgm:indirectBufOther)][|>]] *[[elisp:(blee:ppmm:org-mode-toggle)][|N]]*  derivedImage [[elisp:(outline-show-subtree+toggle)][||]] Label=part_processingContent Generating and Processing the Content body=text
\end{whenOrg}

\begin{htmlonly}

\begin{frame}[fragile,plain,label=part_processingContent]
    \frameaudio{"audio/part_processingContent.mp3"}
    \frametitle{}
    \framesubtitle{}
    \begin{rawhtml}
<div class="center">
<img src="./disposition.gened/part_processingContent/slide-1.png" height="500">
</div>
    \end{rawhtml}
\end{frame}
\end{htmlonly}

\begin{verblatex}

\begin{frame}[fragile,label=part_processingContent]
    \frameaudio{"audio/part_processingContent.mp3"}
    \frametitle{Generating and Processing the Content}
    \framesubtitle{}
%%BxPy: impressiveFrameParSet('part_processingContent', 'always', 'True')
%%BxPy: impressiveFrameParSet('part_processingContent', 'transition', 'UnSpecified')
%%%#+END:

    \tableofcontents

\pnote{

  Now that we have looked at the ``Outputs'' and the ``Inputs'', let's look at how the
  Outputs are generated from the Inputs.

  }
\end{frame}
\end{verblatex}


%%%#+BEGIN: bx:dblock:lcnt:latex-section :mode "auto" :seg-title "From Fresh Debian to Raw-BISOS and Raw-Blee"
%%% Args: :class "book|pres+art" :langs "en+fa" :disabledP "false" :seg-title "str" :short-title "str" :label "auto"
\begin{whenOrg}
*  _[[elisp:(blee:menu-sel:outline:popupMenu)][±]]_ _[[elisp:(blee:menu-sel:navigation:popupMenu)][Ξ]]_ [[elisp:(outline-show-branches+toggle)][|=]] [[elisp:(bx:orgm:indirectBufOther)][|>]] *[[elisp:(blee:ppmm:org-mode-toggle)][|N]]*  Section    [[elisp:(outline-show-subtree+toggle)][||]]   /From Fresh Debian to Raw-BISOS and Raw-Blee/ ::  [[elisp:(org-cycle)][| ]]
\end{whenOrg}

\section{From Fresh Debian to Raw-BISOS and Raw-Blee}
%%%#+END:


%%%#+BEGIN: b:lcnt:pres:frame/derivedVideo :title "Bootstrapping:\\\\ From Fresh Debian to Raw-BISOS and Raw-Blee" :subtitle "" :label "rawBleeBootstrap" :beamer "plain" :comment "body=derivedInsert"
\begin{whenOrg}
*****  _[[elisp:(blee:menu-sel:outline:popupMenu)][±]]_ _[[elisp:(blee:menu-sel:navigation:popupMenu)][Ξ]]_ [[elisp:(outline-show-branches+toggle)][|=]] [[elisp:(bx:orgm:indirectBufOther)][|>]] *[[elisp:(blee:ppmm:org-mode-toggle)][|N]]*  derivedVideo [[elisp:(outline-show-subtree+toggle)][||]] Label=rawBleeBootstrap Bootstrapping:\\ From Fresh Debian to Raw-BISOS and Raw-Blee body=derivedInsert
\end{whenOrg}

\begin{htmlonly}

\begin{frame}[fragile,plain,label=rawBleeBootstrap]
    \frametitle{}
    \framesubtitle{}
    \begin{presentationMode}
    \begin{rawhtml}
<video preload="auto" data-audio-controls src="./video/derived-rawBleeBootstrap.mp4"></video>
    \end{rawhtml}
    \end{presentationMode}

    \begin{articleMode}
    \begin{rawhtml}
    <!-- data-autoplay  controls -->
    <p>
    <video controls  preload="auto" src="./video/derived-rawBleeBootstrap.mp4" height="50%%" width="50%%">
    </video>
    </p>
     \end{rawhtml}
    \end{articleMode}
\end{frame}
\end{htmlonly}

\begin{verblatex}

\begin{frame}[fragile,plain,label=rawBleeBootstrap]
    \frameaudio{"audio/rawBleeBootstrap.mp3"}
    \frametitle{Bootstrapping:\\ From Fresh Debian to Raw-BISOS and Raw-Blee}
    \framesubtitle{}
%%BxPy: impressiveFrameParSet('rawBleeBootstrap', 'always', 'True')
%%BxPy: impressiveFrameParSet('rawBleeBootstrap', 'transition', 'UnSpecified')
%%%#+END:


    \begin{center}
      \begin{large}
        \textbf{Bootstrapping BISOS and Blee}
      \end{large}
  \end{center}

  \begin{center}
    \url{https://github.com/bxGenesis/start}
  \end{center}

\begin{center}
\vspace{0.2in}
\qrcode[height=0.5in]{https://github.com/bxGenesis/start}
\end{center}

\begin{latexonly}
  \begin{alertblock}{Work In Progress! --- Pardon Our Mess}
    Not Ready For General Release/Use. Don't Expect Stability/Support.
  \end{alertblock}
\end{latexonly}

\pnote{

  Let's bootstrap Raw-BISOS and Raw-Blee.

  Starting from scratch, get yourself a fresh copy of Debian 12.

  Then go to https://github.com/bxGenesis/start

  The README.org file of that github repo is same as Chapter 18,
  ``Engineering Adoption of BISOS and ByStar'' of the book.

  We will next run ``raw-bisos.sh'', but prior to that let's take a quick look.
  This bootstrap scripts will do a lot as root on your Fresh-Debian. It is best to
  first try it on a disposable VM.
  raw-bisos.sh adds the current debian user to sudoers.
  Then it installs pipx. And then with pipx it installs from PyPI bisos.provision.
  bisos.provision includes additional bash scripts that are then executed.
  Full installation involves setting up various accounts, groups, various directory hierarchies,
  lots of apt packages and lots of python packages from the bisos namespace.

  If you are ready, copy and paste this line and run it.
  You will be prompted for the root password.
  Then be patient. Full installation can take 15 minutes or so.
  The logs of this script are also captured in ~/raw-bisos-${dateTag}-log.org

}

\end{frame}
\end{verblatex}



%%%#+BEGIN: bx:dblock:lcnt:latex-section :mode "auto" :seg-title "Context for Unified Source Walkthrough"
%%% Args: :class "book|pres+art" :langs "en+fa" :disabledP "false" :seg-title "str" :short-title "str" :label "auto"
\begin{whenOrg}
*  _[[elisp:(blee:menu-sel:outline:popupMenu)][±]]_ _[[elisp:(blee:menu-sel:navigation:popupMenu)][Ξ]]_ [[elisp:(outline-show-branches+toggle)][|=]] [[elisp:(bx:orgm:indirectBufOther)][|>]] *[[elisp:(blee:ppmm:org-mode-toggle)][|N]]*  Section    [[elisp:(outline-show-subtree+toggle)][||]]   /Context for Unified Source Walkthrough/ ::  [[elisp:(org-cycle)][| ]]
\end{whenOrg}

\section{Context for Unified Source Walkthrough}
%%%#+END:


%%%#+BEGIN: b:lcnt:pres:frame/derivedImage :reveal "plain" :beamer "plain" :mode "auto" :title "Context for Unified Source Walkthrough" :subtitle "" :label "figmmDocProductionContext"
\begin{whenOrg}
*****  _[[elisp:(blee:menu-sel:outline:popupMenu)][±]]_ _[[elisp:(blee:menu-sel:navigation:popupMenu)][Ξ]]_ [[elisp:(outline-show-branches+toggle)][|=]] [[elisp:(bx:orgm:indirectBufOther)][|>]] *[[elisp:(blee:ppmm:org-mode-toggle)][|N]]*  derivedImage [[elisp:(outline-show-subtree+toggle)][||]] Label=figmmDocProductionContext Context for Unified Source Walkthrough
\end{whenOrg}

\begin{htmlonly}

\begin{frame}[fragile,plain,label=figmmDocProductionContext]
    \frameaudio{"audio/figmmDocProductionContext.mp3"}
    \frametitle{}
    \framesubtitle{}
    \begin{rawhtml}
<div class="center">
<img src="./disposition.gened/figmmDocProductionContext/slide-1.png" height="500">
</div>
    \end{rawhtml}
\end{frame}
\end{htmlonly}

\begin{verblatex}

\begin{frame}[fragile,plain,label=figmmDocProductionContext]
    \frameaudio{"audio/figmmDocProductionContext.mp3"}
    \frametitle{Context for Unified Source Walkthrough}
    \framesubtitle{}
%%BxPy: impressiveFrameParSet('figmmDocProductionContext', 'always', 'True')
%%BxPy: impressiveFrameParSet('figmmDocProductionContext', 'transition', 'UnSpecified')
%%%#+END:

%%%#+BEGIN: bx:dblock:lcnt:body:odg-artpres  :fig-file "./figures/bxMmDocProc.odg"
\begin{comment}
******  [[elisp:(org-cycle)][| ]]  [[elisp:(blee:ppmm:org-mode-toggle)][Nat]] [[elisp:(beginning-of-buffer)][Top]] [[elisp:(delete-other-windows)][(1)]] || /Figure/ =ODG-ArtPres=  *bxMmDocProc* -- ByStar Multimedia Document Authorship And Generation ::  [[elisp:(org-cycle)][| ]]
\end{comment}

\begin{presentationMode}

\begin{latexonly}
  \begin{figure}
    \begin{center}
       \includegraphics[width=108mm,keepaspectratio]{./figures/bxMmDocProc}
    \end{center}
  \end{figure}
\end{latexonly}

\begin{htmlonly}
  \begin{rawhtml}
<div class="center">
<img src="./figures/bxMmDocProc.png" height="450">
</div>
  \end{rawhtml}
\end{htmlonly}

\end{presentationMode}


\begin{articleMode}

\begin{latexonly}
  \begin{figure}[H]
    \begin{center}
      \includegraphics[width=\textwidth]{./figures/bxMmDocProc}
      \caption{ByStar Multimedia Document Authorship And Generation}
      \label{fig:bxMmDocProc}
    \end{center}
  \end{figure}
\end{latexonly}

\begin{htmlonly}
  %BEGIN IMAGE
  \begin{center}
      \includegraphics[width=\textwidth]{./figures/bxMmDocProc}
  \end{center}
  %END IMAGE
  %HEVEA\imageflush

  \begin{figure}
      \caption{ByStar Multimedia Document Authorship And Generation}
      \label{fig:bxMmDocProc}
  \end{figure}
\end{htmlonly}

\end{articleMode}

%%%#+END:

\pnote{

  Now that we have Raw-BISOS and Raw-Blee installed, we are ready to walkthrough
  the unified source of the very presentation that you are watching.

  The ``bodyPresArtEnFa.tex'' file that we will visit is in COMEEGA-LaTeX syntax with lots of
  org-mode dblocks which generate Beamer-LaTeX frames and conditioned LaTeX bodies.

  After the walkthrough I'll describe dblocks and COMEEGA in more detail.

  At the tail end of the walkthrough, we will also go through the generation process
  which runs XeLaTeX and HeVeA and a lot more.

}


\end{frame}
\end{verblatex}

%%%#+BEGIN: bx:dblock:lcnt:latex-section :mode "auto" :seg-title "Unified Source Walkthrough Screen Capture"
%%% Args: :class "book|pres+art" :langs "en+fa" :disabledP "false" :seg-title "str" :short-title "str" :label "auto"
\begin{whenOrg}
*  _[[elisp:(blee:menu-sel:outline:popupMenu)][±]]_ _[[elisp:(blee:menu-sel:navigation:popupMenu)][Ξ]]_ [[elisp:(outline-show-branches+toggle)][|=]] [[elisp:(bx:orgm:indirectBufOther)][|>]] *[[elisp:(blee:ppmm:org-mode-toggle)][|N]]*  Section    [[elisp:(outline-show-subtree+toggle)][||]]   /A Quick Tour of BISOS and Blee/ ::  [[elisp:(org-cycle)][| ]]
\end{whenOrg}

\section{A Quick Tour of BISOS and Blee}
%%%#+END:


%%%#+BEGIN: b:lcnt:pres:frame/derivedVideo :title "" :subtitle "" :reveal "plain" :beamer "plain" :label "unifiedSourceScreenCapture" :comment "body=derivedInsert"
\begin{whenOrg}
*****  _[[elisp:(blee:menu-sel:outline:popupMenu)][±]]_ _[[elisp:(blee:menu-sel:navigation:popupMenu)][Ξ]]_ [[elisp:(outline-show-branches+toggle)][|=]] [[elisp:(bx:orgm:indirectBufOther)][|>]] *[[elisp:(blee:ppmm:org-mode-toggle)][|N]]*  derivedVideo [[elisp:(outline-show-subtree+toggle)][||]] Label=unifiedSourceScreenCapture  body=derivedInsert
\end{whenOrg}

\begin{htmlonly}

\begin{frame}[fragile,plain,label=unifiedSourceScreenCapture]
    \frametitle{}
    \framesubtitle{}
    \begin{presentationMode}
    \begin{rawhtml}
<video preload="auto" data-audio-controls src="./video/derived-unifiedSourceScreenCapture.mp4"></video>
    \end{rawhtml}
    \end{presentationMode}

    \begin{articleMode}
    \begin{rawhtml}
    <!-- data-autoplay  controls -->
    <p>
    <video controls  preload="auto" src="./video/derived-unifiedSourceScreenCapture.mp4" height="50%%" width="50%%">
    </video>
    </p>
     \end{rawhtml}
    \end{articleMode}
\end{frame}
\end{htmlonly}

\begin{verblatex}

\begin{frame}[fragile,plain,label=unifiedSourceScreenCapture]
    \frameaudio{"audio/unifiedSourceScreenCapture.mp3"}
    \frametitle{}
    \framesubtitle{}
%%BxPy: impressiveFrameParSet('unifiedSourceScreenCapture', 'always', 'True')
%%BxPy: impressiveFrameParSet('unifiedSourceScreenCapture', 'transition', 'UnSpecified')
%%%#+END:

    Video File: ./video/derived-unifiedSourceScreenCapture.mp4

    Video of the Unified Source Walkthrough



\pnote{

Let's look at our input file. It's a LaTeX file in LaTeX mode and it has org syntax org-mode
included in it and I can toggle between LaTeX and org-mode so, now I'm gonna be in org-mode
and org-mode gives me everything that org has to offer including a very convenient navigation
framework. So let's take one slide and take a look at how it was done. So I would come to this
scope slide and while I am there I'm gonna click on N. N takes me to the native LaTeX form
back so that I'll be looking at it not in org but in LaTeX. So we're back in LaTeX and as
you can see it uses a dynamic block starting with the comments and the BEGIN and it uses
the dynamic a dynamic block named a framedDrive image which means the content of this frame
will be dispensed as an image not as text and it also automatically creates for me a name,
a label, that can be used for voiceover augmentation. So a file in the audio directory called
ScopeOfBleeLcnt.mp3 is this audio that will come on top of this slide and then the rest is the LaTeX itself.


}

\end{frame}
\end{verblatex}


%%%#+BEGIN: bx:dblock:lcnt:latex-section :mode "auto" :seg-title "Blee Org Dynamic Blocks --- Everywhere"
%%% Args: :class "book|pres+art" :langs "en+fa" :disabledP "false" :seg-title "str" :short-title "str" :label "auto"
\begin{whenOrg}
*  _[[elisp:(blee:menu-sel:outline:popupMenu)][±]]_ _[[elisp:(blee:menu-sel:navigation:popupMenu)][Ξ]]_ [[elisp:(outline-show-branches+toggle)][|=]] [[elisp:(bx:orgm:indirectBufOther)][|>]] *[[elisp:(blee:ppmm:org-mode-toggle)][|N]]*  Section    [[elisp:(outline-show-subtree+toggle)][||]]   /Blee Org Dynamic Blocks --- Everywhere/ ::  [[elisp:(org-cycle)][| ]]
\end{whenOrg}

\section{Blee Org Dynamic Blocks --- Everywhere}
%%%#+END:


%%%#+BEGIN: b:lcnt:pres:frame/derivedImage :title "Blee Org Dynamic Blocks --- Everywhere" :subtitle "" :label "dblocksEverywhere" :comment "body=itemize" :reveal "plain" :beamer "plain"
\begin{whenOrg}
*****  _[[elisp:(blee:menu-sel:outline:popupMenu)][±]]_ _[[elisp:(blee:menu-sel:navigation:popupMenu)][Ξ]]_ [[elisp:(outline-show-branches+toggle)][|=]] [[elisp:(bx:orgm:indirectBufOther)][|>]] *[[elisp:(blee:ppmm:org-mode-toggle)][|N]]*  derivedImage [[elisp:(outline-show-subtree+toggle)][||]] Label=dblocksEverywhere Blee Org Dynamic Blocks --- Everywhere body=itemize
\end{whenOrg}

\begin{htmlonly}

\begin{frame}[fragile,plain,label=dblocksEverywhere]
    \frameaudio{"audio/dblocksEverywhere.mp3"}
    \frametitle{}
    \framesubtitle{}
    \begin{rawhtml}
<div class="center">
<img src="./disposition.gened/dblocksEverywhere/slide-1.png" height="500">
</div>
    \end{rawhtml}
\end{frame}
\end{htmlonly}

\begin{verblatex}

\begin{frame}[fragile,plain,label=dblocksEverywhere]
    \frameaudio{"audio/dblocksEverywhere.mp3"}
    \frametitle{Blee Org Dynamic Blocks --- Everywhere}
    \framesubtitle{}
%%BxPy: impressiveFrameParSet('dblocksEverywhere', 'always', 'True')
%%BxPy: impressiveFrameParSet('dblocksEverywhere', 'transition', 'UnSpecified')
%%%#+END:


    From \url{https://orgmode.org/manual/Dynamic-Blocks.html}:

    \begin{exampleblock}{A.6 Dynamic Blocks}
      Org supports dynamic blocks in Org documents. They are ...  like any other code block, but the contents are updated automatically by a
      user function.
    \end{exampleblock}

    The concept of updating a block of text in any mode  is very powerful.
    This  should not be limited to just Org documents.

    Blee supports Dynamic Blocks in emacs-lisp-mode, latex-mode, sh-script-mode,
    python-mode, etc.

    \bigskip

    Dynamic Blocks are more useful when they are considered universal to all of emacs
    modes. In Emacs they should be. In Blee, they are. In raw-BISOS, see:  \url{file:/bisos/blee/env3/dblocks},
    \url{https://github.com/bx-blee/poly-dblock} -- \url{https://github.com/bx-blee/blee-dblocks}

%     Like so, for latex-mode
%     \begin{verbatim}
%   (add-hook 'latex-mode-hook
% 	    '(lambda ()
% 	       (setq org-dblock-start-re
% 		     "^[ 	]*%%%#\\+BEGIN:[ 	]+\\(\\S-+\\)\\([ 	]+\\(.*\\)\\)?")
% 	       (setq org-dblock-end-re
% 		     "^[ 	]*%%%#\\+END\\([:
% \n]\\|$\\)")

%     \end{verbatim}

\pnote{

  The concept of ``Org Dynamic Blocks'' is very powerful.
  I think of them as universal visible macros.
  But, why should they be primarily used in just Org-Mode.
  I say let's generalize them to ``Emacs Dynamic Blocks''. Have defaults
  for org-dblock-start-re in every relevant mode and use them everywhere.

  Blee does that. In COMEEGA-LaTeX, Dynamic Blocks create Frame Controls and insert Image and Video contents.

}

\end{frame}
\end{verblatex}


%%%#+BEGIN: bx:dblock:lcnt:latex-section :mode "auto" :seg-title "COMEEGA"
%%% Args: :class "book|pres+art" :langs "en+fa" :disabledP "false" :seg-title "str" :short-title "str" :label "auto"
\begin{whenOrg}
*  _[[elisp:(blee:menu-sel:outline:popupMenu)][±]]_ _[[elisp:(blee:menu-sel:navigation:popupMenu)][Ξ]]_ [[elisp:(outline-show-branches+toggle)][|=]] [[elisp:(bx:orgm:indirectBufOther)][|>]] *[[elisp:(blee:ppmm:org-mode-toggle)][|N]]*  Section    [[elisp:(outline-show-subtree+toggle)][||]]   /COMEEGA/ ::  [[elisp:(org-cycle)][| ]]
\end{whenOrg}

\section{COMEEGA}
%%%#+END:


%%%#+BEGIN: b:lcnt:pres:frame/derivedImage :title "COMEEGA" :subtitle "Collaborative Org-Mode Enhanced Emacs Generalized Authorship" :label "comeega" :comment "body=itemize" :reveal "plain" :beamer "plain"
\begin{whenOrg}
*****  _[[elisp:(blee:menu-sel:outline:popupMenu)][±]]_ _[[elisp:(blee:menu-sel:navigation:popupMenu)][Ξ]]_ [[elisp:(outline-show-branches+toggle)][|=]] [[elisp:(bx:orgm:indirectBufOther)][|>]] *[[elisp:(blee:ppmm:org-mode-toggle)][|N]]*  derivedImage [[elisp:(outline-show-subtree+toggle)][||]] Label=comeega COMEEGA body=itemize
\end{whenOrg}

\begin{htmlonly}

\begin{frame}[fragile,plain,label=comeega]
    \frameaudio{"audio/comeega.mp3"}
    \frametitle{}
    \framesubtitle{}
    \begin{rawhtml}
<div class="center">
<img src="./disposition.gened/comeega/slide-1.png" height="500">
</div>
    \end{rawhtml}
\end{frame}
\end{htmlonly}

\begin{verblatex}

\begin{frame}[fragile,plain,label=comeega]
    \frameaudio{"audio/comeega.mp3"}
    \frametitle{COMEEGA}
    \framesubtitle{Collaborative Org-Mode Enhanced Emacs Generalized Authorship}
%%BxPy: impressiveFrameParSet('comeega', 'always', 'True')
%%BxPy: impressiveFrameParSet('comeega', 'transition', 'UnSpecified')
%%%#+END:

    \begin{itemize}

      \item Consider\\ \colorbox{blue!30}{\textit{Surrounded Programming}} as \textbf{inverse of} \colorbox{blue!60}{\color{white}\textit{Literate Programming}}.


      \item Consider \colorbox{blue!30}{\textit{COMEEGA}} as \textbf{inverse of} org-mode based \colorbox{blue!60}{\color{white}\textit{org-babel}}.

      \item In \colorbox{blue!60}{\color{white}\textit{org-babel}} (org-mode based literate programming), we include various programming
   fragments inside of an org-file. In bable, org recognizes authorship-major-modes and
   supports them.

      \item In \colorbox{blue!30}{\textit{COMEEGA}}, we include various org-mode fragments
            inside of a programming language’s major mode as comments or as doc-strings in
            org format. The combination is then supported through polymode.

    \end{itemize}

    See: \url{https://github.com/bx-blee/comeega}
    

\pnote{

  Much of Blee and BISOS are implemented in COMEEGA. Almost all of our Elisp, Python, Bash
  and LaTeX work uses COMEEGA.

  COMEEGA stands for Collaborative Org-Mode Enhanced Emacs Generalized Authorship. It
  is the inverse of org-babel. COMEEGA adds org-mode to your programming mode.

  Full and proper use of COMEEGA, requires Polymode. Let's call that Poly-COMEEGA.

  %  The concept of Polymode is that of a framework for multiple major modes (MMM) inside a
  % single Emacs buffer.

  But Emacs's Polymode is work-in-progress, particularly now with the
  new tree-sitter. So, in the interim, my usage of COMEEGA has been in the form of
  Toggle-COMEEGA. Where I manually switch between the programming-mode and org-mode. For
  me this has proved to be a fine interim solution.
}

\end{frame}
\end{verblatex}


%%%#+BEGIN: bx:dblock:lcnt:latex-part :disabledP "false" :seg-title "Internationalization --- A Non-Americanist Perspective"
%%% Args: :toc "NU" :tocDepth 3 :part "NU" :label "auto|spec" :partpage t
\begin{whenOrg}
*      ================
*  [[elisp:(blee:ppmm:org-mode-toggle)][|n]] [[elisp:(blee:menu-sel:outline:popupMenu)][+-]] [[elisp:(blee:menu-sel:navigation:popupMenu)][==]]  *Part*   _Internationalization --- A Non-Americanist Perspective_ ::  [[elisp:(org-cycle)][| ]]
\end{whenOrg}

\newpage
\part{Internationalization --- A Non-Americanist Perspective}
%%%#+END:

%%%#+BEGIN: b:lcnt:pres:frame/derivedImage :title "Internationalization --- A Non-Americanist Perspective" :subtitle "" :label "part_internationalization" :comment "body=text" :reveal "plain" :beamer ""
\begin{whenOrg}
*****  _[[elisp:(blee:menu-sel:outline:popupMenu)][±]]_ _[[elisp:(blee:menu-sel:navigation:popupMenu)][Ξ]]_ [[elisp:(outline-show-branches+toggle)][|=]] [[elisp:(bx:orgm:indirectBufOther)][|>]] *[[elisp:(blee:ppmm:org-mode-toggle)][|N]]*  derivedImage [[elisp:(outline-show-subtree+toggle)][||]] Label=part_internationalization Internationalization --- A Non-Americanist Perspective body=text
\end{whenOrg}

\begin{htmlonly}

\begin{frame}[fragile,plain,label=part_internationalization]
    \frameaudio{"audio/part_internationalization.mp3"}
    \frametitle{}
    \framesubtitle{}
    \begin{rawhtml}
<div class="center">
<img src="./disposition.gened/part_internationalization/slide-1.png" height="500">
</div>
    \end{rawhtml}
\end{frame}
\end{htmlonly}

\begin{verblatex}

\begin{frame}[fragile,label=part_internationalization]
    \frameaudio{"audio/part_internationalization.mp3"}
    \frametitle{Internationalization --- A Non-Americanist Perspective}
    \framesubtitle{}
%%BxPy: impressiveFrameParSet('part_internationalization', 'always', 'True')
%%BxPy: impressiveFrameParSet('part_internationalization', 'transition', 'UnSpecified')
%%%#+END:

    \tableofcontents

\pnote{

  Naturally, content processing should be multi-lingual and internationalized.

  Let's look at that dimension.

  }
\end{frame}
\end{verblatex}



%%%#+BEGIN: bx:dblock:lcnt:latex-section :mode "auto" :seg-title "Perso-Arabic Input Methods, bidi and shaping --- Fully Integrated"
%%% Args: :class "book|pres+art" :langs "en+fa" :disabledP "false" :seg-title "str" :short-title "str" :label "auto"
\begin{whenOrg}
*  _[[elisp:(blee:menu-sel:outline:popupMenu)][±]]_ _[[elisp:(blee:menu-sel:navigation:popupMenu)][Ξ]]_ [[elisp:(outline-show-branches+toggle)][|=]] [[elisp:(bx:orgm:indirectBufOther)][|>]] *[[elisp:(blee:ppmm:org-mode-toggle)][|N]]*  Section    [[elisp:(outline-show-subtree+toggle)][||]]   /Perso-Arabic Input Methods, bidi and shaping --- Fully Integrated/ ::  [[elisp:(org-cycle)][| ]]
\end{whenOrg}

\section{Perso-Arabic Input Methods, bidi and shaping --- Fully Integrated}
%%%#+END:


%%%#+BEGIN: b:lcnt:pres:frame/derivedImage :title "Perso-Arabic Input Methods, bidi, shaping - Fully Integrated" :subtitle "In Emacs, In XeLaTeX, In HaVeA and In Blee" :label "persoArabic" :beamer "plain" :comment "body=text"
\begin{whenOrg}
*****  _[[elisp:(blee:menu-sel:outline:popupMenu)][±]]_ _[[elisp:(blee:menu-sel:navigation:popupMenu)][Ξ]]_ [[elisp:(outline-show-branches+toggle)][|=]] [[elisp:(bx:orgm:indirectBufOther)][|>]] *[[elisp:(blee:ppmm:org-mode-toggle)][|N]]*  derivedImage [[elisp:(outline-show-subtree+toggle)][||]] Label=persoArabic Perso-Arabic Input Methods, bidi, shaping - Fully Integrated body=text
\end{whenOrg}

\begin{htmlonly}

\begin{frame}[fragile,plain,label=persoArabic]
    \frameaudio{"audio/persoArabic.mp3"}
    \frametitle{}
    \framesubtitle{}
    \begin{rawhtml}
<div class="center">
<img src="./disposition.gened/persoArabic/slide-1.png" height="500">
</div>
    \end{rawhtml}
\end{frame}
\end{htmlonly}

\begin{verblatex}

\begin{frame}[fragile,plain,label=persoArabic]
    \frameaudio{"audio/persoArabic.mp3"}
    \frametitle{Perso-Arabic Input Methods, bidi, shaping - Fully Integrated}
    \framesubtitle{In Emacs, In XeLaTeX, In HaVeA and In Blee}
%%BxPy: impressiveFrameParSet('persoArabic', 'always', 'True')
%%BxPy: impressiveFrameParSet('persoArabic', 'transition', 'UnSpecified')
%%%#+END:

    \begin{description}
      \item[Emacs:]   \hfill \\
        Very multilingual, good bidi support\\
        EmacsConf 2021 -- Perso-Arabic Input Methods And Making More Emacs Apps BIDI Aware -- \url{https://emacsconf.org/2021/talks/perso-arabic}
      \item[XeLaTeX:]   \hfill \\
        Multilingual, good bidi support\\
        usepackage{bystarpersian}  -- Defines: newfontfamily{persian}, newcommand{farsi}, newenvironment{faPar},{fa}
      \item[HeVeA]   \hfill \\
        Needs usepackage{bystarpersian}
      \item[Reveal.js]   \hfill \\
        Multilingual with bidi support\\
    \end{description}


\pnote{

  I am Iranian and much of what I write is in Farsi. Getting Perso-Arabic text right is
  often a challenge, as it involves Bi-Directional text (BIDI) and shaping of characters.

  In the context of our content generation these need to span all relevant tools, not just emacs.
  For emacs, I have created my own input method called farsi-transliterate-banan. My EmacsConf 2021 talk was about that.

  Now let's look at some examples and spice it up a bit with semantics.

}

\end{frame}
\end{verblatex}


%%%#+BEGIN: bx:dblock:lcnt:latex-section :mode "auto" :seg-title "Ayatollah Khomeini: The so-called Western Intelectual Property is Utterly Invalid"
%%% Args: :class "book|pres+art" :langs "en+fa" :disabledP "false" :seg-title "str" :short-title "str" :label "auto"
\begin{whenOrg}
*  _[[elisp:(blee:menu-sel:outline:popupMenu)][±]]_ _[[elisp:(blee:menu-sel:navigation:popupMenu)][Ξ]]_ [[elisp:(outline-show-branches+toggle)][|=]] [[elisp:(bx:orgm:indirectBufOther)][|>]] *[[elisp:(blee:ppmm:org-mode-toggle)][|N]]*  Section    [[elisp:(outline-show-subtree+toggle)][||]]   /Ayatollah Khomeini: The so-called Western Intelectual Property is Utterly Invalid/ ::  [[elisp:(org-cycle)][| ]]
\end{whenOrg}

\section{Ayatollah Khomeini: The so-called Western Intelectual Property is Utterly Invalid}
%%%#+END:


%%%#+BEGIN: b:lcnt:pres:frame/derivedImage :title "Ayatollah Khomeini:\\\\ The so-called Western Intelectual Property is Utterly Invalid" :subtitle "" :label "ayatollahKhomeini" :beamer "plain" :comment "body=text"
\begin{whenOrg}
*****  _[[elisp:(blee:menu-sel:outline:popupMenu)][±]]_ _[[elisp:(blee:menu-sel:navigation:popupMenu)][Ξ]]_ [[elisp:(outline-show-branches+toggle)][|=]] [[elisp:(bx:orgm:indirectBufOther)][|>]] *[[elisp:(blee:ppmm:org-mode-toggle)][|N]]*  derivedImage [[elisp:(outline-show-subtree+toggle)][||]] Label=ayatollahKhomeini Ayatollah Khomeini:\\ The so-called Western Intelectual Property is Utterly Invalid body=text
\end{whenOrg}

\begin{htmlonly}

\begin{frame}[fragile,plain,label=ayatollahKhomeini]
    \frameaudio{"audio/ayatollahKhomeini.mp3"}
    \frametitle{}
    \framesubtitle{}
    \begin{rawhtml}
<div class="center">
<img src="./disposition.gened/ayatollahKhomeini/slide-1.png" height="500">
</div>
    \end{rawhtml}
\end{frame}
\end{htmlonly}


\begin{verblatex}


\begin{frame}[fragile,plain,label=ayatollahKhomeini]
    \frameaudio{"audio/ayatollahKhomeini.mp3"}
    \frametitle{Ayatollah Khomeini:\\ The so-called Western Intelectual Property is Utterly Invalid}
    \framesubtitle{}
%%BxPy: impressiveFrameParSet('ayatollahKhomeini', 'always', 'True')
%%BxPy: impressiveFrameParSet('ayatollahKhomeini', 'transition', 'UnSpecified')
%%%#+END:


% \begin{adjustwidth}{+0.15in}{+0.15in}
% \columnratio{0.6}
% \begin{paracol}{2}

\RTL
  \begin{faPar}


آنچه كه معروف به حق طبع نزد افراد است حق شرعی به شمار نمی‌آید و زایل نمودن سلطه مردم بر
اموالشان بدون اینكه شرط و عقدی در بین باشد جایز نیست و مجرد نوشتن جمله «حق چاپ و تقلید
محفوظ است» حقی به وجود نمی‌آورد و التزام دیگران را به دنبال ندارد.

  \end{faPar}
\LTR

  \bigskip

  % \switchcolumn

  That which has become famous among people as the
  right of authorship (copyright) is not a theological or an
  ethical right.
  And to take away the control of people on their
  belongings without explicit condition and contract is not
  permitted. And just by writing a sentence saying ``Copyright: Print and Copying Rights are Reserved''
  creates no right and does not require conformance of others.

%\end{paracol}
% \end{quote}
%\end{adjustwidth}
% \columnratio{0.5}


\pnote{

  As and example of proper BIDI text, here is the orignal Farsi text along with English translation of
  Imam Khomeini's text with respect to invalidity of Western Inteletual Proprty Rights
  regime.

}

\end{frame}
\end{verblatex}


%%%#+BEGIN: bx:dblock:lcnt:latex-section :mode "auto" :seg-title "Ayatollah Motahari: Intellectual Property Is Public Property"
%%% Args: :class "book|pres+art" :langs "en+fa" :disabledP "false" :seg-title "str" :short-title "str" :label "auto"
\begin{whenOrg}
*  _[[elisp:(blee:menu-sel:outline:popupMenu)][±]]_ _[[elisp:(blee:menu-sel:navigation:popupMenu)][Ξ]]_ [[elisp:(outline-show-branches+toggle)][|=]] [[elisp:(bx:orgm:indirectBufOther)][|>]] *[[elisp:(blee:ppmm:org-mode-toggle)][|N]]*  Section    [[elisp:(outline-show-subtree+toggle)][||]]   /Ayatollah Motahari: Intellectual Property Is Public Property/ ::  [[elisp:(org-cycle)][| ]]
\end{whenOrg}

\section{Ayatollah Motahari: Intellectual Property Is Public Property}
%%%#+END:


%%%#+BEGIN: b:lcnt:pres:frame/derivedImage :title "Ayatollah Motahari: Intellectual Property Is Public Property" :subtitle "" :label "ayatollahMotahari" :beamer "plain" :comment "body=text"
\begin{whenOrg}
*****  _[[elisp:(blee:menu-sel:outline:popupMenu)][±]]_ _[[elisp:(blee:menu-sel:navigation:popupMenu)][Ξ]]_ [[elisp:(outline-show-branches+toggle)][|=]] [[elisp:(bx:orgm:indirectBufOther)][|>]] *[[elisp:(blee:ppmm:org-mode-toggle)][|N]]*  derivedImage [[elisp:(outline-show-subtree+toggle)][||]] Label=ayatollahMotahari Ayatollah Motahari: Intellectual Property Is Public Property body=text
\end{whenOrg}

\begin{htmlonly}

\begin{frame}[fragile,plain,label=ayatollahMotahari]
    \frameaudio{"audio/ayatollahMotahari.mp3"}
    \frametitle{}
    \framesubtitle{}
    \begin{rawhtml}
<div class="center">
<img src="./disposition.gened/ayatollahMotahari/slide-1.png" height="500">
</div>
    \end{rawhtml}
\end{frame}
\end{htmlonly}

\begin{verblatex}

\begin{frame}[fragile,plain,label=ayatollahMotahari]
    \frameaudio{"audio/ayatollahMotahari.mp3"}
    \frametitle{Ayatollah Motahari: Intellectual Property Is Public Property}
    \framesubtitle{}
%%BxPy: impressiveFrameParSet('ayatollahMotahari', 'always', 'True')
%%BxPy: impressiveFrameParSet('ayatollahMotahari', 'transition', 'UnSpecified')
%%%#+END:


% \begin{adjustwidth}{+0.15in}{+0.15in}
% \begin{quote}
%   \begin{paracol}{2}

\RTL
  \begin{faPar}

    اختراعْ محصول مستقیم نبوغ فکری فرد نیست تا او مالک اثر یا
    اختراع خویش باشد بلکه طبیعت یا اجتماع به دلیل شرایط و
    زمینه‌های خاص در پیدایش چنین اختراعی مؤثر بوده‌اند. ازاین‌رو،
    اموال فکری مانند اختراع و تألیف سرمایه‌ای عمومی و مملوکی
    اشتراکی است نه ملک خصوصی.


  \end{faPar}
\LTR

\bigskip

    Invention is not solely the direct product of an individual's
    intellectual genius, for nature and society, due to specific
    conditions and circumstances, have played a significant role
    in the emergence of such inventions. Therefore, intellectual
    properties such as inventions and creations are considered a
    common and collective asset rather than private ownership.

  % \switchcolumn

% \end{paracol}
% \end{quote}
% \columnratio{0.5}
% \end{adjustwidth}


\pnote{

  And as another example of proper BIDI text, here is Ayatollah Mothari's take on Western
  IPR not being private property.

  Note that these predate by more than half a century Jack Dorsey and Elon Musk's tweets
  of April 11, 2025 saying ``Delete all IP law''.

  This topic is too important and too sensitive to be left to American Billionaires and their tweets.

  Let me again refer you to the logic of polyexistentials in my book.

}
\end{frame}
\end{verblatex}



%%%#+BEGIN: bx:dblock:lcnt:latex-section :mode "auto" :seg-title "Nature of Polyexistentials Citations: Ethical and Religious Cures"
%%% Args: :class "book|pres+art" :langs "en+fa" :disabledP "false" :seg-title "str" :short-title "str" :label "auto"
\begin{whenOrg}
*  _[[elisp:(blee:menu-sel:outline:popupMenu)][±]]_ _[[elisp:(blee:menu-sel:navigation:popupMenu)][Ξ]]_ [[elisp:(outline-show-branches+toggle)][|=]] [[elisp:(bx:orgm:indirectBufOther)][|>]] *[[elisp:(blee:ppmm:org-mode-toggle)][|N]]*  Section    [[elisp:(outline-show-subtree+toggle)][||]]   /Nature of Polyexistentials Citations: Ethical and Religious Cures/ ::  [[elisp:(org-cycle)][| ]]
\end{whenOrg}

\section{Nature of Polyexistentials Citations: Ethical and Religious Cures}
%%%#+END:


%%%#+BEGIN: b:lcnt:pres:frame/derivedImage :title "Nature of Polyexistentials Citations:\\\\ Ethical and Religious Cures" :subtitle "" :label "citation_ethicalCures" :beamer "plain" :comment "body=text"
\begin{whenOrg}
*****  _[[elisp:(blee:menu-sel:outline:popupMenu)][±]]_ _[[elisp:(blee:menu-sel:navigation:popupMenu)][Ξ]]_ [[elisp:(outline-show-branches+toggle)][|=]] [[elisp:(bx:orgm:indirectBufOther)][|>]] *[[elisp:(blee:ppmm:org-mode-toggle)][|N]]*  derivedImage [[elisp:(outline-show-subtree+toggle)][||]] Label=citation_ethicalCures Nature of Polyexistentials Citations:\\ Ethical and Religious Cures body=text
\end{whenOrg}

\begin{htmlonly}

\begin{frame}[fragile,plain,label=citation_ethicalCures]
    \frameaudio{"audio/citation_ethicalCures.mp3"}
    \frametitle{}
    \framesubtitle{}
    \begin{rawhtml}
<div class="center">
<img src="./disposition.gened/citation_ethicalCures/slide-1.png" height="500">
</div>
    \end{rawhtml}
\end{frame}
\end{htmlonly}

\begin{verblatex}

\begin{frame}[fragile,plain,label=citation_ethicalCures]
    \frameaudio{"audio/citation_ethicalCures.mp3"}
    \frametitle{Nature of Polyexistentials Citations:\\ Ethical and Religious Cures}
    \framesubtitle{}
%%BxPy: impressiveFrameParSet('citation_ethicalCures', 'always', 'True')
%%BxPy: impressiveFrameParSet('citation_ethicalCures', 'transition', 'UnSpecified')
%%%#+END:


    \begin{center}
      \begin{large}
        \textbf{Chapter 14: Ethical and Religious Cures}
      \end{large}
  \end{center}

    \bigskip

  \begin{large}
Section 14.2.1.3.1:\\[10pt] Imam Khomeini's Decree Invalidates So-Called Intellectual Property Rights
  \end{large}

    \bigskip

  \begin{large}
Section 14.2.1.3.2:\\[10pt] Ayatollah Motahari Views Intellectual Property as Public Property
  \end{large}



\pnote{

  Chapter 14 of the book is dedicated to Ethics and ownership in Religions.

}
\end{frame}
\end{verblatex}



%%%#+BEGIN: bx:dblock:lcnt:latex-section :mode "auto" :seg-title "Nature of Polyexistentials Citations: Non Americanist Perspectives"
%%% Args: :class "book|pres+art" :langs "en+fa" :disabledP "false" :seg-title "str" :short-title "str" :label "auto"
\begin{whenOrg}
*  _[[elisp:(blee:menu-sel:outline:popupMenu)][±]]_ _[[elisp:(blee:menu-sel:navigation:popupMenu)][Ξ]]_ [[elisp:(outline-show-branches+toggle)][|=]] [[elisp:(bx:orgm:indirectBufOther)][|>]] *[[elisp:(blee:ppmm:org-mode-toggle)][|N]]*  Section    [[elisp:(outline-show-subtree+toggle)][||]]   /Nature of Polyexistentials Citations: Non Americanist Perspectives/ ::  [[elisp:(org-cycle)][| ]]
\end{whenOrg}

\section{Nature of Polyexistentials Citations: Non Americanist Perspectives}
%%%#+END:


%%%#+BEGIN: b:lcnt:pres:frame/derivedImage :title "Nature of Polyexistentials Citations:\\\\ Non Americanist Perspectives" :subtitle "" :label "citation_nonAmericanist" :beamer "plain" :comment "body=text"
\begin{whenOrg}
*****  _[[elisp:(blee:menu-sel:outline:popupMenu)][±]]_ _[[elisp:(blee:menu-sel:navigation:popupMenu)][Ξ]]_ [[elisp:(outline-show-branches+toggle)][|=]] [[elisp:(bx:orgm:indirectBufOther)][|>]] *[[elisp:(blee:ppmm:org-mode-toggle)][|N]]*  derivedImage [[elisp:(outline-show-subtree+toggle)][||]] Label=citation_nonAmericanist Nature of Polyexistentials Citations:\\ Non Americanist Perspectives body=text
\end{whenOrg}

\begin{htmlonly}

\begin{frame}[fragile,plain,label=citation_nonAmericanist]
    \frameaudio{"audio/citation_nonAmericanist.mp3"}
    \frametitle{}
    \framesubtitle{}
    \begin{rawhtml}
<div class="center">
<img src="./disposition.gened/citation_nonAmericanist/slide-1.png" height="500">
</div>
    \end{rawhtml}
\end{frame}
\end{htmlonly}

\begin{verblatex}

\begin{frame}[fragile,plain,label=citation_nonAmericanist]
    \frameaudio{"audio/citation_nonAmericanist.mp3"}
    \frametitle{Nature of Polyexistentials Citations:\\ Non Americanist Perspectives}
    \framesubtitle{}
%%BxPy: impressiveFrameParSet('citation_nonAmericanist', 'always', 'True')
%%BxPy: impressiveFrameParSet('citation_nonAmericanist', 'transition', 'UnSpecified')
%%%#+END:


    \begin{center}
      \begin{large}
        \textbf{Chapter 6: Debunking the Myth of Western IPR Regim}
      \end{large}
  \end{center}

    \bigskip

  \begin{large}
Section 6.3:\\[10pt] Promoting Creativity and Innovation: IP is a Failed Experiment
  \end{large}

    \bigskip

   \begin{center}
      \begin{large}
        \textbf{Chapter 12: Digital Non-Proprietary Movements}
      \end{large}
  \end{center}

  \begin{large}
Section 12.4:\\[10pt] A Cynical Perspective on Freedom Orientation of Americans
  \end{large}

    \bigskip

    \begin{center}
      \begin{large}
        \textbf{Chapter 9: Americanism: Root of the IPR Mistake}
      \end{large}
  \end{center}



\pnote{

  With respect to my preference for Ethics over Freedom, let me refer you to Section 12.4
  ``A Cynical Perspective on Freedom Orientation of Americans'' in which I describe
  where the FOSS labels and the likes of Stallman, Raymond, Moglen and Lessig have gone wrong.

  If you are one of their followers, perhaps Chapter 12 is for you.

}
\end{frame}
\end{verblatex}


%%%#+BEGIN: bx:dblock:lcnt:latex-part :disabledP "false" :seg-title "Autonomous Self-Publication and Federated Re-Publications"
%%% Args: :toc "NU" :tocDepth 3 :part "NU" :label "auto|spec" :partpage t
\begin{whenOrg}
*      ================
*  [[elisp:(blee:ppmm:org-mode-toggle)][|n]] [[elisp:(blee:menu-sel:outline:popupMenu)][+-]] [[elisp:(blee:menu-sel:navigation:popupMenu)][==]]  *Part*   _Autonomous Self-Publication and Federated Re-Publications_ ::  [[elisp:(org-cycle)][| ]]
\end{whenOrg}

\newpage
\part{Autonomous Self-Publication and Federated Re-Publications}
%%%#+END:

%%%#+BEGIN: b:lcnt:pres:frame/derivedImage :title "Autonomous Self-Publication and Federated Re-Publications" :subtitle "" :label "part_publications" :comment "body=text" :reveal "plain" :beamer ""
\begin{whenOrg}
*****  _[[elisp:(blee:menu-sel:outline:popupMenu)][±]]_ _[[elisp:(blee:menu-sel:navigation:popupMenu)][Ξ]]_ [[elisp:(outline-show-branches+toggle)][|=]] [[elisp:(bx:orgm:indirectBufOther)][|>]] *[[elisp:(blee:ppmm:org-mode-toggle)][|N]]*  derivedImage [[elisp:(outline-show-subtree+toggle)][||]] Label=part_publications Autonomous Self-Publication and Federated Re-Publications body=text
\end{whenOrg}

\begin{htmlonly}

\begin{frame}[fragile,plain,label=part_publications]
    \frameaudio{"audio/part_publications.mp3"}
    \frametitle{}
    \framesubtitle{}
    \begin{rawhtml}
<div class="center">
<img src="./disposition.gened/part_publications/slide-1.png" height="500">
</div>
    \end{rawhtml}
\end{frame}
\end{htmlonly}

\begin{verblatex}

\begin{frame}[fragile,label=part_publications]
    \frameaudio{"audio/part_publications.mp3"}
    \frametitle{Autonomous Self-Publication and Federated Re-Publications}
    \framesubtitle{}
%%BxPy: impressiveFrameParSet('part_publications', 'always', 'True')
%%BxPy: impressiveFrameParSet('part_publications', 'transition', 'UnSpecified')
%%%#+END:

    \tableofcontents

\pnote{

  My emphasis thus far has been on content generation. Let's very briefly also look at
  Autonomous Self-Publication and Federated Re-Publications of our content.

}

\end{frame}
\end{verblatex}


%%%#+BEGIN: bx:dblock:lcnt:latex-section :mode "auto" :seg-title "Just The Universal Debian Everywhere"
%%% Args: :class "book|pres+art" :langs "en+fa" :disabledP "false" :seg-title "str" :short-title "str" :label "auto"
\begin{whenOrg}
*  _[[elisp:(blee:menu-sel:outline:popupMenu)][±]]_ _[[elisp:(blee:menu-sel:navigation:popupMenu)][Ξ]]_ [[elisp:(outline-show-branches+toggle)][|=]] [[elisp:(bx:orgm:indirectBufOther)][|>]] *[[elisp:(blee:ppmm:org-mode-toggle)][|N]]*  Section    [[elisp:(outline-show-subtree+toggle)][||]]   /Just The Universal Debian Everywhere/ ::  [[elisp:(org-cycle)][| ]]
\end{whenOrg}

\section{Just The Universal Debian Everywhere}
%%%#+END:

%%%#+BEGIN: b:lcnt:pres:frame/derivedImage :title "Just The Universal Debian Everywhere" :subtitle "" :reveal "plain" :beamer "plain" :label "JustDebian" :comment "body=derivedInsert" :derivedLabel "justDebian"
\begin{whenOrg}
*****  _[[elisp:(blee:menu-sel:outline:popupMenu)][±]]_ _[[elisp:(blee:menu-sel:navigation:popupMenu)][Ξ]]_ [[elisp:(outline-show-branches+toggle)][|=]] [[elisp:(bx:orgm:indirectBufOther)][|>]] *[[elisp:(blee:ppmm:org-mode-toggle)][|N]]*  derivedImage [[elisp:(outline-show-subtree+toggle)][||]] Label=JustDebian Just The Universal Debian Everywhere body=derivedInsert
\end{whenOrg}

\begin{htmlonly}

\begin{frame}[fragile,plain,label=JustDebian]
    \frameaudio{"audio/JustDebian.mp3"}
    \frametitle{}
    \framesubtitle{}
    \begin{rawhtml}
<div class="center">
<img src="./disposition.gened/JustDebian/slide-1.png" height="500">
</div>
    \end{rawhtml}
\end{frame}
\end{htmlonly}

\begin{verblatex}

\begin{frame}[fragile,plain,label=JustDebian]
    \frameaudio{"audio/JustDebian.mp3"}
    \frametitle{Just The Universal Debian Everywhere}
    \framesubtitle{}
%%BxPy: impressiveFrameParSet('JustDebian', 'always', 'True')
%%BxPy: impressiveFrameParSet('JustDebian', 'transition', 'UnSpecified')
%%%#+END:

%%%#+BEGIN: bx:dblock:lcnt:body:odg-artpres  :fig-file "/de/lcnt/lgpc/bystar/permanent/common/figures/bxp-layerings.odg"
\begin{comment}
******  [[elisp:(org-cycle)][| ]]  [[elisp:(blee:ppmm:org-mode-toggle)][Nat]] [[elisp:(beginning-of-buffer)][Top]] [[elisp:(delete-other-windows)][(1)]] || /Figure/ =ODG-ArtPres=  *bxp-layerings* -- ByStar Platform Layerings and Software-Service Continuums ::  [[elisp:(org-cycle)][| ]]
\end{comment}

\begin{presentationMode}

\begin{latexonly}
  \begin{figure}
    \begin{center}
       \includegraphics[width=108mm,keepaspectratio]{/de/lcnt/lgpc/bystar/permanent/common/figures/bxp-layerings}
    \end{center}
  \end{figure}
\end{latexonly}

\begin{htmlonly}
  \begin{rawhtml}
<div class="center">
<img src="/de/lcnt/lgpc/bystar/permanent/common/figures/bxp-layerings.png" height="450">
</div>
  \end{rawhtml}
\end{htmlonly}

\end{presentationMode}


\begin{articleMode}

\begin{latexonly}
  \begin{figure}[H]
    \begin{center}
      \includegraphics[width=\textwidth]{/de/lcnt/lgpc/bystar/permanent/common/figures/bxp-layerings}
      \caption{ByStar Platform Layerings and Software-Service Continuums}
      \label{fig:bxp-layerings}
    \end{center}
  \end{figure}
\end{latexonly}

\begin{htmlonly}
  %BEGIN IMAGE
  \begin{center}
      \includegraphics[width=\textwidth]{/de/lcnt/lgpc/bystar/permanent/common/figures/bxp-layerings}
  \end{center}
  %END IMAGE
  %HEVEA\imageflush

  \begin{figure}
      \caption{ByStar Platform Layerings and Software-Service Continuums}
      \label{fig:bxp-layerings}
  \end{figure}
\end{htmlonly}

\end{articleMode}

%%%#+END:


\pnote{

  From the very beginning the Debian folks understood the importance of ``Universality'' and
  coined the ``Universal Debian'' label. This means that we can base our entire digital ecosystem
  on just the Libre-Halaal Debian distro. And that is what we have done with ByStar.
  In ByStar everything is based on just the Universal Debian everywhere.
  This has made our Usage Environment totally harmonious with our Service Environment
  allowing for very powerful software-service continuums.
  Of course, All of this is immediatly applicable to our ByStar Content Bundle as well.

  Some have asked, why don't you also include Ubuntu?

}
\end{frame}
\end{verblatex}


%%%#+BEGIN: bx:dblock:lcnt:latex-section :mode "auto" :seg-title "Nature of Polyexistentials Citations: Convergence of Debian and Ubuntu"
%%% Args: :class "book|pres+art" :langs "en+fa" :disabledP "false" :seg-title "str" :short-title "str" :label "auto"
\begin{whenOrg}
*  _[[elisp:(blee:menu-sel:outline:popupMenu)][±]]_ _[[elisp:(blee:menu-sel:navigation:popupMenu)][Ξ]]_ [[elisp:(outline-show-branches+toggle)][|=]] [[elisp:(bx:orgm:indirectBufOther)][|>]] *[[elisp:(blee:ppmm:org-mode-toggle)][|N]]*  Section    [[elisp:(outline-show-subtree+toggle)][||]]   /Nature of Polyexistentials Citations: Convergence of Debian and Ubuntu/ ::  [[elisp:(org-cycle)][| ]]
\end{whenOrg}

\section{Nature of Polyexistentials Citations: Convergence of Debian and Ubuntu}
%%%#+END:


%%%#+BEGIN: b:lcnt:pres:frame/derivedImage :title "Nature of Polyexistentials Citations:\\\\ Convergence of Debian and Ubuntu" :subtitle "" :label "citation_debuntu" :beamer "plain" :comment "body=text"
\begin{whenOrg}
*****  _[[elisp:(blee:menu-sel:outline:popupMenu)][±]]_ _[[elisp:(blee:menu-sel:navigation:popupMenu)][Ξ]]_ [[elisp:(outline-show-branches+toggle)][|=]] [[elisp:(bx:orgm:indirectBufOther)][|>]] *[[elisp:(blee:ppmm:org-mode-toggle)][|N]]*  derivedImage [[elisp:(outline-show-subtree+toggle)][||]] Label=citation_debuntu Nature of Polyexistentials Citations:\\ Convergence of Debian and Ubuntu body=text
\end{whenOrg}

\begin{htmlonly}

\begin{frame}[fragile,plain,label=citation_debuntu]
    \frameaudio{"audio/citation_debuntu.mp3"}
    \frametitle{}
    \framesubtitle{}
    \begin{rawhtml}
<div class="center">
<img src="./disposition.gened/citation_debuntu/slide-1.png" height="500">
</div>
    \end{rawhtml}
\end{frame}
\end{htmlonly}

\begin{verblatex}

\begin{frame}[fragile,plain,label=citation_debuntu]
    \frameaudio{"audio/citation_debuntu.mp3"}
    \frametitle{Nature of Polyexistentials Citations:\\ Convergence of Debian and Ubuntu}
    \framesubtitle{}
%%BxPy: impressiveFrameParSet('citation_debuntu', 'always', 'True')
%%BxPy: impressiveFrameParSet('citation_debuntu', 'transition', 'UnSpecified')
%%%#+END:

    \begin{center}
      \begin{large}
        \textbf{Chapter 12: Digital Non-Proprietary Movements}
      \end{large}
  \end{center}

    \bigskip

  \begin{large}
Section 12.1.5:\\[10pt] Business and Economics of FOSS
  \end{large}

    \bigskip

[Includes my 2021 email to Mark Shuttleworth towards persuading him to accept Debian as base and to build on Debian --- and not to compete with Debian.]

\pnote{

  I think the opposite makes more sense. Ubuntu should converge with Debian. I tried to explain this to Mark Shuttleworth in an email a while back.
  I have included that email in Section 12.1.5.

}

\end{frame}
\end{verblatex}



%%%#+BEGIN: bx:dblock:lcnt:latex-section :mode "auto" :seg-title "From Raw-BISOS To Your Own Site and Your Own PALS"
%%% Args: :class "book|pres+art" :langs "en+fa" :disabledP "false" :seg-title "str" :short-title "str" :label "auto"
\begin{whenOrg}
*  _[[elisp:(blee:menu-sel:outline:popupMenu)][±]]_ _[[elisp:(blee:menu-sel:navigation:popupMenu)][Ξ]]_ [[elisp:(outline-show-branches+toggle)][|=]] [[elisp:(bx:orgm:indirectBufOther)][|>]] *[[elisp:(blee:ppmm:org-mode-toggle)][|N]]*  Section    [[elisp:(outline-show-subtree+toggle)][||]]   /From Raw-BISOS To Your Own Site and Your Own PALS/ ::  [[elisp:(org-cycle)][| ]]
\end{whenOrg}

\section{From Raw-BISOS To Your Own Site and Your Own PALS}
%%%#+END:


%%%#+BEGIN: b:lcnt:pres:frame/derivedImage :title "" :subtitle "" :label "bisosPlatformProgression" :beamer "plain" :comment "body=derivedInsert"  :derivedLabel "bisosPlatformProgression"
\begin{whenOrg}
*****  _[[elisp:(blee:menu-sel:outline:popupMenu)][±]]_ _[[elisp:(blee:menu-sel:navigation:popupMenu)][Ξ]]_ [[elisp:(outline-show-branches+toggle)][|=]] [[elisp:(bx:orgm:indirectBufOther)][|>]] *[[elisp:(blee:ppmm:org-mode-toggle)][|N]]*  derivedImage [[elisp:(outline-show-subtree+toggle)][||]] Label=bisosPlatformProgression  body=derivedInsert
\end{whenOrg}

\begin{htmlonly}

\begin{frame}[fragile,plain,label=bisosPlatformProgression]
    \frameaudio{"audio/bisosPlatformProgression.mp3"}
    \frametitle{}
    \framesubtitle{}
    \begin{rawhtml}
<div class="center">
<img src="./disposition.gened/bisosPlatformProgression/slide-1.png" height="500">
</div>
    \end{rawhtml}
\end{frame}
\end{htmlonly}

\begin{verblatex}

\begin{frame}[fragile,plain,label=bisosPlatformProgression]
    \frameaudio{"audio/bisosPlatformProgression.mp3"}
    \frametitle{}
    \framesubtitle{}
%%BxPy: impressiveFrameParSet('bisosPlatformProgression', 'always', 'True')
%%BxPy: impressiveFrameParSet('bisosPlatformProgression', 'transition', 'UnSpecified')
%%%#+END:

%%%#+BEGIN: bx:dblock:lcnt:body:odg-artpres  :beamerSize "max" :fig-file "/de/lcnt/lgpc/bystar/permanent/common/figures/bxp-evolution.odg"
\begin{comment}
******  [[elisp:(org-cycle)][| ]]  [[elisp:(blee:ppmm:org-mode-toggle)][Nat]] [[elisp:(beginning-of-buffer)][Top]] [[elisp:(delete-other-windows)][(1)]] || /Figure/ =ODG-ArtPres=  *bxp-evolution* -- Ingidients of BISOS Platforms and Their Progression ::  [[elisp:(org-cycle)][| ]]
\end{comment}

\begin{presentationMode}

\begin{latexonly}
  \begin{figure}
    \begin{center}
       \includegraphics[width=145mm,keepaspectratio]{/de/lcnt/lgpc/bystar/permanent/common/figures/bxp-evolution}
    \end{center}
  \end{figure}
\end{latexonly}

\begin{htmlonly}
  \begin{rawhtml}
<div class="center">
<img src="/de/lcnt/lgpc/bystar/permanent/common/figures/bxp-evolution.png" height="450">
</div>
  \end{rawhtml}
\end{htmlonly}

\end{presentationMode}


\begin{articleMode}

\begin{latexonly}
  \begin{figure}[H]
    \begin{center}
      \includegraphics[width=\textwidth]{/de/lcnt/lgpc/bystar/permanent/common/figures/bxp-evolution}
      \caption{Ingidients of BISOS Platforms and Their Progression}
      \label{fig:bxp-evolution}
    \end{center}
  \end{figure}
\end{latexonly}

\begin{htmlonly}
  %BEGIN IMAGE
  \begin{center}
      \includegraphics[width=\textwidth]{/de/lcnt/lgpc/bystar/permanent/common/figures/bxp-evolution}
  \end{center}
  %END IMAGE
  %HEVEA\imageflush

  \begin{figure}
      \caption{Ingidients of BISOS Platforms and Their Progression}
      \label{fig:bxp-evolution}
  \end{figure}
\end{htmlonly}

\end{articleMode}

%%%#+END:

\pnote{

  In this presentation, we have stopped at the ``Raw-BISOS'' stage. We can further evolve
  Raw-BISOS and make it be ``Sited'' and provide autonomous publication services.

  But here by going through EmacsConf and youtube we are using the
  ``Federated Re-Publications'' model.

}
\end{frame}
\end{verblatex}



%%%#+BEGIN: bx:dblock:lcnt:latex-part :disabledP "false" :seg-title "Blee-Panels --- BISOS and Blee's Self-Documentation Facilities"
%%% Args: :toc "NU" :tocDepth 3 :part "NU" :label "auto|spec" :partpage t
\begin{whenOrg}
*      ================
*  [[elisp:(blee:ppmm:org-mode-toggle)][|n]] [[elisp:(blee:menu-sel:outline:popupMenu)][+-]] [[elisp:(blee:menu-sel:navigation:popupMenu)][==]]  *Part*   _Blee-Panels --- BISOS and Blee's Self-Documentation Facilities_ ::  [[elisp:(org-cycle)][| ]]
\end{whenOrg}

\newpage
\part{Blee-Panels --- BISOS and Blee's Self-Documentation Facilities}
%%%#+END:

%%%#+BEGIN: b:lcnt:pres:frame/derivedImage :title "Blee-Panels --- BISOS and Blee's Self-Documentation Facilities" :subtitle "" :label "part_bleePanels" :comment "body=text" :reveal "plain" :beamer ""
\begin{whenOrg}
*****  _[[elisp:(blee:menu-sel:outline:popupMenu)][±]]_ _[[elisp:(blee:menu-sel:navigation:popupMenu)][Ξ]]_ [[elisp:(outline-show-branches+toggle)][|=]] [[elisp:(bx:orgm:indirectBufOther)][|>]] *[[elisp:(blee:ppmm:org-mode-toggle)][|N]]*  derivedImage [[elisp:(outline-show-subtree+toggle)][||]] Label=part_bleePanels Blee-Panels --- BISOS and Blee's Self-Documentation Facilities body=text
\end{whenOrg}

\begin{htmlonly}

\begin{frame}[fragile,plain,label=part_bleePanels]
    \frameaudio{"audio/part_bleePanels.mp3"}
    \frametitle{}
    \framesubtitle{}
    \begin{rawhtml}
<div class="center">
<img src="./disposition.gened/part_bleePanels/slide-1.png" height="500">
</div>
    \end{rawhtml}
\end{frame}
\end{htmlonly}

\begin{verblatex}

\begin{frame}[fragile,label=part_bleePanels]
    \frameaudio{"audio/part_bleePanels.mp3"}
    \frametitle{Blee-Panels --- BISOS and Blee's Self-Documentation Facilities}
    \framesubtitle{}
%%BxPy: impressiveFrameParSet('part_bleePanels', 'always', 'True')
%%BxPy: impressiveFrameParSet('part_bleePanels', 'transition', 'UnSpecified')
%%%#+END:

    \tableofcontents

\pnote{

  Something this large, should be well documented.

  }
\end{frame}
\end{verblatex}


%%%#+BEGIN: bx:dblock:lcnt:latex-section :mode "auto" :seg-title "Content Processing Blee Panels --- Screen Captures"
%%% Args: :class "book|pres+art" :langs "en+fa" :disabledP "false" :seg-title "str" :short-title "str" :label "auto"
\begin{whenOrg}
*  _[[elisp:(blee:menu-sel:outline:popupMenu)][±]]_ _[[elisp:(blee:menu-sel:navigation:popupMenu)][Ξ]]_ [[elisp:(outline-show-branches+toggle)][|=]] [[elisp:(bx:orgm:indirectBufOther)][|>]] *[[elisp:(blee:ppmm:org-mode-toggle)][|N]]*  Section    [[elisp:(outline-show-subtree+toggle)][||]]   /Content Processing Blee Panels --- Screen Captures/ ::  [[elisp:(org-cycle)][| ]]
\end{whenOrg}

\section{Content Processing Blee Panels --- Screen Captures}
%%%#+END:

%%%#+BEGIN: b:lcnt:pres:frame/derivedVideo :title "Content Processing Blee Panels" :subtitle "" :label "lcntBleePanelsVideo" :comment "body=itemize" :reveal "plain" :beamer "plain"
\begin{whenOrg}
*****  _[[elisp:(blee:menu-sel:outline:popupMenu)][±]]_ _[[elisp:(blee:menu-sel:navigation:popupMenu)][Ξ]]_ [[elisp:(outline-show-branches+toggle)][|=]] [[elisp:(bx:orgm:indirectBufOther)][|>]] *[[elisp:(blee:ppmm:org-mode-toggle)][|N]]*  derivedVideo [[elisp:(outline-show-subtree+toggle)][||]] Label=lcntBleePanelsVideo Content Processing Blee Panels body=itemize
\end{whenOrg}

\begin{htmlonly}

\begin{frame}[fragile,plain,label=lcntBleePanelsVideo]
    \frametitle{}
    \framesubtitle{}
    \begin{presentationMode}
    \begin{rawhtml}
<video preload="auto" data-audio-controls src="./video/derived-lcntBleePanelsVideo.mp4"></video>
    \end{rawhtml}
    \end{presentationMode}

    \begin{articleMode}
    \begin{rawhtml}
    <!-- data-autoplay  controls -->
    <p>
    <video controls  preload="auto" src="./video/derived-lcntBleePanelsVideo.mp4" height="50%%" width="50%%">
    </video>
    </p>
     \end{rawhtml}
    \end{articleMode}
\end{frame}
\end{htmlonly}

\begin{verblatex}

\begin{frame}[fragile,plain,label=lcntBleePanelsVideo]
    \frameaudio{"audio/lcntBleePanelsVideo.mp3"}
    \frametitle{Content Processing Blee Panels}
    \framesubtitle{}
%%BxPy: impressiveFrameParSet('lcntBleePanelsVideo', 'always', 'True')
%%BxPy: impressiveFrameParSet('lcntBleePanelsVideo', 'transition', 'UnSpecified')
%%%#+END:


    \begin{center}
    Video of tour of Blee-Lcnt Panels comes here.
    \end{center}

  %%% Let's walk through Blee-Lcnt panels, with two goals in mind. First to show what documentation is available for Blee-Lcnt
  %%% and second what capabilities the Panels provide.


\pnote{

  In Emacs, the way that we have been dealing with documentation and information retrieval
  is archaic. Man-pages, TeXInfo, Helpful-Mode and convention based Doc-Strings are old
  and limited.

  In BISOS and Blee, we use Blee-Panels for all kinds of documentation.

  Let me show you some examples.

}

\end{frame}
\end{verblatex}


%%%#+BEGIN: bx:dblock:lcnt:latex-part :disabledP "false" :seg-title "Moving Forward"
%%% Args: :toc "NU" :tocDepth 3 :part "NU" :label "auto|spec" :partpage t
\begin{whenOrg}
*      ================
*  [[elisp:(blee:ppmm:org-mode-toggle)][|n]] [[elisp:(blee:menu-sel:outline:popupMenu)][+-]] [[elisp:(blee:menu-sel:navigation:popupMenu)][==]]  *Part*   _Moving Forward_ ::  [[elisp:(org-cycle)][| ]]
\end{whenOrg}

\newpage
\part{Moving Forward}
%%%#+END:


%%%#+BEGIN: b:lcnt:pres:frame/derivedImage :title "Moving Forward" :subtitle "" :label "part_movingForward" :comment "body=text" :reveal "plain" :beamer ""
\begin{whenOrg}
*****  _[[elisp:(blee:menu-sel:outline:popupMenu)][±]]_ _[[elisp:(blee:menu-sel:navigation:popupMenu)][Ξ]]_ [[elisp:(outline-show-branches+toggle)][|=]] [[elisp:(bx:orgm:indirectBufOther)][|>]] *[[elisp:(blee:ppmm:org-mode-toggle)][|N]]*  derivedImage [[elisp:(outline-show-subtree+toggle)][||]] Label=part_movingForward Moving Forward body=text
\end{whenOrg}

\begin{htmlonly}

\begin{frame}[fragile,plain,label=part_movingForward]
    \frameaudio{"audio/part_movingForward.mp3"}
    \frametitle{}
    \framesubtitle{}
    \begin{rawhtml}
<div class="center">
<img src="./disposition.gened/part_movingForward/slide-1.png" height="500">
</div>
    \end{rawhtml}
\end{frame}
\end{htmlonly}

\begin{verblatex}

\begin{frame}[fragile,label=part_movingForward]
    \frameaudio{"audio/part_movingForward.mp3"}
    \frametitle{Moving Forward}
    \framesubtitle{}
%%BxPy: impressiveFrameParSet('part_movingForward', 'always', 'True')
%%BxPy: impressiveFrameParSet('part_movingForward', 'transition', 'UnSpecified')
%%%#+END:

    \tableofcontents

\pnote{

  So, what next?

}
\end{frame}
\end{verblatex}


%%%#+BEGIN: bx:dblock:lcnt:latex-section :mode "auto" :seg-title "Your Participation"
%%% Args: :class "book|pres+art" :langs "en+fa" :disabledP "false" :seg-title "str" :short-title "str" :label "auto"
\begin{whenOrg}
*  _[[elisp:(blee:menu-sel:outline:popupMenu)][±]]_ _[[elisp:(blee:menu-sel:navigation:popupMenu)][Ξ]]_ [[elisp:(outline-show-branches+toggle)][|=]] [[elisp:(bx:orgm:indirectBufOther)][|>]] *[[elisp:(blee:ppmm:org-mode-toggle)][|N]]*  Section    [[elisp:(outline-show-subtree+toggle)][||]]   /Your Participation/ ::  [[elisp:(org-cycle)][| ]]
\end{whenOrg}

\section{Your Participation}
%%%#+END:

%%%#+BEGIN: b:lcnt:pres:frame/derivedImage :title "Your Participation" :subtitle "" :label "yourParticipation" :comment "body=itemize" :reveal "plain" :beamer "plain"
\begin{whenOrg}
*****  _[[elisp:(blee:menu-sel:outline:popupMenu)][±]]_ _[[elisp:(blee:menu-sel:navigation:popupMenu)][Ξ]]_ [[elisp:(outline-show-branches+toggle)][|=]] [[elisp:(bx:orgm:indirectBufOther)][|>]] *[[elisp:(blee:ppmm:org-mode-toggle)][|N]]*  derivedImage [[elisp:(outline-show-subtree+toggle)][||]] Label=yourParticipation Your Participation body=itemize
\end{whenOrg}

\begin{htmlonly}

\begin{frame}[fragile,plain,label=yourParticipation]
    \frameaudio{"audio/yourParticipation.mp3"}
    \frametitle{}
    \framesubtitle{}
    \begin{rawhtml}
<div class="center">
<img src="./disposition.gened/yourParticipation/slide-1.png" height="500">
</div>
    \end{rawhtml}
\end{frame}
\end{htmlonly}

\begin{verblatex}

\begin{frame}[fragile,plain,label=yourParticipation]
    \frameaudio{"audio/yourParticipation.mp3"}
    \frametitle{Your Participation}
    \framesubtitle{}
%%BxPy: impressiveFrameParSet('yourParticipation', 'always', 'True')
%%BxPy: impressiveFrameParSet('yourParticipation', 'transition', 'UnSpecified')
%%%#+END:


    \begin{columns}

      \begin{column}{0.5\textwidth}

        \begin{center}
          \colorbox{blue!30}{\color{black} \textbf{General Audience - Emacs Users}}
        \end{center}
      \end{column}

      \begin{column}{0.5\textwidth}

        \begin{center}
          \colorbox{blue!60}{\color{white} \textbf{Emacs Developers}}
        \end{center}
      \end{column}

    \end{columns}

    \bigskip

    \begin{columns}

      \begin{column}{0.5\textwidth}

        \begin{itemize}
          \item Read ``Nature of Polyexistentials''
          \item Evaluate Invalidity of IPR Regime
          \item Evaluate Merits of\\ Libre-Halaal Label vs FOSS
          \item Always Apply Affero GPL to\\ Your Own Software
          \item Eastern Societies: Engage Policy Makers for Abolishment of IPR
          \item Appraise the ByStar Blueprint
        \end{itemize}
      \end{column}

      \begin{column}{0.5\textwidth}

        \begin{itemize}
          \item Expand Scope of Org Dynamic Blocks to the Entirety of Emacs
          \item Create a tree-sitter Polymode with good org-mode support
          \item Make Something Like Poly-COMEEGA Emacs Native
          \item Make Something Like Blee-Panels Org or Emacs Native
        \end{itemize}
      \end{column}

    \end{columns}



\pnote{

  If Blee, BISOS, ByStar, Libre-Halaal, Polyexistentials and these
  Content Processing capabilities have piqued your interest,
  please feel welcome to contact me.


}

\end{frame}
\end{verblatex}



%%%#+BEGIN: bx:dblock:lcnt:latex-section :mode "auto" :seg-title "Emacs Conf is Great"
%%% Args: :class "book|pres+art" :langs "en+fa" :disabledP "false" :seg-title "str" :short-title "str" :label "auto"
\begin{whenOrg}
*  _[[elisp:(blee:menu-sel:outline:popupMenu)][±]]_ _[[elisp:(blee:menu-sel:navigation:popupMenu)][Ξ]]_ [[elisp:(outline-show-branches+toggle)][|=]] [[elisp:(bx:orgm:indirectBufOther)][|>]] *[[elisp:(blee:ppmm:org-mode-toggle)][|N]]*  Section    [[elisp:(outline-show-subtree+toggle)][||]]   /Emacs Conf is Great/ ::  [[elisp:(org-cycle)][| ]]
\end{whenOrg}

\section{Emacs Conf is Great}
%%%#+END:

%%%#+BEGIN: b:lcnt:pres:frame/derivedImage :title "Thanks" :subtitle "" :label "emacsConfIsGreat" :comment "body=itemize" :reveal "plain" :beamer ""
\begin{whenOrg}
*****  _[[elisp:(blee:menu-sel:outline:popupMenu)][±]]_ _[[elisp:(blee:menu-sel:navigation:popupMenu)][Ξ]]_ [[elisp:(outline-show-branches+toggle)][|=]] [[elisp:(bx:orgm:indirectBufOther)][|>]] *[[elisp:(blee:ppmm:org-mode-toggle)][|N]]*  derivedImage [[elisp:(outline-show-subtree+toggle)][||]] Label=emacsConfIsGreat Thanks body=itemize
\end{whenOrg}

\begin{htmlonly}

\begin{frame}[fragile,plain,label=emacsConfIsGreat]
    \frameaudio{"audio/emacsConfIsGreat.mp3"}
    \frametitle{}
    \framesubtitle{}
    \begin{rawhtml}
<div class="center">
<img src="./disposition.gened/emacsConfIsGreat/slide-1.png" height="500">
</div>
    \end{rawhtml}
\end{frame}
\end{htmlonly}

\begin{verblatex}

\begin{frame}[fragile,label=emacsConfIsGreat]
    \frameaudio{"audio/emacsConfIsGreat.mp3"}
    \frametitle{Thanks}
    \framesubtitle{}
%%BxPy: impressiveFrameParSet('emacsConfIsGreat', 'always', 'True')
%%BxPy: impressiveFrameParSet('emacsConfIsGreat', 'transition', 'UnSpecified')
%%%#+END:


    \begin{exampleblock}{With Big Thanks To All EmacsConf Organizers}
        And Sacha In Particular
    \end{exampleblock}


\pnote{

These Emacs Conferences have proven to be very useful and productive.
I look forward to your thoughts, feedback and questions.

}

\end{frame}
\end{verblatex}


%%%#+BEGIN: bx:dblock:lcnt:latex-section :disabledP "false" :seg-title "Thanks"
%%% Args: :class "book|pres+art" :langs "en+fa" :disabledP "false" :seg-title "str" :short-title "str" :label "auto"
\begin{whenOrg}
*  _[[elisp:(blee:menu-sel:outline:popupMenu)][±]]_ _[[elisp:(blee:menu-sel:navigation:popupMenu)][Ξ]]_ [[elisp:(outline-show-branches+toggle)][|=]] [[elisp:(bx:orgm:indirectBufOther)][|>]] *[[elisp:(blee:ppmm:org-mode-toggle)][|N]]*  Section    [[elisp:(outline-show-subtree+toggle)][||]]   /Thanks/ ::  [[elisp:(org-cycle)][| ]]
\end{whenOrg}

\section{Thanks}
%%%#+END:


%%%#+BEGIN: b:lcnt:pres:frame:begin/blank  :label "endVideo" :comment "body=video"
\begin{whenOrg}
*****  _[[elisp:(blee:menu-sel:outline:popupMenu)][±]]_ _[[elisp:(blee:menu-sel:navigation:popupMenu)][Ξ]]_ [[elisp:(outline-show-branches+toggle)][|=]] [[elisp:(bx:orgm:indirectBufOther)][|>]] *[[elisp:(blee:ppmm:org-mode-toggle)][|N]]*  /Frame:begin-blank/ [[elisp:(outline-show-subtree+toggle)][||]] *Label=endVideo* UnSpecified -- body=video
\end{whenOrg}

\begin{frame}[fragile,plain,label=endVideo]
    \frametitle{}
%%BxPy: impressiveFrameParSet('endVideo', 'always', 'True')
%%BxPy: impressiveFrameParSet('endVideo', 'transition', 'UnSpecified')
%%%#+END:

%%%#+BEGIN: b:lcnt:pres:frame:body:mm/video :videoPath "./video/endVideo.mp4" :comment ""
\begin{whenOrg}
******  _[[elisp:(blee:menu-sel:outline:popupMenu)][±]]_ _[[elisp:(blee:menu-sel:navigation:popupMenu)][Ξ]]_ [[elisp:(outline-show-branches+toggle)][|=]] [[elisp:(bx:orgm:indirectBufOther)][|>]] *[[elisp:(blee:ppmm:org-mode-toggle)][|N]]*  FrmCntnt-Video [[elisp:(outline-show-subtree+toggle)][||]] Label=UnSpecified UnSpecified
\end{whenOrg}
\begin{presentationMode}
\begin{htmlonly}
  \begin{rawhtml}
<video preload="auto" data-audio-controls src="./video/endVideo.mp4"></video>
  \end{rawhtml}
\end{htmlonly}
\end{presentationMode}

\begin{articleMode}
\begin{htmlonly}
  \begin{rawhtml}
      <!-- data-autoplay  controls -->
    <p>
    <video  controls   preload="auto"  src="./video/endVideo.mp4"  height="50%%" width="50%%">
    </video>
    </p>
  \end{rawhtml}
\end{htmlonly}
\end{articleMode}

\begin{presentationMode}
\begin{latexonly}
    \begin{center}
      Video File: ./video/endVideo.mp4
    \end{center}
\end{latexonly}
\end{presentationMode}

\begin{articleMode}
\begin{latexonly}
    \begin{center}
      Video File: ./video/endVideo.mp4
    \end{center}
\end{latexonly}
\end{articleMode}
%%%#+END:

\pnote{

  I want to thank all the  EmacsConf 2025 Organizers
  for their great work.
And Sacha in particular.

}

\end{frame}

\begin{comment}
*  [[elisp:(org-cycle)][| ]]  Local Vars  ::                  *Org-Mode And Emacs Specific Configurations*   [[elisp:(org-cycle)][| ]]
\end{comment}

%% Local Variables:
%% major-mode: latex-mode
%% fill-column: 90
%% End:
